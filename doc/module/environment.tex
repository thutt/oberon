\chap{Environment}

This module provides access to the host OS' shell environment
variables.  Environment variables can be changed within the Oberon
environment, but any change made will not be reflected into the shell
environment from which the Oberon program was launched.  Moreover,
changes to the shell environment made after the Oberon program was
launched will not be reflected in Oberon.

These variables are key-value pairs specified in the shell
environment that can be used to configure runtime aspects of programs.
For example, a default directory for file output, or a search path for
program executables.

As a key-value pair, the shell environment variables have the
following form:

\begin{alltt}
  Key     := name
  Value   := element \{ ':' element \}

  Name    := letter \{ letter | digit | '_' | \}
  element := <ASCII characters except ':'>
\end{alltt}

\section{Source}

\begin{tabularx}{\textwidth}{lX}
  Source & \texttt{Environment.Mod} \\
  Test & \texttt{CTEnvironment.Mod} \\
\end{tabularx}

\section{Constants}

There are no constants exported from this module.

\section{Types}
\subsection{\texttt{Text}}\label{environment:text}
\begin{alltt}
Text = POINTER TO ARRAY OF CHAR;
\end{alltt}

\begin{invariant}
  \asciizfn{Text\deref}
\end{invariant}

\begin{semantics}
  This type is used to refer to data that is stored in an internal
  \emph{key-value} cache.  The unprocessed data was returned from the
  host OS' when querying the value of an \emph{environment variable}.

  If the data contains multiple value, such as directory names, it
  must be split using \texttt{Split} (\xref{environment:split}) to
  gain access to individual values.
\end{semantics}


\subsection{\texttt{Elements}}\label{environment:elements}
\begin{alltt}
Elements = POINTER TO ARRAY OF Text;
\end{alltt}

\begin{invariant}
  (\forall i: 0 \leq i < \texttt{LEN(Elements\deref)}:
    \texttt{Elements[}i\texttt{]} \neq \nil \logicaland \valid(\texttt{Elements[}i\texttt{]\deref}))
\end{invariant}

\begin{semantics}
  Data of this type is produced by using \texttt{Split}
  (\xref{environment:split}) to break down raw OS environment data
  into an accessible data structure.
\end{semantics}


\section{Procedures}

\subsection{Delete}\label{environment:delete}
\begin{alltt}
  PROCEDURE Delete*(key : ARRAY OF CHAR);
\end{alltt}

\begin{precondition}
  ASCIIZ(\texttt{key})
\end{precondition}

\begin{semantics}
  This procedure effectively removes \emph{key} from the internal data
  cache.  Subsequent uses of \texttt{Lookup}
  (\xref{environment:lookup}) for \texttt{key} will return \nil until
  a new value is set.  A new value can be set using \texttt{Set}
  (\xref{environment:set}).
\end{semantics}

\begin{postcondition}
  \texttt{Environment.Lookup(key)} = \nil
\end{postcondition}


\subsection{Set}\label{environment:set}
\begin{alltt}
  PROCEDURE Set*(key   : ARRAY OF CHAR;
                 value : ARRAY OF CHAR);
\end{alltt}

\begin{precondition}
  ASCIIZ(\texttt{key}) \logicaland ASCIIZ(\texttt{value})
\end{precondition}

\begin{semantics}
This procedure sets \emph{key} to \emph{value} in the internal data
cache.  If \emph{key} does not exist in the internal data cache, it is
added.  If \emph{key} exists, it is set to \emph{value} and the
previous value is discarded.
\end{semantics}

\begin{postcondition}
  \texttt{Environment.Lookup(key)\deref} = \texttt{value}
\end{postcondition}


\subsection{Lookup}\label{environment:lookup}
\begin{alltt}
  PROCEDURE Lookup*(key : ARRAY OF CHAR) : Text;
\end{alltt}

\begin{precondition}
  ASCIIZ(\texttt{key})
\end{precondition}

\begin{semantics}
This procedure looks up \emph{key} in the internal data cache.  If
\emph{key} is present, its value is returned.  If \emph{key} is not
present, \nil is returned.
\end{semantics}

\begin{postcondition}
  \result = \nil \logicalor \validfn{\result\deref}
\end{postcondition}


\subsection{Split}\label{environment:split}
\begin{alltt}
  PROCEDURE Split*(v  : Text;
                   ch : CHAR) : Elements;
\end{alltt}

\begin{precondition}
  \texttt{v} = \nil \logicalor \validfn{v\deref}
\end{precondition}

\begin{semantics}
This procedure splits the input, \emph{v}, using \emph{ch} as the
character separating the multiple choices of the environment variable.

If \emph{v} is \texttt{NIL}, \texttt{NIL} is returned.

Otherwise, \result is an array containing the choices present in the
variable.

The test module provides examples of how this function can be used.
\end{semantics}

\begin{postcondition}
  \result = \nil \logicalor \validfn{\result\deref}
\end{postcondition}

