\chap{GetOpt}

The \texttt{GetOpt} module provides a uniform interface for processing
command line arguments with this Oberon implementation.

There are three types of command line arguments, shown in
\figref{GetOpt:options-example}, that can occur:

\begin{enumerate}
\item Program Option

  A \emph{program option} is a command line argument that is supplied
  by the user to modify the default behavior of a program.  For
  example, the interpreter of this Oberon system allows \emph{program
    options} to control, for example, \emph{diagnostic output}
  (\texttt{--diagnostic}) and printing the number of instructions
  executed at the end of a run (\texttt{--instruction-count}).

  There are two types of \emph{program options} that are allowed:

  \begin{enumerate}
  \item Short\label{getopt:short-option}

    A \emph{short} option is a single '-', followed by a single
    character.  For example, \texttt{-l}.

  \item Long

    A \emph{long} option is two (2) '-', followed by two (2) or more
    characters.  For example, \texttt{--instruction-count}.

    The maximum number of characters is limited by the maximal array
    index of this implementation of Oberon.

  \end{enumerate}

\item Program Option Argument

  Sometimes a \emph{program option}, like the specification of an
  output file pathname, will require an argument of its own.  A
  \emph{program options}'s argument is referred to as a \emph{program
    option argument}.

\item Program Argument

  The final type of argument is, a \emph{program argument}, is defined
  as not being a \emph{program option} nor being a \emph{program
    option argument}.  In the case of this Oberon system, an example
  of a \emph{program argument} is the name of command to execute on
  the command line.
\end{enumerate}

Here is a sample that shows all of the types of supported arguments.

\begin{figure}[h]
  \begin{alltt}
    skl-oberon --  \allttbackslash
    CTGetOpt.Test  \allttbackslash    # Program argument
    --symbols /tmp \allttbackslash    # Long option & argument
    alpha          \allttbackslash    # Program argument
    -v             \allttbackslash    # Short option
    beta           \allttbackslash    # Program argument
    gamma               # Program argument
  \end{alltt}
  \caption{Different types of program arguments}\label{fig:GetOpt:options-example}
\end{figure}




\section{Constants}
\subsection{Status Values}\label{GetOpt:status-values}
The symbols documented in this section to denote the \emph{status} for
each command line argument processed by \xrefsym{GetOpt}{Parse}.

\begin{tabularx}{\linewidth}{l|X}
  Symbol & Meaning \\
  \hline

  \texttt{Done} & The \texttt{Done} status indicates that there are no
  more command line arguments to process. \\


  \texttt{Error} & The \texttt{Error} status indicates that an generic
  error occurred processing the next command line argument.  In some
  cases, a message will be written to the console that describes the
  error. \\


  \texttt{NotFound} & The \texttt{NotFound} status indicates that a
  properly formed option was found on the command line, but the option
  has not been configured in the current \xrefsym{GetOpt}{Handle}. \\


  \texttt{Success}\label{GetOpt:Success} & The \texttt{Success} status
  indicates that \texttt{Parse} has successfully processed a command
  line argument.
\end{tabularx}

\subsection{Option Creation Flags}\label{GetOpt:flags}

The flags described in this section are used when creating a command
line argument with \xrefsym{GetOpt}{AddOption}.


\begin{tabularx}{\linewidth}{l|X}\label{GetOpt:ValueNeeded}
  Symbol & Meaning \\
  \hline

  \texttt{ValueNeeded} & This flag indicates that the option requires
  an argument.  At runtime, the immediately following command line
  argument follow will be used.

  If an option that specifies \texttt{ValueNeeded} is not followed by
  an argument, an error will result.

\end{tabularx}


\subsection{Argument Creation}\label{GetOpt:creation}

Each option added with \xrefsym{GetOpt}{AddOption} must be assigned a
numerical value to differentiate it from all other options.  The
constants described below parameterize the creation, and processing of
options provided to a program.

\begin{tabularx}{\linewidth}{l|X}
  Symbol & Meaning \\
  \hline

  \texttt{ValueArgument} & This symbol signifies that the command line
  argument recognized by \xrefsym{GetOpt}{Parse} is a \emph{program
    argument}.

  A successfully parsed argument with this value set is a regular
  program argument, not an option.

  See \xrefsym{GetOpt}{ArgumentDesc} and \xrefsym{GetOpt}{Parse}. \\


  \texttt{ValueHelp}\label{GetOpt:ValueHelp} & The \texttt{ValueHelp}
  symbol is reserved for use by the \texttt{GetOpt} module.  It is
  used to identify the programmatic value of the automatically-added
  \texttt{--help} (and \texttt{-h}) argument.

  If this is used when creating an option with
  \xrefsym{GetOpt}{AddOption}, an error will result. \\

  \texttt{ValueMin} & Each option must be assigned a unique numerical
  value for greater or equal to \texttt{ValueMin}. \\

  \texttt{ValueMax} & Each option must be assigned a unique numerical
  value for identification less than or equal to \texttt{ValueMax}.
\end{tabularx}

\section{Types}

\subsection{\texttt{ArgumentDesc}}\label{GetOpt:ArgumentDesc}
\begin{alltt}
  ArgumentDesc* = RECORD
    value-    : INTEGER;
    argument- : CommandLine.Parameter;
  END;
\end{alltt}

This type describes a fully parsed command line argument when
\xrefsym{GetOpt}{Parse} returns \xrefsym{GetOpt}{Success}.

\begin{tabularx}{\linewidth}{l|X}
  Field & Description \\

  \hline \texttt{value} & This field holds the \texttt{value} argument
  provided to \xrefsym{GetOpt}{AddOption} for the matched command line
  option. \\

  \texttt{argument} & This will contain the option's argument value
  when \xrefsym{GetOpt}{ValueNeeded} was specified in the call to
  \xrefsym{GetOpt}{AddOption}. \\
\end{tabularx}

\begin{invariant}
  (\texttt{value} = \texttt{ValueHelp}) \logicalor
  (\texttt{value} >= \texttt{ValueMin} \logicaland \texttt{value} <= \texttt{ValueMax})
  \logicaland \\
  ((\texttt{argument} = \nil) \logicalor \validfn{argument\deref})
\end{invariant}


\subsection{\texttt{Handle}}\label{GetOpt:Handle}
\begin{alltt}
  Handle* = POINTER TO HandleDesc;
\end{alltt}

\texttt{Handle} is a pointer to an opaque type that implements the
data structure maintaining the state of a program's command line
options.

\subsection{\texttt{Status}}\label{GetOpt:Status}
\begin{alltt}
  Status* = SHORTINT;
\end{alltt}

A \emph{status} value is returned by \xrefsym{GetOpt}{Parse}, and it
can obtain one of the values described in
\xref{GetOpt}{status-values}.


\section{Procedures}
\subsection{\texttt{AddOption}}\label{GetOpt:AddOption}
\begin{alltt}
PROCEDURE AddOption*(h     : Handle;
                     value : INTEGER;
                     flags : SET;
                     short : CHAR;
                     name  : ARRAY OF CHAR;
                     help  : ARRAY OF CHAR) : BOOLEAN;
\end{alltt}

\begin{semantics}
This procedure is used to add options to the \texttt{Handle},
\texttt{h}.

The \texttt{value} argument sets the numerical value when the option
is matched by \xrefsym{GetOpt}{Parse}.

The \texttt{flags} argument allows specialization of the option.
Including the \texttt{ValueNeeded} flag causes the option to require
an argument.  If the argument is not supplied following the option,
then \texttt{Parse} will return an error for that option.  If
\texttt{ValueNeeded} is not included, then option may not have an
argument.

The \texttt{short} argument specifies the \ascii character to be used
for the short name of this option.  If no short argument is desired,
use \texttt{0X}.  See \xref{getopt:short-option} on
\xpageref{getopt:short-option}.

The \texttt{name} argument specifies the long name of this argument.
A long name is required.

The \texttt{help} argument contains the \emph{help} text that will
displayed when \xrefsym{GetOpt}{Help} is invoked.

All of \texttt{value}, \texttt{short}, and \texttt{name} must be
unique to \texttt{h}.

If no \emph{short option} is desired, \texttt{short} should be set to
\texttt{0X}.

\end{semantics}

\begin{precondition}
(\texttt{h} \neq \nil \logicaland \validfn{\texttt{h}\deref} \logicaland \\
  (\texttt{value} \geq \texttt{ValueMin}) \logicaland \\
  (\texttt{value} \leq \texttt{ValueMax}) \logicaland \\
  (\texttt{value} \textrm{ not already used by another option}) \logicaland \\
  (flags - \{\texttt{ValueNeeded}\} = \{\}) \logicaland \\
  (\texttt{LEN(name)} > 2 \logicaland
  \texttt{name[0]} \neq \texttt{'-'} \logicaland
  \texttt{name[1]} \neq \texttt{'-'}) \logicaland \\
  (\exists i: 0 <= i < \texttt{LEN(name)}: \texttt{name[}i\texttt{] = 0X}) \logicaland \\
  (\exists i: 0 <= i < \texttt{LEN(help)}: \texttt{help[}i\texttt{] = 0X})  \logicaland \\
  (\texttt{short} = \texttt{0X} \logicalor \texttt{short} \textrm{ not
    already used by another option}) \logicaland \\
  (\texttt{name} \textrm{ not already used by another option})
\end{precondition}

\begin{postcondition}
  (\result \implies \textrm{Option added}) \logicalor \\
  (\logicalnot \result \implies \textrm{Option not added})
\end{postcondition}

\subsection{\texttt{Create}}\label{GetOpt:Create}
\begin{alltt}
PROCEDURE Create*() : Handle;
\end{alltt}

\begin{semantics}
This procedure creates and initializes a xrefsym{GetOpt}{Handle} that
can be used by other procedures in this module.
\end{semantics}

\begin{precondition}
\texttt{None}
\end{precondition}

\begin{postcondition}
  \result = \nil \logicalor \validfn{\result\deref}
\end{postcondition}

\subsection{\texttt{Help}}\label{GetOpt:Help}
\begin{alltt}
PROCEDURE Help*(h : Handle);
\end{alltt}

\begin{semantics}
When invoked, this procedure writes the set of all arguments and their
help strings to the console.  It is the responsibility of the user of
this module to invoke this procedure when the
\xrefsym{GetOpt}{ValueHelp} \emph{status} is returned by
\xrefsym{GetOpt}{Parse}.
\end{semantics}

\begin{precondition}
\texttt{h} \neq \nil \logicaland \valid(\texttt{h}\deref)
\end{precondition}

\begin{postcondition}
\texttt{None}
\end{postcondition}

\subsection{\texttt{}}\label{GetOpt:Parse}
\begin{alltt}
PROCEDURE Parse*(    h        : Handle;
                 VAR argument : ArgumentDesc) : Status;
\end{alltt}

\begin{semantics}
This procedure is called repeatedly to process all command line
arguments in order of appearance on the command line.  It returns
a \emph{status} value indicating the state of the procedure call.  See
table \figref{GetOpt:Parse} for semantics of each result.

\begin{figure}[h]
  \begin{tabularx}{\linewidth}{lX}
    \texttt{Status} & Semantics \\
    \hline

    \texttt{Success} & A command line argument has been successfully
    found.  The contents of \texttt{argument}
    (\xrefsym{GetOpt}{ArgumentDesc}) describe the argument that was
    recognized. \\

    \texttt{NotFound} & A short or long form command line option was
    found on the command line that did not match any of the options
    added to the \xrefsym{GetOpt}{Handle}.

    \xrefsym{GetOpt}{AddOption} can be used to add options to the
    \xrefsym{GetOpt}{Handle}. \\

    \texttt{Error} & An error occurred processing a command line
    argument. When this value is returned, a brief message describing
    the error will also be written to the console.  It is up to the
    user to fix the erroneous command line.  \\

    \texttt{Done} & All command line arguments have been processed. \\
  \end{tabularx}
  \caption{Semantics of \texttt{Parse} result}\label{fig:GetOpt:Parse}
\end{figure}

\end{semantics}

\begin{precondition}
\texttt{h} \neq \nil \logicaland \valid(\texttt{h}\deref)
\end{precondition}

\begin{postcondition}
(\result \in \{ \texttt{Success}, \texttt{NotFound}, \texttt{Error},
  \texttt{Done} \} \logicaland \\
  (\result  = \texttt{Success} \implies \validfn{argument})
  \logicaland \\
  (\result  = \texttt{NotFound} \implies \logicalnot \validfn{argument})
  \logicaland \\
  (\result  = \texttt{Error} \implies \logicalnot \validfn{argument})
  \logicaland \\
  (\result  = \texttt{Done} \implies \logicalnot \validfn{argument}))
\end{postcondition}
