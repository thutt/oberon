\chap{Pathnames}

A \emph{pathname} on a hierarchical \posix file system refers to a
directory or a file in a manner analogous to a street address can
refer to a building or an individual at a particular location.
Consider, for example, this fictitious business address:

\begin{xnote}
JOYCE L. BROWNING \\
2045 ROYAL ROAD \\
06570 ST PAUL \\
FRANCE
\end{xnote}

Although not as well organized hierarchically as a \posix pathname,
such a street address provides enough information to locate a specific
person in the country of France.  Just like a pathname facilitates
finding a specific file on a file system.

In this system, a pathname has the following characteristics:
\begin{enumerate}

  \item An \asciiz sequence of characters that specifies a file on a
    hierarchical files system.

    For example, \texttt{/var/log/syslog}.

  \item The '\texttt{/}' character is used to separate directory names
    in the hierarchy from each other in the pathname.

  \item Multiple '\texttt{/}' characters in a row are converted to a
    single instance.

  \item When the special character '\TILDE' appears at the beginning
    of a \emph{pathname}, it is replaced with the value of the shell
    environment variable \texttt{HOME}.

    For example, \texttt{\TILDE/scripts/remindo/remindo}

  \item If '\texttt{/}' appears at the beginning of the pathname, it
    is called an \emph{absolute pathname}, rooted at the top of the
    file system.

    For example, '\texttt{/etc/fstab}'.

  \item If '\texttt{/}' does not appear at the beginning of the
    pathname, it is called a \emph{relative pathname}.  A relative
    pathname is rooted at the \emph{current working
      directory}\footnote{The semantics of \emph{current working
        directory} are beyond the scope of this document}.

    For example, '\texttt{bin/emacs}' or '\texttt{fstab}'.

  \item All pathnames elements, up to the final element, are
    collectively known as the \emph{directory name}.

    The \emph{directory name} of '\texttt{bin/emacs}' is
    '\texttt{./bin}'.

  \item The character sequence '\texttt{.}' appearing as an element of
    the \emph{directory name} represents the \emph{current directory}.

    For example,

    \texttt{\TILDE/scripts/remindo/remindo}

    and

    \texttt{\TILDE/scripts/remindo/././././remindo}

    are equivalent.

  \item The character sequence '\texttt{..}' appearing as an element of
    the \emph{directory name} represents the \emph{parent directory}.

    For example,

    \texttt{\TILDE/scripts/remindo/remindo}

    and

    \texttt{\TILDE/scripts/remindo/../remindo/../remindo/../remindo}

    are equivalent.

  \item The final element in a \emph{pathname} is called the
    \emph{filename}.  The \emph{filename} includes the \emph{basename}
    and an optional \emph{extension}.

    The \emph{filename} of '\texttt{./bin/emacs}' is '\texttt{emacs}'.

    The \emph{filename} of '\texttt{/etc/fuse.conf}' is
    '\texttt{fuse.conf}'.

  \item The \emph{basename} is the portion of a \emph{filename} that
    is not the \emph{extension}.

    The \emph{basename} of '\texttt{converter.sh}' is '\texttt{converter}'.

    The \emph{basename} of '\texttt{converter.sh.save}' is '\texttt{converter.sh}'.

  \item A \emph{filename} may have an \emph{extension}.  An extension
    is, inclusively, the final '\texttt{.}' followed by \ascii
    characters up to the terminating \texttt{0X} character.

    The \emph{extension} of '\texttt{converter.sh}' is '\texttt{.sh}'.

    The \emph{extension} of '\texttt{converter.sh.save}' is '\texttt{.save}'.

  \item A special form of \emph{filename}, called a \emph{dotfile}, is
    when a filename begins with '\texttt{.}'.

    A \emph{dotfile} may have an extension.

    For example, '\texttt{.emacs}' or '\texttt{.emacs.save}'.

\end{enumerate}

\section{Constants}

There are no constants exported from this module.


\section{Types}

\subsection{\texttt{Element}}\label{pathnames:element}
\begin{alltt}
  Element* = POINTER TO ARRAY OF CHAR;
\end{alltt}

This type is used to refer to any individual part of a
\emph{pathname}, including a full reconstituted pathname.

\begin{invariant}
  (\forall i: 0 \leq i < \texttt{LEN(Element\deref)}:
  \texttt{Element[}i\texttt{]} \neq \texttt{'/'}) \logicaland
  \\
  (\texttt{Element[0]} \neq \texttt{0X}) \logicaland
  (\exists i: 1 \leq i < \texttt{LEN(Element\deref)}:
  \texttt{Element[}i\texttt{]} = \texttt{0X})
\end{invariant}

\begin{xnote}
It is possible for a pathname, or the element of a pathname, to be
syntactically correct, but semantically incorrect for the host
operatating system. There are many ways that a pathname can be correct
and incorrect at the same time, for example, a referring to a file in
a directory that does not exist will be correct and incorrect at the
same time.

This module only deals with syntax, while the host operating system
will deal with semantics when operations (open, close, delete, etc.)
are attempted on a pathname.
\end{xnote}


\subsection{\texttt{Pathname}}\label{pathnames:pathname}
\begin{alltt}
  Pathname* = POINTER TO PathnameDesc;
\end{alltt}

\begin{semantics}
  This type is a reference \texttt{PathnameDesc}.

It is used to manipulate the decomposed \emph{pathname} returned value
by \texttt{Create} (\xref{pathnames:create}).
\end{semantics}

\subsection{\texttt{PathnameDesc}}\label{pathnames:pathnamedesc}

\begin{alltt}
  PathnameDesc* = RECORD
    elements-  : Elements;
    basename-  : Element;
    extension- : Element;
    absolute-  : BOOLEAN;
  END;
\end{alltt}

\begin{semantics}
  This record is used to hold the decomposed pathame created by
  \texttt{Create} (\xref{pathnames:create}).

  \begin{description}
  \item{\texttt{elements}}

    This field holds the individual elements of the original pathname
    that were separated by \texttt{'/'}.  The field will be
    \texttt{NIL} if the original pathname was only a filename.

    \begin{invariant}
      (\texttt{elements} = \nil) \logicalor \\
      (\forall i: 0 \leq i < \texttt{LEN(elements\deref)}:
      (\texttt{elements[i]} \neq \nil) \logicaland (\validfn{elements[i]\deref}))
    \end{invariant}

    \item{\texttt{basename}}

      This field contains the \emph{basename} from the original pathname.

    \begin{invariant}
      (\texttt{basename} \neq \nil) \logicaland \\
      (\exists i: 0 \leq i < \texttt{LEN(basename\deref)}:
      \texttt{basename[i]} = \texttt{0X})
    \end{invariant}

    \item{\texttt{extension}}

      This field contains the \emph{extension} from the original
      pathname.  If there was no extension in the original pathname,
      this field will be NIL.

    \begin{invariant}
      (\texttt{extension} = \nil) \logicalor
      ((\texttt{extension[0]} = \texttt{'.'}) \logicaland \\
      (\exists i: 1 \leq i < \texttt{LEN(extension\deref)}:
      \texttt{extension[i]} = \texttt{0X}))
    \end{invariant}

    \item{\texttt{absolute}}

      If \texttt{TRUE}, the first character of the original pathname
      was \texttt{'/'}.  If \texttt{FALSE}, the first character of the
      original pathname was not \texttt{'/'}.
  \end{description}
\end{semantics}


\section{Procedures}
\subsection{\texttt{Basename}}\label{pathnames:basename}
\begin{alltt}
  PROCEDURE (self : Pathname) Basename*() : Element;
\end{alltt}

\begin{semantics}
This procedure returns the \emph{basename} contained in the
\texttt{self} variable.
\end{semantics}

\begin{precondition}
(\texttt{self} \neq \nil) \logicaland \validfn{self\deref}
\end{precondition}

\begin{postcondition}
(\texttt{result} \neq \nil) \logicaland \validfn{result\deref}
\end{postcondition}


\subsection{\texttt{Create}}\label{pathnames:create}
\begin{alltt}
  PROCEDURE Create*(path : ARRAY OF CHAR) : Pathname;
\end{alltt}

\begin{semantics}
This procedure decomposes the input \texttt{path} into an internal
representation which can be operated upon by other procedures in this
modlue.

It returns a pointer to this internal representation, if the input is
properly formed.  If the input is not properly formed, \nil is
returned.
\end{semantics}

\begin{precondition}
  (\texttt{path[0]} \neq \texttt{0X}) \logicaland
  (\exists i: 1 \leq i < \texttt{LEN(path)}:
  \texttt{path[}i\texttt{]} = \texttt{0X})
\end{precondition}

\begin{postcondition}
(\texttt{result} = \nil) \logicalor \validfn{result\deref}
\end{postcondition}

\subsection{\texttt{Dirname}}\label{pathnames:dirname}
\begin{alltt}
  PROCEDURE (self : Pathname) Dirname*() : Element;
\end{alltt}

\begin{semantics}
  This procedure returns the \emph{directory name} contained in the
  \texttt{self} variable.
\end{semantics}

\begin{precondition}
(\texttt{self} \neq \nil) \logicaland \validfn{self\deref}
\end{precondition}

\begin{postcondition}
(\texttt{result} \neq \nil) \logicaland \validfn{result\deref}
\end{postcondition}


\subsection{\texttt{Delete}}\label{pathnames:delete}
\begin{alltt}
  PROCEDURE (self : Pathname) Delete*(beg, end : INTEGER;
                                      VAR success : BOOLEAN);
\end{alltt}

\begin{semantics}
  This procedure deletes the open-ended range, $[beg, end)$ of
    elements from the \texttt{self}, when the range is valid.

    The variable \texttt{success} is set to \texttt{TRUE} when the
    input range is valid, otherwise it is set to \texttt{FALSE}.
\end{semantics}

\begin{precondition}
  (\texttt{self} \neq \nil) \logicaland \validfn{self\deref}
  \logicaland \\
  (\texttt{beg} \geq 0) \logicaland (\texttt{beg} < \texttt{end})
  \logicaland (\texttt{end} \leq \texttt{LEN(self.elements\deref)})
\end{precondition}

\begin{postcondition}
  (\texttt{success} \implies \textrm{Given range of
    elements deleted from the pathname.}) \logicalor \\
  (\texttt{\logicalnot success} \implies \textrm{Input range invalid;
    pathname unchanged.})
\end{postcondition}

\subsection{\texttt{Extension}}\label{pathnames:extension}
\begin{alltt}
  PROCEDURE (self : Pathname) Extension*() : Element;
\end{alltt}

\begin{semantics}
  This procedure returns the optional \emph{extension} contained in the
  \texttt{self} variable.

  If there is no \emph{extension}, then \nil is returned.
\end{semantics}

\begin{precondition}
(\texttt{self} \neq \nil) \logicaland \validfn{self\deref}
\end{precondition}

\begin{postcondition}
(\texttt{result} = \nil) \logicalor \validfn{result\deref}
\end{postcondition}


\subsection{\texttt{Filename}}\label{pathnames:filename}
\begin{alltt}
  PROCEDURE (self : Pathname) Filename*() : Element;
\end{alltt}

\begin{semantics}
  This procedure returns the \emph{filename} contained in the
  \texttt{self} variable.
\end{semantics}

\begin{precondition}
(\texttt{self} \neq \nil) \logicaland \validfn{self\deref}
\end{precondition}

\begin{postcondition}
(\texttt{result} \neq \nil) \logicaland \validfn{result\deref}
\end{postcondition}


\subsection{\texttt{Insert}}\label{pathnames:insert}
\begin{alltt}
  PROCEDURE (self : Pathname) Insert*(beg        : INTEGER;
                                      path        : Pathname;
                                      VAR success : BOOLEAN);
\end{alltt}

\begin{semantics}
This procedure inserts \texttt{path} at position \texttt{beg} in the
pathname contained in \texttt{self}.

An \emph{absolute} \texttt{path} can only be inserted at position zero
(0).

The internally decomposed equivalent of \texttt{path.Pathname()} result
is inserted into the \emph{pathname} held in \texttt{self} at position
\texttt{beg}.

If the preconditions are met, \texttt{success} is set to
\texttt{TRUE}, otherwise \texttt{FALSE}.

\end{semantics}

\begin{precondition}
  (\texttt{self} \neq \nil) \logicaland \validfn{self\deref} \logicaland \\
  ((\texttt{beg} = 0) \logicalor \logicalnot \texttt{path.absolute}) \logicaland \\
    (\texttt{beg} \geq 0) \logicaland (\texttt{beg} \leq \texttt{LEN(self.elements\deref)})
\end{precondition}

\begin{postcondition}
  (\texttt{success} \implies \validfn{self\deref} \logicaland
  \texttt{path}\textrm{\xspace is inserted}) \logicalor \\
  (\logicalnot \texttt{success} \implies \texttt{self\deref}\textrm{\xspace is unchanged})
\end{postcondition}


\subsection{\texttt{Pathname}}\label{pathnames:pathname}
\begin{alltt}
  PROCEDURE (self : Pathname) Pathname*() : Element;
\end{alltt}

\begin{semantics}
  This procedure returns the full \emph{pathname} contained in the
  \texttt{self} variable.
\end{semantics}

\begin{precondition}
(\texttt{self} \neq \nil) \logicaland \validfn{self\deref}
\end{precondition}

\begin{postcondition}
(\texttt{result} \neq \nil) \logicaland \validfn{result\deref}
\end{postcondition}


\subsection{\texttt{SetAbsolute}}\label{pathnames:setAbsolute}
\begin{alltt}
  PROCEDURE (self : Pathname)
            SetAbsolute*(v : BOOLEAN);
\end{alltt}

\begin{semantics}
  This procedure sets \texttt{self.absolute} to \texttt{v}.  When a
  pathname is \emph{absolute}, it is rooted at the top of the file
  system.  When it is not \emph{absolute}, it is rooted at the
  \emph{current working directory}.
\end{semantics}

\begin{precondition}
(\texttt{self} \neq \nil) \logicaland \validfn{self\deref}
\end{precondition}

\begin{postcondition}
  (\texttt{self.absolute}  = \texttt{v})
\end{postcondition}


\subsection{\texttt{SetBasename}}\label{pathnames:setBasename}
\begin{alltt}
  PROCEDURE (self : Pathname)
            SetBasename*(basename    : ARRAY OF CHAR;
                         VAR success : BOOLEAN);
\end{alltt}

\begin{semantics}
  This procedure changes the \emph{basename} of the \texttt{self}
  variable.  Because every pathname must have a basename, the first
  character of \texttt{basename} cannot be \texttt{0X}.

  The \texttt{success} argument indicates the sucess or failure of the
  operation.  If \texttt{success} is \texttt{FALSE}, the \emph{basename}
  will not be changed.
\end{semantics}

\begin{precondition}
(\texttt{self} \neq \nil) \logicaland \validfn{self\deref} \logicaland
  (\texttt{basename[0]} \neq 0X) \logicaland \\
  (\exists i: 1 \leq i < \texttt{LEN(basename)}: \texttt{basename[}i\texttt{]} = \texttt{0X})
\end{precondition}

\begin{postcondition}
  (\texttt{success} \implies \textrm{\emph{Basename} changed.}) \logicalor
  (\logicalnot \texttt{success} \implies \textrm{\emph{Basename} not changed.})
\end{postcondition}


\subsection{\texttt{SetExtension}}\label{pathnames:setextension}
\begin{alltt}
  PROCEDURE (self : Pathname)
            SetExtension*(ext         : ARRAY OF CHAR;
                          VAR success : BOOLEAN);
\end{alltt}

\begin{semantics}
  This procedure changes the \emph{extension} of the \texttt{self} variable.

  To remove the extension, set \texttt{ext[0]} to \texttt{0X}, or pass
  the empty string (\texttt{""}).

  The \texttt{success} argument indicates the sucess or failure of the
  operation.  If \texttt{success} is \texttt{FALSE}, the \emph{extension}
  will be unchanged.
\end{semantics}

\begin{precondition}
(\texttt{self} \neq \nil) \logicaland \validfn{self\deref}  \logicaland
  (\texttt{ext[0]} = \texttt{'.'}) \logicaland \\
  (\exists i: 1 \leq i < \texttt{LEN(ext)}: \texttt{ext[}i\texttt{]} = \texttt{0X})
\end{precondition}

\begin{postcondition}
  (\texttt{success} \implies \textrm{\emph{Extension} changed.}) \logicalor
  (\logicalnot \texttt{success} \implies \textrm{\emph{Extension} not changed.})
\end{postcondition}
