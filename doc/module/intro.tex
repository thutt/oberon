\chap{Introduction}

This manual provides a programmatic reference to the modules added to
the ETHZ Oberon distribution.

\section{Conventions}

This manual will document the externally visible state of modules so
that users can understand the declared API; the external API will
generally not changed.  It will not document the internal state,
because that can change over time

\subsection{Invariant}\label{intro:invariant}

The documentation will use \emph{invariant properties} of the exported
types to simplify the documentation.  An invariant is described by
David Gries:

\begin{quote}
... The adjective \emph{invariant} means \emph{constant} or
  \emph{unchanging}.  In mathematics the term means \emph{unaffected
    by the group of mathematical operations under
    consideration}...\footnote{The Science of Programming, David
    Gries.  Page 141, Copyright 1981, Springer Verlag, ISBN
    0-387-96480-0, ISBN 3-540-96840-0}
\end{quote}

In the case of this documentation, it means that the invariant
expression must always be true.  If it is not true, the program is in
undefined state, and its continued execution cannot be expected to be
correct.

As an example, a pointer \texttt{P} that cannot be \nil after
module initialization would be written as shown in
\figref{intro:invariant}.

\begin{figure}[h]
  \begin{invariant}
    \texttt{P} \neq \nil
  \end{invariant}
  \caption{\texttt{P} may never be \nil}\label{fig:intro:invariant}
\end{figure}

\subsection{Precondition}

A \emph{precondition} is a boolean expression that must be true before
invoking a procedure.


\subsection{Postcondition}

A \emph{postcondition} is a boolean expression that will be true
after invoking a procedure.

\subsection{Or ($\logicalor$)}\label{intro:or}

This symbol is \emph{disjunctive or}.  For simplicity, it this
documentation, it has the same precdence as \emph{conjunctive and}
(\xref{intro:and}), so if precedence is desired, uses parenthesis to
disambiguate the order.

\subsection{And ($\logicaland$)}\label{intro:and}

This symbol is \emph{conjunctive and}.  For simplicity, in this
documentation, it has the same precdence as \emph{disjunctive or}
(\xref{intro:or}), so if precedence is desired, uses parenthesis to
disambiguate the order.

\subsection{Exists ($\exists$)}

The \emph{exists} construction allows iteration over a type or
variable to verify an aspect regarding the whole item.  It has two
parts:
\begin{enumerate*}[label=\arabic*)]
\item A variable with its bounding range,
\item A boolean expression which must hold for \emph{at least one}
  element in the range.
\end{enumerate*}  Given the type declaration below, one could assert
each \texttt{String} must be \texttt{0X} terminated as shown in
\figref{intro:exists}.

\begin{alltt}
  TYPE
    String = POINTER TO ARRAY OF CHAR;
\end{alltt}

\begin{figure}[h]
  \begin{invariant}
    (\exists i: 0 \leq i < \texttt{LEN(String)}: \texttt{String[}i\texttt{]} = \texttt{0X})
  \end{invariant}
  \caption{Each \texttt{String} must be \texttt{0X} terminated.}\label{fig:intro:exists}
\end{figure}

\subsection{For All ($\forall$)}\label{intro:forall}

The \emph{for all} construction allows iteration over a type or
variable to verify each element.  It has two parts:
\begin{enumerate*}[label=\arabic*)]
\item A variable with its bounding range,
\item A boolean expression which must hold for \emph{each} element in
  the range.
\end{enumerate*}  Given the type declarations below, one could assert
that every value in a variable of type \texttt{Table} is \nil
or references a valid string as shown in \figref{intro:forall}.

\begin{alltt}
  TYPE
    String = POINTER TO ARRAY OF CHAR;
    Table = POINTER TO ARRAY OF String;
\end{alltt}

\begin{figure}[h]
  \begin{invariant}
    \begin{small}
      \begin{array}{lll}
        (\forall i: 0 \leq i < \texttt{LEN(Table)}:
        & \texttt{Table[}i\texttt{]} = \nil \logicalor \\
        & (\exists j: 0 \leq j < \texttt{LEN(String)}: \texttt{Table[}i\texttt{][j]} = \texttt{0X})
      \end{array}
    \end{small}
  \end{invariant}
  \caption{Each table entry is \nil or references a valid string.}\label{fig:intro:forall}
\end{figure}


\subsection{ASCIIZ}

\asciiz is a shorthand for the following invariant, which states that
the character array \texttt{V} must be \texttt{0X}-terminated.

\begin{invariant}
(\exists i: 0 \leq i < \texttt{LEN(V)}: \texttt{V[}i\texttt{]} = \texttt{0X})
\end{invariant}

\subsection{Validity}

The meta function \texttt{valid} is used as a shorthand to assert that
all invariants (\xref{intro:invariant}) hold on the argument.

Using this meta function, the invariant in \xref{intro:forall} could
be simplified as shown in \figref{intro:valid}

\begin{figure}[h]
  \begin{invariant}
    (\forall i: 0 \leq i < \texttt{LEN(Table)}:
    \texttt{Table[}i\texttt{]} = \nil \logicalor valid(\texttt{Table[}i\texttt{]\deref}))
  \end{invariant}
  \caption{Each table entry is \nil or references a valid string.}\label{fig:intro:valid}
\end{figure}


\subsection{Result}

The meta variable \result{} is used as a shorthand to refer to the
result of a function invocation.
