\section{Heap Block Descriptors}\label{memlay:block-descriptors}

Each dynamically allocated memory block returned by the memory
allocator will have a four byte \emph{tag} at offset -4 describing the
contents of the block.  This block tag is used by the memory
allocation system and is part of the administrative overhead.

The tag, which is a pointer to a type descriptor in many contexts,
reserves the upper two bits for the garbage collector, effectively
limiting the Oberon heap size to $2^{30}$ bytes
(1~gigabyte)\footnote{In practical terms, the LMS implementation uses
  identity mapped memory on Linux, and the kernel fails to map if more
  than 128 megabytes is requested.}.

\subsection{Block Tag Validity}

Normally a block's tag has invariant properties, but during garbage
collection, the block tag of dynamically allocated \texttt{RECORD}
types is overwritten with an index into the pointer table in the type
descriptor of the block.  The overwriting of an actual block tag --
which is nominally a pointer to a type descriptor -- violates the
invariants, and so a method to determine if a block tag is really a
tag, or an index into the pointer table is employed.

The highest two bits of the block tag are used to determine the
semantics of the block tag, as shown in
\figref{memlay:block-tag-upper-bits}.

\begin{figure}[h]
  \begin{eqnarray}
    \neg (31 \in \textrm{\texttt{tag}}) \wedge \neg (30 \in \textrm{\texttt{tag}})  & \implies & \textrm{offset into pointer table} \nonumber \\
    \neg (31 \in \textrm{\texttt{tag}}) \wedge (30 \in \textrm{\texttt{tag}}) & \implies & \textrm{normal tag} \nonumber \\
    (31 \in \textrm{\texttt{tag}}) \wedge \neg (30 \in \textrm{\texttt{tag}}) & \implies & \textrm{invalid} \nonumber \\
    (31 \in \textrm{\texttt{tag}}) \wedge (30 \in \textrm{\texttt{tag}})  & \implies & \textrm{final offset into pointer table} \nonumber
  \end{eqnarray}
  \caption{Determining semantics of a block tag}
  \label{memlay:block-tag-upper-bits}
\end{figure}

\subsection{Heap Block Kind}

The low nibble of the block tag contains part of the information
necessary for the \gc to perform its work; these low bits must be
masked off before the tag is used\footnote{LMS implementation ensures
  that all dyanamically allocated blocks are aligned to sixteen bytes,
  therefore the low-order four bits need not be masked off before
  using a normal pointer, but they must be masked when dealing with
  type descriptors contained in block tags.}.

\Figref{memlay:block-tag-values} describes the meanings of the low
four bits of the tag.  Do not change the value of these flags without
updating both the \gc in Kernel.Mod and the bootstrap loader.

\begin{figure}
  \begin{tabularx}{\linewidth}{|c|l|c|X|}
    \hline \S &  Mnemonic & Pattern  &    Description\\

    \hline \xref{memlay:descriptors-kind-blkdesc} &
    \texttt{BlkDesc} & 0 0 0 0 & The tag is the address of a type
    descriptor. \\

    \hline \xref{memlay:descriptors-kind-blkmark} &
    \texttt{BlkMark} & 0 0 0 1 & A block marked during garbage
    collection. \\

    \hline \xref{memlay:descriptors-kind-blksyst} &
    \texttt{BlkSyst} & 0 0 1 0 & A system block. System blocks are not garbage
    collected. \\

    \hline \xref{memlay:descriptors-kind-blkfree} &
    \texttt{BlkFree} & 0 1 0 0 & An unallocated memory block. \\

    \hline \xref{memlay:descriptors-kind-blkaray}  &
    \texttt{BlkAray}  & 1 0 0 0 & A dynamically allocated array block. \\
    \hline
  \end{tabularx}
  \caption{Block Tag Bit Values} \label{memlay:block-tag-values}
\end{figure}

\subsubsection{\texttt{BlkDesc}}\label{memlay:descriptors-kind-blkdesc}

A block tag that does not have any of \{ \texttt{BlkSyst},
\texttt{BlkFree}, \texttt{BlkAray} \} set references the type
descriptor for the associated heap block (which is a \texttt{RECORD}).

\begin{Ventry}{Address of next block}
\item[Size of block]
  The size of a \texttt{BlkDesc} block is determined by inspecting the type
  descriptor.

\item[Address of next block] The address of the next block can only
  be computed after the size of the block has been determined.
\end{Ventry}


\Figref{memlay:blkdesc-extract} shows how to turn the tag into the
address of the type descriptor. See \xref{memlay:type-descriptors} for
information on the format of these descriptors.


\begin{figure}[h]
$$\textrm{block tag} - \lbrace \texttt{BlkMark} \rbrace \equiv \textrm{descriptor address}$$
  \caption{Extracting a type descriptor address from the block tag}
  \label{memlay:blkdesc-extract}
\end{figure}

\subsubsection{\texttt{BlkMark}}\label{memlay:descriptors-kind-blkmark}

The \texttt{BlkMark} bit is used only by the \gc and must be masked
off by all other processes.  During the mark phase of garbage
collection, all blocks which are reachable by a pointer will have this
bit set.  During the \emph{sweep} phase, all blocks which do not have
the bit set will be reclaimed by the heap management system.

\subsubsection{\texttt{BlkSyst}}\label{memlay:descriptors-kind-blksyst}

A tag marked with \texttt{BlkSyst} is a \emph{system block} and is
always ignored by the \gc.  The invariants for tags with
\texttt{BlkSyst} are shown in \figref{memlay:blksyst-invariants}.

\begin{Ventry}{Address of next block}
\item[Size of block]      $\textrm{tag} - \lbrace\texttt{BlkMark}, \texttt{BlkSyst}\rbrace$
\item[Address of next block]     $\adr{tag} + \size{LONGINT} + \textrm{block size}$
\end{Ventry}

\begin{figure}[h]
    $$(\texttt{BlkSyst} \implies \neg \texttt{BlkFree}) \wedge
    (\texttt{BlkSyst} \implies \neg \texttt{BlkAray})$$
  \caption{BlkSyst invariants}
  \label{memlay:blksyst-invariants}
\end{figure}

\subsubsection{\texttt{BlkFree}}\label{memlay:descriptors-kind-blkfree}

A tag marked with \texttt{BlkFree} indicates that the accociated
memory block is not currently allocated.
\Figref{memlay:blkfree-invariants} shows the invariants when this bit
in the tag.

Additionally, since the memory allocation system manages unallocated
memory blocks on one of several free block lists, a pointer to the
next unallocated memory block (on the same list) is also stored in the
first four bytes of the block.

\begin{Ventry}{Address of next block on free list}
\item[Size of block]      $\textrm{tag} - \lbrace\texttt{BlkMark}, \texttt{BlkFree}\rbrace$
\item[Address of next block]  $\adr{tag} + \size{LONGINT} + \textrm{block size}$
\item[Address of next block on free list] $\mem{\adr{tag} + \size{LONGINT}}$
\end{Ventry}


\begin{figure}[h]
    $$(\texttt{BlkFree} \implies \neg \texttt{BlkSyst}) \wedge
    (\texttt{BlkFree} \implies \neg \texttt{BlkAray})$$
  \caption{BlkFree invariants}
  \label{memlay:blkfree-invariants}
\end{figure}

\subsubsection{\texttt{BlkAray}}\label{memlay:descriptors-kind-blkaray}

A tag can be marked with \texttt{BlkAray} in two circumstances.  The
first is known as the \emph{external} tag, and it is set on the memory
containing a dynamically allocated array.  The second, known as the
\emph{internal} tag, is located four bytes before the sixteen-byte
allocated array data.  \Figref{memlay:blkaray-invariants} shows the
invariants when this bit in a tag.

\begin{enumerate}
\item The \emph{external} tag is part of the preamble for a
  dynamically allocated array.  When the flags are masked off, the
  block tag is the address of the actual dynamically allocated array
  data.

  See \xref{memlay:heap-array} for information on the internal layout
  of these blocks.

\begin{Ventry}{Address of next block}
\item[Size of block] $\mem{\adr{tag} + 4}$
\item[Address of next block] $\adr{tag} + \textrm{size of block}$
\item[Address of array]    $\textrm{tag} - \lbrace \texttt{BlkMark},
  \texttt{BlkAray} \rbrace$
\end{Ventry}

\item The \emph{internal} tag is present four bytes before the actual
  data of a dynamically allocated array.  This usage is mainly a
  simplification for the mark phase.  It enables easy detection of
  array blocks so that the external tag can be set.

  Note, however, this usage also violates the rule in
  \xref{memlay:descriptors-kind-blkdesc} that indicates unmarked type
  descriptor pointers must not have any additional bits set so that no
  masking has to be done prior to the usage of the descriptor.  This
  deviation is deemed an acceptable exception to the rule since the
  descriptor for dynamically allocated arrays is only used by the heap
  allocator and the garbage collector; all uses of this descriptor are
  handled in the Kernel module and the extra overhead of masking out
  the flags will have little runtime overhead.

\begin{Ventry}{Address of next block}
\item[Size of block]      not applicable
\item[Address of next block]     not applicable
\item[Address of array]    not applicable
\end{Ventry}
\end{enumerate}

Since no pointers in the program will actually point to the preamble
of a dynamically allocated array, and array type descriptors are only
used by the heap allocator and the garbage collector, there is no
ambiguity between the two distinct uses of the same bit.

\begin{figure}[h]
    $$(\texttt{BlkAray} \implies \neg \texttt{BlkSyst}) \wedge
    (\texttt{BlkAray} \implies \neg \texttt{BlkFree})$$ \\
  \caption{BlkAray invariants}
  \label{memlay:blkaray-invariants}
\end{figure}
