\chapter{Architecture}
This chapter describes the various attributes of the \ac{CPU}, such as
general purpose registers, control registers and the execution model.

\section{Program Counter}

Internally, the processor maintains a special register called the
\ac{PC}.  The \ac{PC} holds the address in memory at which the
processor is currently executing.  The \ac{PC} is not directly
accessible to the user, but it can be indirectly changed with control
flow instructions and traps (\xref{sect:CR1}, \xref{sect:CR5}).

The \ac{PC} always references the instruction which is currently
executing.

\section{General Purpose Registers}

The processor has 32 general purpose registers, named \asmreg{0}
through \asmreg{31}.  The registers are four bytes in size. Each of
the registers can be used in any instruction which has register
operands, though several registers have special properties.  The
registers with special properties are described in the following
sections.

\subsection{\asmreg{0}}

\asmreg{0} always has the value zero; writes to this register are
discarded.

\subsection{\asmreg{29}}

This register is used by software as the \ac{SFP}.  The \ac{SFP} is
used to access local variables and procedure arguments.  Procedure
arguments are accessed at positive offsets, while procedure-local
variables are accessed at negative offsets.

See \xrefinst{enter}, \xrefinst{leave}.

\subsection{\asmreg{30}}

This is the hardware \ac{SP}.

See \xrefinst{push}, \xrefinst{pushf}, \xrefinst{pushd},
\xrefinst{pop}, \xrefinst{popf} and \xrefinst{popd}.

\subsection{\asmreg{31}}

\asmreg{31} is the hardware \ac{LR}.  The \ac{LR} is assigned the
\emph{return address} when executing certain types of jumps.  This
register is used to implement a \emph{call} / \emph{return} semantic.

See \xrefinst{jal}, \xrefinst{jral}.


\section{Floating Point Registers}
The processor has 32 floating point registers, named \fasmreg{0}
through \fasmreg{31}.  Each of the regsiters can be used in any
floating point instruction which has register operands.

Each floating point register is an IEEE 754 double precision real value.

\section{Control Registers}\label{sect:control-registers}

The \ac{CPU} contains several \acp{CR} that allows one to alter how
the \ac{CPU} functions, or provides information about events that have
occurred internally.  \acp{CR} are four bytes in size.

The control registers are read and written with special instructions,
such as \xrefinst{lcr} and \xrefinst{scr}.


\subsection{CR0: Hardware Fault Return Address}

This \ac{CR} contains the address at which the most recent fault
occurred.

\begin{figure}[h]
  \centering
    \begin{bytefield}{32}
      \bitheader[endianness=big]{0,31}          \\
      \bitbox{32}{address of fault}
    \end{bytefield}
  \caption{\texttt{CR0} encoding}
\end{figure}

The value in this register is written by the hardware when a fault is
raised.  See \xref{sect:CR1} for details on what causes a hardware
fault to be raised.


\subsection{CR1: Hardware Fault Handler Address}\label{sect:CR1}

This register contains the address to which the \skl processor will
transfer execution when an fault is raised.  The handler is
responsible for saving and restoring any registers it uses.

\begin{figure}[h]
  \centering
    \begin{bytefield}{32}
      \bitheader[endianness=big]{0,31}          \\
      \bitbox{32}{hardware fault handler address}
    \end{bytefield}
  \caption{\texttt{CR1} encoding}
\end{figure}

The code address that is put into this register must be four byte
aligned.  \ac{CPU} behavior is undefined if the address does not refer
to code or is not aligned to four bytes.

\Figref{hwdexc} shows the different hardware faults that can
occur.

\begin{figure}[h]
  \centering
    \begin{tabularx}{\linewidth}{|l|X|}
      \hline Cause & Description \\
      \hline Illegal Write & An attempt was made to read from memory
      that is outside the memory accessible to the \ac{CPU}. \\

      \hline Illegal Read & An attempt was made to read from memory
      that is outside the memory accessible to the \ac{CPU}. \\

      \hline Misaligned Instruction & An attempt was made to execute
      an instruction at an address that is not a multiple of four. \\

      \hline Invalid Opcode & An attempt was made to execute an
      instruction that is not supported by the \ac{CPU}. \\

      \hline
    \end{tabularx}
  \caption{Hardware Faults}\label{fig:hwdexc}
\end{figure}

The \ac{CPU} will transfer control to the address in this register
when a hardware fault is raised, and therefore it is important to
initialize this register early in the boot process.


\subsection{CR2: Fault Status}\label{CR:CR2}

This \ac{CR} contains information describing the most recent fault
that was raised. \Figref{cr2-encoding} shows the bitwise encoding of
this register, and \figref{cr2-semantics} shows the semantics of the
bit fields.

\begin{figure}[h]
  \centering
    \begin{bytefield}{32}
      \bitheader[endianness=big]{0,1,2,4,5,6,7,31}        \\
      \bitbox{25}{reserved}
      \bitbox{2}{A}
      \bitbox{3}{T}
      \bitbox{1}{I}
      \bitbox{1}{C}
    \end{bytefield}
  \caption{\texttt{CR2} encoding}\label{fig:cr2-encoding}
\end{figure}

The contents of this register can be used to determine how a hardware
fault should be handled.

% Commented out because 'break' is not documented and not implemented.
%For example, \xrefinst{break} can be
%used to implement a breakpoint in a debugger.

\begin{figure}[h]
\begin{tabularx}{\linewidth}{|l|l|X|}
\hline Field & Meaning & Semantic \\
\hline C & Cause &
\begin{tabular}{ll}
 0 & Fault caused by an external interrupt. \\
 1 & Fault caused by a trap internal to the \ac{CPU}. \\
\end{tabular} \\

\hline I & Interrupt Flag &
\begin{tabular}{ll}
   & Reserved \\
%0 & Interrupts were disabled when fault occurred. \\
%1 & Interrupts enabled when fault occurred. \\
\end{tabular} \\

\hline T & Trap Type\footnote{This field is undefined when an external trap occurs.} &

\begin{tabular}{ll}
  000 & undefined opcode \\
  % break is not documented and not implemented.
  %  001 & \texttt{break} instruction \\
  001 & reserved \\
  010 & bad instruction alignment \\
  011 & out-of-bounds read \\
  100 & out-of-bounds write \\
  101 & divide by zero \\
  110 & \emph{reserved} \\
  111 & \emph{reserved} \\
\end{tabular} \\

\hline A & Active &
\begin{tabular}{lp{0.6\textwidth}}
 00 & No faults active. \\

 01 & A fault is currently being handled. \\

 10 & A second fault was raised while processing the
 first fault.  The virtual hardware terminates on this condition.  \\

 11 & \emph{reserved} \\
\end{tabular} \\

\hline
\end{tabularx}

\caption{\texttt{CR2} semantics}\label{fig:cr2-semantics}
\end{figure}

\subsection{CR3: Reserved}

\subsection{CR4: Reserved}

\subsection{CR5: \emph{Software Trap} Handler Address}\label{sect:CR5}
This \ac{CR} holds the address to which control will be transferred
when a software trap occurs.


\begin{figure}[h]
  \centering
    \begin{bytefield}{32}
      \bitheader[endianness=big]{0,31}          \\
      \bitbox{32}{software trap handler address}
    \end{bytefield}
  \caption{\texttt{CR5} encoding}\label{fig:cr5-encoding}
\end{figure}

When execution begins at the address contained in \texttt{CR5},
\asmreg{1} contains the trap number that occurred, and \asmreg{31}
contains the address of the instruction that trapped.  See
\figref{cr5-r1-value} for possible values that can be in \asmreg{1}.

\begin{figure}
  \begin{tabularx}{\linewidth}{|l|X|}
    \hline Value & Semantic \\
    \hline 8 & Oberon \texttt{ASSERT} failed. \\
    \hline 12 & Out-of-range value used with \texttt{SHORTINT},
    \texttt{INTEGER}, or \texttt{SET}. See \xrefinst{traprange}. \\
    \hline 13 & Out-of-bounds array index used.  See \xrefinst{traparray}.\\
    \hline 14 & Compiler-generated Oberon \emph{type guard} failed. \\
    \hline 15 & Explicit Oberon \emph{type guard} failed. \\
    \hline 16 & Unhandled case in Oberon \texttt{CASE} statement. \\
    \hline 17 & Oberon \texttt{HALT} function used. \\
    \hline 18 & Unhandled case in Oberon \texttt{WITH} statement. \\
    \hline 19 & \texttt{NIL} pointer used in Oberon \emph{type guard}.   See \xrefinst{trapnil}. \\
    \hline 20 & \texttt{NIL} pointer dereference.  See \xrefinst{trapnil}.  \\

    \hline 128 & \texttt{Files.ReadBytes} or \texttt{Files.WriteBytes}
    buffer was smaller than specified number of bytes. \\ \hline
  \end{tabularx}
  \caption{CR5: Possible \asmreg{1} values}\label{fig:cr5-r1-value}
\end{figure}

The handler is responsible for saving and restoring any registers it uses.
