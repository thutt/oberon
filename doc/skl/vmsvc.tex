\chap{VMSVC Reference}\label{chap:vmsvc}

The \texttt{vmsvc} instruction provides the ability for the virtual
\ac{CPU}, the software running on it and the interpreter to cooperate
in a manner that allows for host OS services (such as console or file
IO) to be performed.

As described in \xrefinst{vmsvc}, the instruction takes the address of
a \emph{descriptor} that is known to the interpreter.  The interpreter
examines the \emph{service id} field to determine what service to
provide.  Furthermore, employing Oberon's \emph{type extension} allows
the \texttt{Kernel} module and the interpreter to be closely coupled,
and allows different VM services to be implemented; the interpreter
will have one part of the implementation (interfacing with the host
OS), and an Oberon module will contain another part (executing the
\texttt{vmsvc} instruction).

This following sections provide an overview of the services provided.

\section{Bootstrap}

This service is used during bootstrapping of the Oberon system.  It
allows the bootstrap loader execute a single module's initialization
code and then return back to the loader.

This service uses the base \texttt{Kernel.VMSVCServiceDesc} type.

\section{Trace}

Certain build types of this software allow all instructions to be
traced when the Oberon system is started.  Because this comes at a
significant performance cost, it's also possible to control tracing
via \texttt{Kernel.VMSVCTracing} using this service call.

\begin{figure}[H]
  \centering
  \begin{bytefield}{32}
    \bitheader[endianness=big]{0-31} \\
    \begin{leftwordgroup}{0}
      \bitbox{32}{service id}
    \end{leftwordgroup} \\
    \begin{leftwordgroup}{4}
      \bitbox{32}{enable}
    \end{leftwordgroup} \\
  \end{bytefield}
  \caption{\texttt{Kernel.VMServiceTraceDesc}}\label{fig:vmsvc-tracedesc}
\end{figure}

\section{EarlySysTrap}

If a software trap occurs before the Oberon system is fully booted,
the virtual \ac{CPU} will still jump to the address in \creg{5},
regardless its initialization state.  The bootstrapping system must
not raise system traps, and therefore does not provide a formal
handler for them.  If they occur before the system is \emph{ready},
this service will terminate the bootstrapping process.
\figref{vmsvc-systrap}, shows the descriptor layout for this early
system trap system.

\begin{figure}[H]
  \centering
  \begin{bytefield}{32}
    \bitheader[endianness=big]{0-31} \\
    \begin{leftwordgroup}{0}
      \bitbox{32}{service id}
    \end{leftwordgroup} \\
    \begin{leftwordgroup}{4}
      \bitbox{32}{ip}
    \end{leftwordgroup} \\
    \begin{leftwordgroup}{8}
      \bitbox{32}{\ac{SFP}}
    \end{leftwordgroup} \\
    \begin{leftwordgroup}{12}
      \bitbox{32}{trap id}
    \end{leftwordgroup} \\
  \end{bytefield}
  \caption{\texttt{Kernel.VMServiceSysTrapDesc}}\label{fig:vmsvc-systrap}
\end{figure}


\section{DebugLog}
To facilitate console output from Oberon during bootstrapping, this
debugging service was created.  Once the \texttt{Console} module is
loaded and initialized, it can be used to write to the console.  The
descriptor for this service is shown in \figref{vmsvc-debug}.

\begin{figure}[H]
  \centering
  \begin{bytefield}{32}
    \bitheader[endianness=big]{0-31} \\
    \begin{leftwordgroup}{0}
      \bitbox{32}{service id}
    \end{leftwordgroup} \\
    \begin{leftwordgroup}{4}
      \bitbox{32}{operation}
    \end{leftwordgroup} \\
    \begin{leftwordgroup}{4}
      \bitbox{32}{data}
    \end{leftwordgroup} \\
    \begin{leftwordgroup}{4}
      \bitbox{32}{adr}
    \end{leftwordgroup} \\
  \end{bytefield}
  \caption{\texttt{Kernel.VMServiceTraceDesc}}\label{fig:vmsvc-debug}
\end{figure}


\section{Terminate}

This service terminates the Oberon system; it allows an exit code to
be returned to the parent process.

\begin{figure}[H]
  \centering
  \begin{bytefield}{32}
    \bitheader[endianness=big]{0-31} \\
    \begin{leftwordgroup}{0}
      \bitbox{32}{service id}
    \end{leftwordgroup} \\
    \begin{leftwordgroup}{4}
      \bitbox{32}{rc}
    \end{leftwordgroup} \\
  \end{bytefield}
  \caption{\texttt{Kernel.VMServiceTraceDesc}}\label{fig:vmsvc-terminate}
\end{figure}


\section{Time}

Timekeeping is a necessary component of Oberon.  To request time
information from the host OS, the descriptor shown in
\figref{vmsvc-time} is used.


\begin{figure}[H]
  \centering
  \begin{bytefield}{32}
    \bitheader[endianness=big]{0-31} \\
    \begin{leftwordgroup}{0}
      \bitbox{32}{service id}
    \end{leftwordgroup} \\
    \begin{leftwordgroup}{4}
      \bitbox{32}{op}
    \end{leftwordgroup} \\
    \begin{leftwordgroup}{4}
      \bitbox{32}{time}
    \end{leftwordgroup} \\
  \end{bytefield}
  \caption{\texttt{HostOS.SVCTime}}\label{fig:vmsvc-time}
\end{figure}


\section{Directory}

The \texttt{FileDir} module implements several directory operations:
\emph{open}, \emph{close} and \emph{read}.  To open and close, the
descriptor is \figref{vmsvc-dir}, and to read it's
\figref{vmsvc-readdir}.

\begin{figure}[H]
  \centering
  \begin{bytefield}{32}
    \bitheader[endianness=big]{0-31} \\
    \begin{leftwordgroup}{0}
      \bitbox{32}{service id}
    \end{leftwordgroup} \\
    \begin{leftwordgroup}{4}
      \bitbox{32}{op}
    \end{leftwordgroup} \\
    \begin{leftwordgroup}{8}
      \bitbox{32}{path}
    \end{leftwordgroup} \\
    \begin{leftwordgroup}{12}
      \bitbox{32}{handle}
    \end{leftwordgroup} \\
  \end{bytefield}
  \caption{\texttt{FileDir.SVCOpenCloseDirDesc}}\label{fig:vmsvc-dir}
\end{figure}

\begin{figure}[H]
  \centering
  \begin{bytefield}{32}
    \bitheader[endianness=big]{0-31} \\
    \begin{leftwordgroup}{0}
      \bitbox{32}{service id}
    \end{leftwordgroup} \\
    \begin{leftwordgroup}{4}
      \bitbox{32}{op}
    \end{leftwordgroup} \\
    \begin{leftwordgroup}{8}
      \bitbox{32}{handle}
    \end{leftwordgroup} \\
    \begin{leftwordgroup}{12}
      \bitbox{32}{done}
    \end{leftwordgroup} \\
    \begin{leftwordgroup}{16}
      \bitbox{32}{name}
    \end{leftwordgroup} \\
  \end{bytefield}
  \caption{\texttt{FileDir.SVCReadDirDesc}}\label{fig:vmsvc-readdir}
\end{figure}




\section{File}

The implemented system supports the following rudimentary file operations.

\begin{itemize}
\item{Open}

  This service allows files to be opened and managed from within the
  Oberon system.

  \begin{figure}[H]
    \centering
    \begin{bytefield}{32}
      \bitheader[endianness=big]{0-31} \\
      \begin{leftwordgroup}{0}
        \bitbox{32}{service id}
      \end{leftwordgroup} \\
      \begin{leftwordgroup}{4}
        \bitbox{32}{op}
      \end{leftwordgroup} \\

      \begin{leftwordgroup}{8}
        \bitbox{32}{flags}
      \end{leftwordgroup} \\
      \begin{leftwordgroup}{12}
        \bitbox{32}{mode}
      \end{leftwordgroup} \\
      \begin{leftwordgroup}{16}
        \bitbox{32}{pathname}
      \end{leftwordgroup} \\
      \begin{leftwordgroup}{20}
        \bitbox{32}{fp}
      \end{leftwordgroup} \\
    \end{bytefield}
    \caption{\texttt{FileDir.SVCFileOpenDesc}}\label{fig:vmsvc-openfile}
  \end{figure}

\item{Close}

  Resources no longer used by Oberon are automatically garbage
  collected.  But, the host OS must still be notified when a resource
  is no longer needed.  Consequently, an open file can be closed by
  using this service.

  \begin{figure}[H]
    \centering
    \begin{bytefield}{32}
      \bitheader[endianness=big]{0-31} \\
      \begin{leftwordgroup}{0}
        \bitbox{32}{service id}
      \end{leftwordgroup} \\
      \begin{leftwordgroup}{4}
        \bitbox{32}{op}
      \end{leftwordgroup} \\
      \begin{leftwordgroup}{8}
        \bitbox{32}{fp}
      \end{leftwordgroup} \\
    \end{bytefield}
    \caption{\texttt{FileDir.SVCFileCloseDesc}}\label{fig:vmsvc-closefile}
  \end{figure}

\item{Read / Write}

  Data can be read from, and written to, an open file by using this
  service descriptor.

  \begin{figure}[H]
    \centering
    \begin{bytefield}{32}
      \bitheader[endianness=big]{0-31} \\
      \begin{leftwordgroup}{0}
        \bitbox{32}{service id}
      \end{leftwordgroup} \\
      \begin{leftwordgroup}{4}
        \bitbox{32}{op}
      \end{leftwordgroup} \\
      \begin{leftwordgroup}{8}
        \bitbox{32}{fp}
      \end{leftwordgroup} \\
      \begin{leftwordgroup}{12}
        \bitbox{32}{bytes}
      \end{leftwordgroup} \\
      \begin{leftwordgroup}{16}
        \bitbox{32}{buffer}
      \end{leftwordgroup} \\
      \begin{leftwordgroup}{20}
        \bitbox{32}{result}
      \end{leftwordgroup} \\
    \end{bytefield}
    \caption{\texttt{FileDir.SVCFileReadWriteDesc}}\label{fig:vmsvc-readfile}
  \end{figure}


\item{Remove}

  A file can be removed from the host OS' file system using this
  service descriptor.

  \begin{figure}[H]
    \centering
    \begin{bytefield}{32}
      \bitheader[endianness=big]{0-31} \\
      \begin{leftwordgroup}{0}
        \bitbox{32}{service id}
      \end{leftwordgroup} \\
      \begin{leftwordgroup}{4}
        \bitbox{32}{op}
      \end{leftwordgroup} \\
      \begin{leftwordgroup}{8}
        \bitbox{32}{pathname}
      \end{leftwordgroup} \\
      \begin{leftwordgroup}{12}
        \bitbox{32}{result}
      \end{leftwordgroup} \\
    \end{bytefield}
    \caption{\texttt{FileDir.SVCFileUnlinkDesc}}\label{fig:vmsvc-unlink}
  \end{figure}

\item{Rename}

  A host OS file can be renamed using this service descriptor.

  \begin{figure}[H]
    \centering
    \begin{bytefield}{32}
      \bitheader[endianness=big]{0-31} \\
      \begin{leftwordgroup}{0}
        \bitbox{32}{service id}
      \end{leftwordgroup} \\
      \begin{leftwordgroup}{4}
        \bitbox{32}{op}
      \end{leftwordgroup} \\
      \begin{leftwordgroup}{8}
        \bitbox{32}{old}
      \end{leftwordgroup} \\
      \begin{leftwordgroup}{12}
        \bitbox{32}{new}
      \end{leftwordgroup} \\
      \begin{leftwordgroup}{16}
        \bitbox{32}{result}
      \end{leftwordgroup} \\
    \end{bytefield}
    \caption{\texttt{FileDir.SVCFileRenameDesc}}\label{fig:vmsvc-rename}
  \end{figure}


\item{Seek}

  The current read / write position within an open host OS file can be
  changed using this service descriptor.

  \begin{figure}[H]
    \centering
    \begin{bytefield}{32}
      \bitheader[endianness=big]{0-31} \\
      \begin{leftwordgroup}{0}
        \bitbox{32}{service id}
      \end{leftwordgroup} \\
      \begin{leftwordgroup}{4}
        \bitbox{32}{op}
      \end{leftwordgroup} \\

      \begin{leftwordgroup}{8}
        \bitbox{32}{fp}
      \end{leftwordgroup} \\
      \begin{leftwordgroup}{12}
        \bitbox{32}{pos}
      \end{leftwordgroup} \\
      \begin{leftwordgroup}{16}
        \bitbox{32}{mode}
      \end{leftwordgroup} \\
      \begin{leftwordgroup}{20}
        \bitbox{32}{newpos}
      \end{leftwordgroup} \\
    \end{bytefield}
    \caption{\texttt{FileDir.SVCFileSeekDesc}}\label{fig:vmsvc-seek}
  \end{figure}


\item{Make Temporary}

  A temporary host OS file can be created using this service descriptor.

  \begin{figure}[H]
    \centering
    \begin{bytefield}{32}
      \bitheader[endianness=big]{0-31} \\
      \begin{leftwordgroup}{0}
        \bitbox{32}{service id}
      \end{leftwordgroup} \\
      \begin{leftwordgroup}{4}
        \bitbox{32}{op}
      \end{leftwordgroup} \\
      \begin{leftwordgroup}{8}
        \bitbox{32}{template}
      \end{leftwordgroup} \\
      \begin{leftwordgroup}{12}
        \bitbox{32}{fd}
      \end{leftwordgroup} \\
    \end{bytefield}
    \caption{\texttt{FileDir.SVCMkstempDesc}}\label{fig:vmsvc-mkstemp}
  \end{figure}
\end{itemize}


\section{FillMemory}

This service  speeds filling memory with a constant value.

\begin{figure}[H]
  \centering
  \begin{bytefield}{32}
    \bitheader[endianness=big]{0-31} \\
    \begin{leftwordgroup}{0}
      \bitbox{32}{service id}
    \end{leftwordgroup} \\
    \begin{leftwordgroup}{4}
      \bitbox{32}{adr}
    \end{leftwordgroup} \\
    \begin{leftwordgroup}{8}
      \bitbox{32}{size}
    \end{leftwordgroup} \\
    \begin{leftwordgroup}{12}
      \bitbox{32}{val}
    \end{leftwordgroup} \\
  \end{bytefield}
  \caption{\texttt{Kernel.VMServiceFillMemoryDesc}}\label{fig:vmsvc-fillmemory}
\end{figure}


\section{EarlyHwdTrap}

If a hardware fault occurs before the Oberon system is fully booted,
the virtual \ac{CPU} will still jump to the address in \creg{1},
regardless its initialization state.  The bootstrapping system must
not raise hardware fault, and therefore does not provide a formal
handler for them.  If they occur before the system is \emph{ready},
this service will terminate the bootstrapping process.
\Figref{vmsvc-hwdtrap}, shows the descriptor layout for this early
system trap system.

\begin{figure}[H]
  \centering
  \begin{bytefield}{32}
    \bitheader[endianness=big]{0-31} \\
    \begin{leftwordgroup}{0}
      \bitbox{32}{service id}
    \end{leftwordgroup} \\
    \begin{leftwordgroup}{4}
      \bitbox{32}{ip}
    \end{leftwordgroup} \\
    \begin{leftwordgroup}{8}
      \bitbox{32}{\ac{SFP}}
    \end{leftwordgroup} \\
    \begin{leftwordgroup}{12}
      \bitbox{32}{trap id}
    \end{leftwordgroup} \\
    \begin{leftwordgroup}{16}
      \bitbox{32}{cr2}
    \end{leftwordgroup} \\
  \end{bytefield}
  \caption{\texttt{Kernel.VMServiceHwdTrapDesc}}\label{fig:vmsvc-hwdtrap}
\end{figure}


% \section{Environment} Not implemented; not documented.

