\chap{Instruction Encodings}

The processor has a small set of ways in which an instruction can be
encoded, which greatly contributes to the simplicity of the
architecture.  Several of the encodings have been borrowed from the
\mips architecture.

The following sections show the different classes of instructions and
their generic encodings.

Instructions must be aligned on a four byte boundary.

Bits 31\dots26 specify the instruction class, while bits 4\dots0
generally holds the instruction opcode.

%%%%%%
\section{Branching}

\begin{figure}[h!]
  \centering
    \jumpbox{\asmreg{0}}{opc}\usebox{\jumptypebox}
    \caption{Branch Instruction Format}
    \label{fig:branching-format}
\end{figure}

The instructions in this class, shown in \figref{branching-format}
allow control transfer, and are usually used in conjunction with
conditional expressions.


%%%%%%
\section{Bit Test}
The \emph{bit test} instructions facilitate easily determining the
value of a single bit in a register or word of memory.  There are two
encodings, one that operates with two input registers, and one that
uses an immediate value.

\begin{figure}[h]
  \centering
  \btbox{\sr{0}}{0} \usebox{\bttypebox}
  \caption{Register Bit Test Instruction Format}
  \label{fig:bit-test-format}
\end{figure}

Both formats result in the value of the tested bit being written into
the destination register.  The format of the two register instruction
is shown in \figref{bit-test-format}, while the format of the
immediate-register instruction is shown in
\figref{bit-test-immediate-format}.

\begin{figure}[h]
  \centering
  \btbox{C}{1} \usebox{\bttypebox}
  \caption{Immediate Bit Test Instruction Format}
  \label{fig:bit-test-immediate-format}
\end{figure}

%%%%%%
\section{Conditional Set}

Conditional set instructions are used to turn the result of a
comparison into a boolean value.  They are encoded as shown in
\figref{conditional-set-format}.

\begin{figure}[h!]
  \centering
\cbox{opc}\usebox{\ctypebox}
    \caption{Conditional Set Instruction Format}
    \label{fig:conditional-set-format}
\end{figure}

%%%%%%
\section{Control Register}

\acp{CR} can be read and written.  There is one instruction encoding
for each.

\begin{itemize}
\item Read \ac{CR}

  \begin{figure}[H]
    \centering
    \lcrbox\usebox{\lcrtypebox}     % Load register from CR
    \caption{Store Control Register Instruction Format}
    \label{fig:store-control-register-format}
  \end{figure}

\item Write \ac{CR}

  \begin{figure}[h!]
    \centering
      \scrbox\usebox{\scrtypebox}     % Store to CR from register
      \caption{Load Control Register Instruction Format}
      \label{fig:load-control-register-format}
  \end{figure}
\end{itemize}




%%%%%%
\section{General Register}

These instructions allow freely mixing the two different types of
arithmetic registers.  \Figref{general-register-format} shows the
encoding of instructions of this class, and \figref{register-bank}
shows the meanings of the \emph{register bank (b)} fields.

\begin{figure}[h!]
  \centering
    \gregbox{opc}{b}{b}{b}\usebox{\gregtypebox}
    \caption{General Register Instruction Format}
    \label{fig:general-register-format}
\end{figure}

The operation occurs using the larger of the two specified input
register types.  The smaller register value is converted to the larger
value, the operation is evaluated, and then the value is appropriately
converted and stored into the specified destination register.

\begin{figure}[h!]
  \centering
    \begin{tabular}{r|l}
      Encoding & Interpretation \\
      \hline 0 & four byte integer register. \\
      1 & eight byte IEEE 754 \emph{double} register. \\
    \end{tabular}
    \caption{Register Bank Specification}
    \label{fig:register-bank}
\end{figure}


%%%%%%
\section{Floating Point Register}

\Figref{float-register-format} shows the encoding of instructions
that operate upon integer registers. This class of instructions
performs an operation with two integer registers and writes the result
to the destination register.

\begin{figure}[h!]
  \centering
  \fprbox{opc}\usebox{\fprtypebox}
  \caption{Floating Point Register Instruction Format}
  \label{fig:float-register-format}
\end{figure}


%%%%%%
\section{Integer Register}

\Figref{integer-register-format} shows the encoding of instructions
that operate upon integer registers. This class of instructions
performs an operation with two integer registers and writes the result
to the destination register.

\begin{figure}[h!]
  \centering
  \rbox{opc}\usebox{\rtypebox}
  \caption{Integer Register Instruction Format}
  \label{fig:integer-register-format}
\end{figure}


%%%%%%
\section{Jump Register and Link}

This instruction is used for control transfer instructions that jump
to the address in the source register and write a \emph{return
  address} into the destination register.  The encoding format is
shown in \figref{jump-register-and-link-format}.

\begin{figure}[h!]
  \centering
  \jralbox\usebox{\jraltypebox}
    \caption{Jump Register and Link Instruction Format}
    \label{fig:jump-register-and-link-format}
\end{figure}


%%%%%%
\section{Miscellaneous}

\begin{figure}[h]
  \centering
\miscbox{opc}{0}{0}\usebox{\mtypebox}
    \caption{Miscellaneous Instruction Format}
    \label{fig:misc-format}
\end{figure}

The miscellaneous instruction encoding is used by instructions that do
not fit into any other instruction class.  The encoding is shown in
\figref{misc-format}.


%%%%%%
\section{Register Immediate}


This class of instructions allows immediate data to be loaded into a
register.  There are three types of immediates that can be loaded:

\begin{itemize}
\item four byte integer
  \begin{figure}[h!]
    \centering
      \regmemimmbox{9}\usebox{\regmemtypebox}
      \caption{Register Immediate Instruction Format}
  \end{figure}

\item four byte single precision real

  \begin{figure}[h!]
    \centering
      \regmemimmfloatbox{5}\usebox{\regmemimmfloattypebox}
      \caption{Four Byte Single Precision Real Immediate Format}
      \label{fig:four-byte-float-format}
  \end{figure}

\item eight byte double precision real
  \begin{figure}[h!]
    \centering
      \regmemimmdoublebox{3}\usebox{\regmemimmdoubletypebox}
      \caption{Eight Byte Double Precision Real Immediate Format}
      \label{fig:eight-byte-float-format}
  \end{figure}
\end{itemize}

%%%%%%
\section{Register Memory}\label{sect:register-memory}.

This class of instructions facilitates moving data to and from system
memory into \acp{GPR}.  The encoding is shown in
\figref{register-memory-format}.

\begin{figure}[H]
  \centering
    \regmembox{opc}\usebox{\regmemtypebox}
    \caption{Register Memory Instruction Format}
    \label{fig:register-memory-format}
\end{figure}

The semantics of the \texttt{S} field, a multiplicative scaling factor
applied to the index register, are shown in
\figref{register-memory-scale}.

\begin{figure}[H]
  \centering
  \begin{tabular}{l|l}
    S &  Scale Factor\\
    \hline
    0 & 1 \\
    1 & 2 \\
    2 & 4 \\
    3 & 8
  \end{tabular}
  \caption{Multiplicative Scaling of Index Register}
  \label{fig:register-memory-scale}
\end{figure}

The address on which the instruction operates is computed as shown
 in \figref{register-memory-address}.

\begin{figure}[H]
  \centering
    \begin{math}
      \begin{array}{lcl}
        \textrm{address} & \becomes & \sr{base} + \sr{index} \times 2^{\texttt{S}} + \textrm{signed offset} \\
      \end{array}
    \end{math}
  \caption{Effective Address Computation}\label{fig:register-memory-address}
\end{figure}


%%%%%%
\section{Sign Extension}

The sign extension instruction class, shown in
\figref{sign-extend-format}, either copies bit 7 to bits ${8..31}$ or
copies bit 15 to bits ${16..31}$.  The encoding for this instruction
class is shown in \figref{sign-extend-format}.

\begin{figure}[H]
  \centering
    \extbox{opc}\usebox{\exttypebox}
    \caption{Sign Extension Instruction Format}
    \label{fig:sign-extend-format}
\end{figure}

%%%%%%
\section{Stack Operations}

\begin{figure}[h!]
  \centering
    \stkbox{opc}{\sr{0}}\usebox{\stktypebox}
    \caption{Stack Instruction Format}
    \label{fig:stack-format}
\end{figure}

The stack instructions allow \texttt{push} operations to move register
constents onto the stack, and \emph{pop} operations to move stack
contents into registers.  The encoding is shown in
\figref{stack-format}.


%%%%%%
\section{System Register}

The special system registers allow programs to access internal
\ac{CPU} state.  For example, the number of cycles that the \ac{CPU}
has executed since power-on.  System registers may be read, but cannot
be written.  The instruction format is show in \figref{sysreg-format}.


\begin{figure}[H]
  \centering
  \sregbox{opc}\usebox{\sregtypebox}
    \caption{System Register Instruction Format}
    \label{fig:sysreg-format}
\end{figure}

%%%%%%
\section{System Trap}

The system trap instructions provide support for Oberon-specific
runtime features, as follows:

\begin{itemize}
\item \texttt{NIL} Pointer Dereference Checking

  \begin{figure}[H]
    \centering
    \trapbox{0}{20}{0}\usebox{\traptypebox}
    \caption{\texttt{trapnil} Trap Instruction Format}
  \end{figure}

\item Integer Range Checking
  \begin{figure}[H]
    \centering
    \trapbox{0}{12}{C}\usebox{\traptypebox}
    \caption{\texttt{traprange} Instruction Format}
  \end{figure}


\item Array Index Bounds Checking
  \begin{figure}[H]
    \centering
    \trapbox{\sr{1}}{13}{0}\usebox{\traptypebox}
    \caption{\texttt{traparray} Instruction Format}
  \end{figure}
\end{itemize}
