\chap{Runtime Environment}
When powered-on, all instruction execution, data reads and data writes
takes place within the memory allocated to the \ac{CPU}.  This memory
is known as \ac{CPU} memory.  The memory is split into two parts for
the Oberon runtime system: the stack, and the heap.  The heap is where
all code and data is placed, while the stack holds data for the
activation frames for all open procedures.

The \skl address space is a full 32 bits, but the Oberon system
reserves the top two bits of an address for use by the \ac{GC}.  This
reservation results in the maximal memory layout shown in
\figref{accessible-memory}, though in practice the accessible memory
will be smaller than shown here due to host OS limitations.

\begin{figure}[h]
  \centering
    \begin{bytefield}{32}
      \memsection{FFFFFFFFH}{40000000H}{3}{-- inaccessible --} \\
      \memsection{3FFFFFFFH}{00000000H}{3}{-- accessible --}
    \end{bytefield}
  \caption{Accessible Memory Layout}\label{fig:accessible-memory}
\end{figure}

The accessible memory is divided into the \emph{stack} and
\emph{heap}, based on runtime options provided when the Oberon
environment is started.  The default memory configuration on Linux
is shown in \figref{linux-default-memory}.


\begin{figure}
  \centering
    \begin{bytefield}{32}
      \memsection{FFFFFFFFH}{08000000H}{3}{-- inaccessible --} \\
      \memsection{07FFFFFFH}{04200000H}{3}{heap} \\
      \memsection{041FFFFFH}{04000000H}{3}{stack} \\
      \memsection{03FFFFFFH}{00000000H}{3}{-- inaccessible --}
    \end{bytefield}
  \caption{Default Memory Layout}\label{fig:linux-default-memory}
\end{figure}


\section{Procedure Arguments}

The Oberon compiler generates code that pushes procedure arguments on
the stack from left-to-right.  That is, a procedure's arguments are
pushed onto the stack in the order in which they were declared in source.

\section{Procedure Stack Layout}

\Figref{example-stack-layout} provides an example of how the stack
appears with several open procedures on it.  Note that because the
stack grows down, procedure arguments are accessed at positive offsets
from the \ac{SFP}, while local variables are accessed at negative
offsets.

\newlength{\stkSFP}
\settowidth{\stkSFP}{~R29~}
\newcommand{\stkadr}[1]{#1{\hspace{\stkSFP}{}~}}
\newcommand{\stkrel}[2]{#1}
\newcommand{\sfpadr}[1]{#1~R29}
\newcommand{\stkchev}[1]{\small{\lchevron#1\rchevron}}

\begin{figure}
  \centering
  \begin{bytefield}{32}
      \memsection{\stkadr{41FFFC8H}}{\stkadr{41FFFCBH}}{2}{\stkchev{argument}} \\
      \begin{rightwordgroup}
        {Frame of \texttt{CTNest.Global}}
        \memsection{\sfpadr{41FFFCCH}}{\stkadr{41FFFCFH}}{2}{41FFFD8H} \\
        \memsection{\stkadr{41FFFD0H}}{\stkadr{41FFFD3H}}{2}{\small{\texttt{CTNest.Test}}}
      \end{rightwordgroup} \\
      \begin{rightwordgroup}
        {Argument of \texttt{CTNest.Global}}
        \memsection{\stkadr{41FFFD4H}}{\stkadr{41FFFD7H}}{2}{\stkchev{R29 + 8}}
      \end{rightwordgroup} \\
      \begin{rightwordgroup}
        {Frame of \texttt{CTNest.Test}}
        \memsection{\sfpadr{41FFFD8H}}{\stkadr{41FFFDBH}}{2}{41FFFF4H} \\
        \memsection{\stkadr{41FFFDCH}}{\stkadr{41FFFDFH}}{2}{\small{\texttt{Modules.Init}}}
      \end{rightwordgroup} \\
      \begin{rightwordgroup}
        {Local variables of \texttt{Modules.Init}}
        \memsection{\stkadr{41FFFE0H}}{\stkadr{41FFFE3H}}{2}{\stkchev{R29 - 14H}} \\
        \memsection{\stkadr{41FFFE4H}}{\stkadr{41FFFE7H}}{2}{\stkchev{R29 - 10H}} \\
        \memsection{\stkadr{41FFFE8H}}{\stkadr{41FFFEBH}}{2}{\stkchev{R29 - 0CH}} \\
        \memsection{\stkadr{41FFFECH}}{\stkadr{41FFFECH}}{2}{\stkchev{R29 - 8H}} \\
        \memsection{\stkadr{41FFFF0H}}{\stkadr{41FFFF3H}}{2}{\stkchev{R29 - 1H}}
      \end{rightwordgroup} \\
      \begin{rightwordgroup}
        {Frame of \texttt{Modules.Init}}
        \memsection{\sfpadr{41FFFF4H}}{\stkadr{41FFFF7H}}{2}{41FFFFCH} \\
        \memsection{\stkadr{41FFFF8H}}{\stkadr{41FFFFBH}}{2}{\small{\texttt{Modules.Modules}}}
      \end{rightwordgroup} \\
      \begin{rightwordgroup}
        {Frame of \texttt{Modules} initialization}
      \memsection{\sfpadr{41FFFFCH}}{\stkadr{41FFFFFH}}{2}{4200010H} \\
      \memsection{\stkadr{4200000H}}{\stkadr{4200003H}}{2}{\small{\texttt{Kernel.BootstrapModuleInit}}}
      \end{rightwordgroup} \\
      \memsection{\stkadr{4200004H}}{\stkadr{420000BH}}{3}{} \\
      \memsection{\stkadr{420000CH}}{\stkadr{420000FH}}{2}{\stkchev{top~of~stack}}
    \end{bytefield}
  \caption{Example Stack Layout}\label{fig:example-stack-layout}
\end{figure}

\section{Nested Procedure Stack}
The stack layout for the activation of a nested procedure is similar
to a non-nested procedure.  The only difference being that a nested
procedure also has a \emph{dyanmic link} to the enclosing procedure's
stack frame.  This allows the compiler to generate code to access
local variables from the parent's scope.
~Figref{example-stack-dynamic-link-layout} provides sample stack
layout for this scenario.

\begin{figure}[h]
  \centering
  \begin{bytefield}{32}
      \begin{rightwordgroup}
        {Local variables}
        \memsection{\stkadr{41FFF74H}}{\stkadr{41FFF77H}}{2}{\stkchev{R29 - 10H}} \\
        \memsection{\stkadr{41FFF78H}}{\stkadr{41FFF7BH}}{2}{\stkchev{R29 - 0CH}} \\
        \memsection{\stkadr{41FFF7CH}}{\stkadr{41FFF7FH}}{2}{\stkchev{R29 - 8H}} \\
        \memsection{\stkadr{41FFF80H}}{\stkadr{41FFF83H}}{2}{\stkchev{R29 - 4H}}
      \end{rightwordgroup} \\
      \begin{rightwordgroup}
        {Frame with dynamic link}
        \memsection{\sfpadr{41FFF84H}}{\stkadr{41FFF87H}}{2}{41FFFA0H} \\
        \memsection{\stkadr{41FFF88H}}{\stkadr{41FFF8BH}}{2}{return address} \\
        \memsection{\stkadr{41FFF8CH}}{\stkadr{41FFF8FH}}{2}{41FFFA0H}
      \end{rightwordgroup} \\
      \memsection{\stkadr{41FFF90H}}{\stkadr{41FFF93H}}{2}{\stkchev{argument}} \\
      \begin{rightwordgroup}
        {Local variables}
        \memsection{\stkadr{41FFF94H}}{\stkadr{41FFF93H}}{2}{\stkchev{R29 - 10H}} \\
        \memsection{\stkadr{41FFF98H}}{\stkadr{41FFF9BH}}{2}{\stkchev{R29 - 0CH}} \\
        \memsection{\stkadr{41FFF9CH}}{\stkadr{41FFF9FH}}{2}{\stkchev{R29 - 8H}}
      \end{rightwordgroup} \\
      \begin{rightwordgroup}
        {Frame with dynamic link}
        \memsection{\sfpadr{41FFFA0H}}{\stkadr{41FFFA3H}}{2}{41FFFB8H} \\
        \memsection{\stkadr{41FFFA4H}}{\stkadr{41FFFA7H}}{2}{return address} \\
        \memsection{\stkadr{41FFFA8H}}{\stkadr{41FFFABH}}{2}{41FFFB8H}
      \end{rightwordgroup}
    \end{bytefield}
  \caption{Example Stack Dynamic Link Layout}\label{fig:example-stack-dynamic-link-layout}
\end{figure}
