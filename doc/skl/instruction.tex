\chap{Instruction Reference}

The SKL instruction set is heavily influenced by the \mips and
\natsemi architectures, but also has instructions to specifically
supoprt Oberon and its runtime system.  All aspects of instruction
encoding and semantics are covered in this chapter.


\begin{instruction}{abs}
  \abstype{abs}
     {\aaddopc}
     {$\dr \becomes |\sr{1}|$}
     {Sets \dr to $|\sr{1}|$.}
\end{instruction}


\begin{instruction}{add}
\gregtype{add}{\aaddopc}{$\dr \becomes \sr{0} + \sr{1}$}{Adds \sr{0} and
\sr{1}, and places the result into \dr.}
\end{instruction}


\begin{instruction}{and}
  \rtype{and}{\randopc}{$\dr \becomes \sr{0} \wedge \sr{1}$}{Performs
    a bitwise and of \sr{0} and\sr{1}, and places the result into
    \dr.}
\end{instruction}


\begin{instruction}{arctan}
  \fprtype{arctan}
     {\arctanopc}
     {
       \begin{math}
         \begin{array}{lcl}
           \dr & \becomes & \arctan(\sr{0})
         \end{array}
       \end{math}
     }
     {Takes the arctangent of the value in \sr{0}.  The domain of the
     value in \texttt{\sr{0}}, as well as the result, conform to
     the \texttt{ANSI C} \texttt{atan} library function.

     \sr{0} and \dr are both floating point registers.}
\end{instruction}


\begin{instruction}{ash}
  \rtype{ash}{\rashopc}
     {
       \begin{math}
         \begin{array}{rcl}
           \sr{1} \geq 0 & \implies & \textrm{shift} \becomes \sr{1} \bmod 32        \\
              &          & \dr \becomes \sr{0} \times 2^\textrm{shift}   \\
           \sr{1} <    0 & \implies & \textrm{shift} \becomes \abs{\sr{1}} \bmod 32  \\
                         &          & \dr \becomes \sr{0} \bdiv {2^\textrm{shift}} \\
         \end{array}
       \end{math}
     }
     {
       Arithmetically shifts \sr{0} the number of bits specified by
       \sr{1} and stores the result in \dr.  When \sr{1} is postive,
       \sr{0} is left shifted, and when \sr{1} is negative, \sr{0} is
       right shifted.

       The right shift treats \sr{0} as a signed value.  All bits
       shifted into the uppermost bit will have the same value as the
       uppermost bit before the shift is executed.

       With each bit shifted left, the vacated bit zero will set
       to zero.
     }
\end{instruction}


\begin{instruction}{bitset}
  \rtype{bitset}{\rbitsetopc}
     {
       \begin{math}
         \begin{array}{rcl}
           \sr{0} \leq \sr{1} & \implies & \textrm{lo} \becomes \abs{\sr{0}} \bmod 32        \\
                              &          & \textrm{hi} \becomes \abs{\sr{1}} \bmod 32        \\
                              &          & \dr \becomes (2^{\textrm{hi} - \textrm{lo} + 1} - 1) \times 2^\textrm{lo} \\
           \sr{0} >    \sr{1} & \implies & \textrm{Software trap 12}
         \end{array}
       \end{math}
     }
     {
       Sets bits \sr{0} through, and including, bit \sr{1} in \dr.
       If \sr{1} is less than \sr{0}, then software fault 12 is
       raised.  See \figref{cr5-r1-value}.
     }
\end{instruction}


\begin{instruction}{bt}
  \btrrtype{bt}{\btrropc}
     {
       $\dr \becomes ({\sr{1}} \bdiv {2^{\textrm{\sr{0}} \bmod 32}}) \wedge 1$
     }
     {
       Sets \dr to the bit in \sr{1} specified by \sr{0}.  Only the low
       five bits of \sr{0} are used.
     }
\end{instruction}


\begin{instruction}{bti}
  \btcrtype{bti}
     {\btcropc}
     {
       $\dr \becomes ({\sr{1}} \bdiv {2^{\textrm{C}}}) \wedge 1$
     }
     {
       Tests if the bit specified by the immediate, \emph{C}, is set
       in \sr{1}, and sets \dr to the state of the bit.
     }
\end{instruction}


\begin{instruction}{btm}
  \btrrtype{btm}
     {\btmopc}
     {
       $\dr \becomes {(\texttt{MEMORY[\sr{1}]}} \bdiv {2^{\sr{0} \bmod 32}}) \wedge  1$
     }
     {
       Tests if the bit specified by \sr{0} is set in the memory
       referenced by \sr{1} and sets \dr to the state of the bit.  Only
       the low five bits of \sr{0} are used.
     }
\end{instruction}


\begin{instruction}{btmi}
  \btcrtype{btmi}{\btmiopc}
    {
      $\dr \becomes {(\texttt{MEMORY[\sr{1}]}} \bdiv {2^{\textrm{C}}}) \wedge  1$
    }
    {
      Tests if the bit specified by the immediate, \emph{C}, is set in
      the memory referenced by \sr{1} and sets \dr to the state of the
      bit.
    }
\end{instruction}


\begin{instruction}{btmc}
  \btrrtype{btmc}{\btmcopc}
    {$\dr \becomes {(\texttt{MEMORY[\sr{1}]}} \bdiv {2^{\sr{0} \bmod 32}}) \wedge  1$

     $\texttt{MEMORY[\sr{1}]} \becomes \texttt{MEMORY[\sr{1}]} \wedge \neg {2^{\textrm{\sr{0}}}}$}
    {
      Tests if the bit specified by \sr{0} is set in the memory
      referenced by \sr{1}, sets \dr to the state of the bit and
      clears the bit in memory.  Only the low five bits of \sr{0}
      are used.
    }
\end{instruction}


\begin{instruction}{btmci}
  \btcrtype{btmci}{\btmciopc}
    {$\dr \becomes {(\texttt{MEMORY[\sr{1}]}} \bdiv {2^{\textrm{C}}}) \wedge  1$

     $\texttt{MEMORY[\sr{1}]} \becomes \texttt{MEMORY[\sr{1}]} \wedge \neg {2^{\textrm{C}}}$}
    {Tests if the bit specified by \texttt{C} is set in the memory
      referenced by \sr{1}, sets \dr to the state of the bit and
      clears the bit in memory.}
\end{instruction}


%%%%
\begin{instruction}{btms}
  \btrrtype{btms}{\btmsopc}
    {$\dr \becomes {(\texttt{MEMORY[\sr{1}]}} \bdiv {2^{\sr{0} \bmod 32}}) \wedge  1$

     $\texttt{MEMORY[\sr{1}]} \becomes \texttt{MEMORY[\sr{1}]} \vee {2^{\textrm{\sr{0}}}}$}
    {Tests if the bit specified by \sr{0} is set in the memory
      referenced by \sr{1}, sets \dr to the state of the bit and
      sets the bit in memory.  Only the low five bits of \sr{0}
      are used.}
\end{instruction}


\begin{instruction}{btmsi}
  \btcrtype{btmsi}{\btmsiopc}
    {$\dr \becomes {(\texttt{MEMORY[\sr{1}]}} \bdiv {2^{\textrm{C}}}) \wedge  1$

     $\texttt{MEMORY[\sr{1}]} \becomes \texttt{MEMORY[\sr{1}]} \vee {2^{\textrm{C}}}$}
    {Tests if the bit specified by \texttt{C} is set in the memory
      referenced by \sr{1}, sets \dr to the state of the bit and
      sets the bit in memory.}
\end{instruction}


%%%%
%
% This instruction is not documented because it is not implemented.
%
%% \begin{instruction}{break}\label{inst:break}
%%   \todo{Should each instruction in this class get a special encoding,
%%     or should they be decoded in an ad hoc manner?}
%%   \mtype{break}{\mbreakopc}{0}{0}{Breakpoint.}{}
%% \end{instruction}

\begin{instruction}{cmp}\label{inst:cmp}
  \cmptype{cmp}{\acmpopc}
  {
    \begin{math}
      \dr \becomes \left\lbrace\begin{array}{lll}
      \textrm{bit}_0 & \textrm{\acs{ZF}} & \sr{0} - \sr{1} = 0 \\
      \textrm{bit}_1 & \textrm{\acs{SF}} & \sr{0} - \sr{1} < 0 \\
      \textrm{bit}_2 & \textrm{\acs{CF}} & \sr{0} < \sr{1} \\
      \textrm{bit}_3 & \textrm{\acs{OF}} &
      (\sr{0} \bdiv 2^{31} \neq \sr{1} \bdiv 2^{31}) \wedge \\
      & & (\sr{0} \bdiv 2^{31} \neq (\sr{0} - \sr{1}) \bdiv 2^{31}) \\
      \textrm{bit}_{31..4} & & \textrm{reserved} \\
      \end{array}
      \right.
    \end{math}
  }
  {
    Subtracts \sr{1} from \sr{0}, and sets \dr as shown
  }

\end{instruction}


\begin{instruction}{cmps}
  \rtype{cmps}
     {\rcmpsopc}
     {
       $\dr \becomes \textrm{See \xrefinst{cmp}}$
     }
     {
       Compares the two \acs{ASCIIZ} strings in the memory referenced
       by \sr{0} and \sr{1}.  The bytes of the string are compared
       one-by-one, and \dr is set to the result of the first non-equal
       comparison, or the final comparison between two \texttt{0X} bytes.
       The bits in \dr are formatted the same as for the \xrefinst{cmp}
       instruction.  The flags in \dr can be interpreted as if a comparison
       with \emph{0} had been performed, with the meaning as follows:

       \begin{tabularx}{\linewidth}{ll}
         $\dr < 0$ & $\textrm{String}_{\sr{0}}$ less than $\textrm{String}_{\sr{1}}$ \\
         $\dr = 0$ & $\textrm{String}_{\sr{0}}$ equal to $\textrm{String}_{\sr{1}}$ \\
         $\dr > 0$ & $\textrm{String}_{\sr{0}}$ greater than $\textrm{String}_{\sr{1}}$ \\
       \end{tabularx}
     }
\end{instruction}


\begin{instruction}{cos}
  \fprtype{cos}
     {\cosopc}
     {
       \begin{math}
         \begin{array}{lcl}
           \dr & \becomes & \cos(\sr{0})
         \end{array}
       \end{math}
     }
     {Takes the cosine of the value in \sr{0}.  The domain of the
     value in \texttt{\sr{0}}, as well as the result, conform to
     the \texttt{ANSI C} \texttt{cos} library function.

     \sr{0} and \dr are both floating point registers.}
\end{instruction}


%%%%
%
% This instruction is not documented because it is not implemented.
%
%% \begin{instruction}{di}
%%     \sregtype{di}{\sregdiopc}{Unknown}{Disable Interrupts.}

%%     \todo{What happens to interrupts which are pending when this
%%       instruction executes?  Does it take affect immediately or after
%%       one more instruction?}
%% \end{instruction}


\begin{instruction}{div}\label{inst:div}
  \gregtype{div}{\adivopc}
    {$\dr \becomes \sr{0} \bdiv \sr{1}$}
    {Computes \sr{0} divided by \sr{1} and stores the result in \dr.

    A \emph{divide by zero} fault is raised if \sr{1} is zero.

    For integers, the \texttt{div} and \xrefinst{mod} instructions
    are related as follows:

    \begin{math}
      \begin{array}{lcl}
        x \bmod y & \equiv  & x - y \lfloor {x \bdiv y} \rfloor \\
        y > 0 & \rightarrow & 0 \leq x \bmod y < y \\
        y = 0 & \rightarrow & undefined \\
        y < 0 & \rightarrow & y < x \bmod y \leq 0 \\
      \end{array}
    \end{math}}
\end{instruction}


%%%%
%
% This instruction is not documented because it is not implemented.
%
%% \begin{instruction}{ei}
%%     \sregtype{ei}{\sregeiopc}{Unknown}{Enable Interrupts.}

%%   \todo{When is the next interrupt delivered?  Immediately after
%%     \texttt{ei}?}
%% \end{instruction}

\begin{instruction}{enter}\label{inst:enter}
  \stkframetype{enter}{words}{\enteropc}
  {
    \begin{math}
      \begin{array}{lcl}
                                  \sr{30} & \becomes &  \sr{30} - 4 \\
        \texttt{MEMORY[\textrm{\sr{30}}]} & \becomes &  \sr{31} \\
                                  \sr{30} & \becomes &  \sr{30} - 4 \\
        \texttt{MEMORY[\textrm{\sr{30}}]} & \becomes &  \dr \\
                                      \dr & \becomes &  \sr{30} \\
                                  \sr{30} & \becomes &  \sr{30} - \textrm{words} \times 4 \\
      \end{array}
    \end{math}
  }
  {
    Creates a local stack frame upon procedure entry.

    The return address, in \sr{31}, and \dr are saved on the stack,
    and then the \ac{SP} is decreased by \texttt{words} four byte
    words to accommodate local procedure variables.

    Procedure arguments are accessed at positive offsets from \dr, and
    local variables are accessed at negative offsets.

    The second argument, \texttt{words}, is a 16 bit unsigned
    integer. In practice, \dr is \sr{29}.
  }
\end{instruction}



%%%%
%
% This instruction is not documented because it is not implemented.
%
%% \begin{instruction}{eret}\label{inst:eret}
%%   \todo{This reenables interrupts. Interrupts will be disabled when
%%     the fault is raised.}
%%   \mtype{eret}{\meretopc}{0}{0}{Return from interrupt.}{}

%%   \todo{Describe that the \emph{source register 0} field must be zero,
%%   becuase causes CR0 to be read for the return address.}
%% \end{instruction}


\begin{instruction}{exp}
  \fprtype{exp}
     {\expopc}
     {
       \begin{math}
         \begin{array}{lcl}
           \dr & \becomes & $$e^{\sr{0}}$$
         \end{array}
       \end{math}
     }
     {Takes the value of \emph{e} and raises to the power of the value
     in \sr{0}.  The domain of the value in \texttt{\sr{0}}, as well
     as the result, conform to the \texttt{ANSI C} \texttt{exp}
     library function.

     \sr{0} and \dr are both floating point registers.}
\end{instruction}


\begin{instruction}{j}
  \jjumptype{j}{0}{0}{\jopc}
     {

       This instruction performs an unconditional jump to
       \texttt{address}.

       The value of \texttt{address}, which is a four byte unsigned
       integer, must be aligned to four bytes.
     }
\end{instruction}

\begin{instruction}{jal}\label{inst:jal}
  \jaljumptype{jal}{31}{0}{\jalopc}
     {
       This instruction unconditionally jumps to \texttt{address} and
       sets the \ac{LR} to the address of the next instruction.

       A \emph{call} / \emph{return} semantic can be created with this
       instruction and \xrefinst{jral}.
     }
\end{instruction}


\begin{instruction}{jeq}
  \jumptype
     {jeq}
     {\jjeqopc}
     {
       Jumps to the specified address if the comparison result in \sr{0}
       indicates equality, otherwise continues at the next
       instruction.
     }
     {
       \sr{0}.ZF = 1
     }
\end{instruction}


\begin{instruction}{jge}
  \jumptype
     {jge}
     {\jjgeopc}
     {This instruction treats the \xrefinst{cmp} result as if both
       inputs were signed numbers, and will consequently jump to the
       specified address if the result in \sr{0} indicates that
       comparison's \sr{0} was \emph{greater or equal} to \sr{1}.
       Otherwise continues at the next instruction.}
     {\sr{0}.SF = \sr{0}.OF}
\end{instruction}

\begin{instruction}{jgeu}
  \jumptype
     {jgeu}
     {\jjgeuopc}
     {This instruction treats the \xrefinst{cmp} result as if both
       inputs were unsigned numbers, and will consequently jump to the
       specified address if the result in \sr{0} indicates that
       comparison's \sr{0} was \emph{greater or equal} to \sr{1}.
       Otherwise continues at the next instruction.}
     {\sr{0}.CF = 0}
\end{instruction}

\begin{instruction}{jgt}
  \jumptype
     {jgt}
     {\jjgtopc}
     {This instruction treats the \xrefinst{cmp} result as if both
       inputs were signed numbers, and will consequently jump to the
       specified address if the result in \sr{0} indicates that
       comparison's \sr{0} was \emph{greater than} \sr{1}.
       Otherwise continues at the next instruction.}
     {\sr{0}.ZF = 0 \wedge \sr{0}.SF = \sr{0}.OF}
\end{instruction}

\begin{instruction}{jgtu}
  \jumptype
     {jgtu}
     {\jjgtuopc}
     {This instruction treats the \xrefinst{cmp} result as if both
       inputs were unsigned numbers, and will consequently jump to the
       specified address if the result in \sr{0} indicates that
       comparison's \sr{0} was \emph{greater than} \sr{1}.
       Otherwise continues at the next instruction.}
     {\sr{0}.CF = 0 \wedge \sr{0}.ZF = 0}
\end{instruction}

\begin{instruction}{jle}
  \jumptype
     {jle}
     {\jjleopc}
     {This instruction treats the \xrefinst{cmp} result as if both
       inputs were signed numbers, and will consequently jump to the
       specified address if the result in \sr{0} indicates that
       comparison's \sr{0} was \emph{less or equal to} \sr{1}.
       Otherwise continues at the next instruction.}
     {\sr{0}.ZF = 1 \vee \sr{0}.SF \neq \sr{0}.OF}
\end{instruction}

\begin{instruction}{jleu}
  \jumptype
     {jleu}
     {\jjleuopc}
     {This instruction treats the \xrefinst{cmp} result as if both
       inputs were unsigned numbers, and will consequently jump to the
       specified address if the result in \sr{0} indicates that
       comparison's \sr{0} was \emph{less or equal to} \sr{1}.
       Otherwise continues at the next instruction.}
     {\sr{0}.CF = 1 \vee \sr{0}.ZF = 1}
\end{instruction}

\begin{instruction}{jlt}
  \jumptype
     {jlt}
     {\jjltopc}
     {This instruction treats the \xrefinst{cmp} result as if both
       inputs were signed numbers, and will consequently jump to the
       specified address if the result in \sr{0} indicates that
       comparison's \sr{0} was \emph{less than} \sr{1}.
       Otherwise continues at the next instruction.}
     {\sr{0}.SF \neq \sr{0}.OF}
\end{instruction}

\begin{instruction}{jltu}
  \jumptype
     {jltu}
     {\jjltuopc}
     {This instruction treats the \xrefinst{cmp} result as if both
       inputs were unsigned numbers, and will consequently jump to the
       specified address if the result in \sr{0} indicates that
       comparison's \sr{0} was \emph{less than} \sr{1}.
       Otherwise continues at the next instruction.}
     {\sr{0}.CF = 1}
\end{instruction}

\begin{instruction}{jne}
  \jumptype
     {jne}
     {\jjneopc}
     {Jumps to the specified address if the comparison result in \sr{0}
       indicates the compared values were \emph{not equal}, otherwise continues
       at the next instruction.}
     {\sr{0}.ZF = 0}
\end{instruction}


\begin{instruction}{jral}\label{inst:jral}
  \jraltype
      {Unconditionally jumps to contents of \sr{0} and saves the
        return address in \dr.

        The address contained in \sr{0} must be aligned to a word
        boundary or a \emph{misaligned instruction} fault will
        occur at the address of this instruction.}
\end{instruction}


\begin{instruction}{la}
  \laregmemtype{la}
     {\laopc}
     {
       This instructions computes a value using \sr{base}, \sr{index},
       \texttt{S} and \texttt{signed offset} then stores the
       result in \dr.
     }
\end{instruction}


\begin{instruction}{lb}
  \regmemtype{lb}
     {\lbopc}
     {
       \begin{math}
         \begin{array}{lcl}
           {\dr}_{[7..0]} & \becomes & \texttt{MEMORY}[\sr{base} +
            (\sr{index} \times 2^{\texttt{S}}) + \texttt{signed offset}] \\
           {\dr}_{[31..8]} & \becomes & {\dr}_{[7]} \\
         \end{array}
       \end{math}
     }
     {
       Loads and sign extends a byte from memory into a 32-bit
       register.  To sign extend, bit seven of the original data
       is copied to bits ${8..31}$ of the destination register.
     }
\end{instruction}

\begin{instruction}{lbu}
  \regmemtype{lbu}
     {\lbuopc}
     {
       \begin{math}
         \begin{array}{lcl}
           {\dr}_{[7..0]} & \becomes & \texttt{MEMORY}[\sr{base} +
            (\sr{index} \times 2^{\texttt{S}}) + \texttt{signed offset}] \\
           {\dr}_{[31..8]} & \becomes & 0 \\
         \end{array}
       \end{math}
     }
     {
       Loads and zero extends a byte from memory into a 32-bit
       register.  To zero extend, zero is copied to bits ${8..31}$
       of the destination register.
     }
\end{instruction}


\begin{instruction}{lcc}
  \sregtype{lcc}
     {\sreglccopc}
     {$\dr \becomes \textrm{Fetch Instruction Count}$}
     {Loads the internal count of the number of instructions fetched
       since power-on into \dr.}{\dr}
\end{instruction}


\begin{instruction}{lcr}\label{inst:lcr}
  \lcrtype{lcr}
     {
       Loads the specifiied control register into \dr.
     }
\end{instruction}


\begin{instruction}{ld}
  \regmemtype{ld}
     {\ldopc}
     {
       \begin{math}
         \begin{array}{lcl}
           {\dr} & \becomes & \texttt{MEMORY}[\sr{base} + (\sr{index} \times 2^{\texttt{S}}) + \texttt{signed offset}] \\
         \end{array}
       \end{math}
     }
     {
       Loads an eight byte double precision real value
       into the destination register from the memory
       address specified.
     }
\end{instruction}


\begin{instruction}{ldi}
  \regmemimmdoubletype{ldi}
     {\ldiopc}
     {

       Loads a double precision real value into the destination
       register from data fields in the instruction.

     }
\end{instruction}


\begin{instruction}{leave}\label{inst:leave}
  \stkframetype{leave}{words}{\leaveopc}
     {
% Rd is not set until end
       \begin{math}
         \begin{array}{lcl}
           \sr{\lchevron{internal}\rchevron}     & \becomes & \texttt{MEMORY[\textrm{\dr}]} \\
           \sr{31} & \becomes & \texttt{MEMORY[\textrm{\dr} + 4]} \\
           \dr     & \becomes & \sr{\lchevron{internal}\rchevron} \\
           \sr{30} & \becomes & \dr + (2 + \textrm{words}) \times 4 \\
         \end{array}
       \end{math}
     }
     {This instruction desconstructs the local stack frame created
       by \texttt{enter} by:

       \begin{itemize}
       \item Restoring \sr{29}.
       \item Restoring \dr from the stack.
       \item Restoring \sr{31} from the stack.
       \item Removing  \texttt{words} procedure arguments provided by the caller from the stack.
       \end{itemize}
     }
\end{instruction}

\begin{instruction}{lf}
  \regmemtype{lf}
     {\lfopc}
     {
       \begin{math}
         \begin{array}{lcl}
           {\dr} & \becomes & \texttt{MEMORY}[\sr{base} + (\sr{index} \times 2^{\texttt{S}}) + \texttt{signed offset}] \\
         \end{array}
       \end{math}
     }
     {
       Loads a four byte single precision real value
       into the destination register from the memory
       address specified.
     }
\end{instruction}


\begin{instruction}{lfi}
  \regmemimmfloattype{lfi}
     {\lfiopc}
     {
       Loads a single precision real value into the
       destination register from data field in the instruction.
     }
\end{instruction}


\begin{instruction}{lh}
  \regmemtype{lh}
     {\lhopc}
     {
       \begin{math}
         \begin{array}{lcl}
           {\dr}_{[15..0]} & \becomes & \texttt{MEMORY}[\sr{base} +
            (\sr{index} \times 2^{\texttt{S}}) + \texttt{signed offset}] \\
           {\dr}_{[31..16]} & \becomes & {\dr}_{[15]} \\
         \end{array}
       \end{math}
     }
     {
       Loads and sign extends a 16 bit signed halfword from memory
       into a 32 bit register.  To sign extend, bit 15 of
       the original data is copied to bits ${16..31}$ of the
       destination register.
     }
\end{instruction}

\begin{instruction}{lhu}
  \regmemtype{lhu}
     {\lhuopc}
     {
       \begin{math}
         \begin{array}{lcl}
           {\dr}_{[15..0]} & \becomes & \texttt{MEMORY}[\sr{base} +
            (\sr{index} \times 2^{\texttt{S}}) + \texttt{signed offset}] \\
           {\dr}_{[31..16]} & \becomes & 0 \\
         \end{array}
       \end{math}
     }
     {
       Loads and zero extends a 16-bit halfword from memory into a
       32-bit register.  To zero extend, zero is copied to bits
       ${16..31}$ of the destination register.
     }
\end{instruction}


\begin{instruction}{ln}
  \fprtype{ln}
     {\lnopc}
     {
       \begin{math}
         \begin{array}{lcl}
           \dr & \becomes & \ln(\sr{0})
         \end{array}
       \end{math}
     }
     {Takes the natural logarithm of the value in \sr{0}.  The domain
     of the value in \texttt{\sr{0}}, as well as the result, conform
     to the \texttt{ANSI C} \texttt{log} library function.

     \sr{0} and \dr are both floating point registers.}
\end{instruction}


\begin{instruction}{lsh}
\rtype{lsh}
   {\rlshopc}
   {
     \begin{math}
       \begin{array}{rcl}
         \sr{1} \geq 0 & \implies & shift \becomes \sr{1} \bmod 32  \\
                       &          & \dr \becomes \sr{0} \times 2^{shift}  \\
         \sr{1} <    0 & \implies & shift \becomes \abs{\sr{1}} \bmod 32        \\
                       &          & \dr \becomes {\sr{0}} \bdiv {2^{shift} }\\
       \end{array}
     \end{math}
   }
   {
     Logically shifts \sr{0} the number of bits specified by \sr{1}
     and stores the result in \dr.

     When \sr{1} is negative, a right shift is used, and when \sr{1}
     is positive, a left shift is used.

     The right shift treats \sr{0} as an unsigned value, and so all
     bits shifted into the upper bit will be zero.  All bits shifted
     in at bit zero will have the value zero.
   }
\end{instruction}


\begin{instruction}{lw}
  \regmemtype{lw}
     {\lwopc}
     {
       \begin{math}
         \begin{array}{lcl}
           {\dr} & \becomes & \texttt{MEMORY}[\sr{base} +
            (\sr{index} \times 2^{\texttt{S}}) + \texttt{signed offset}] \\
         \end{array}
       \end{math}
     }
     {
       Loads 32-bit word from memory into a 32-bit register.
     }
\end{instruction}


\begin{instruction}{lwi}
  \regmemimmtype{lwi}
     {\lwiopc}
     {
       This instruction loads the value of \sr{base} plus the
       immediate value specified in the instruction into \dr.
     }
\end{instruction}

\begin{instruction}{mod}\label{inst:mod}
  \gregtype{mod}
     {\amodopc}
     {$\dr \becomes \sr{0} \bmod \sr{1}$}
     {
       \todo{Implement real mod, or document that it's not implemented.}
       Computes the remainder of \sr{0} divided by \sr{1} and stores
       the result in \dr.

       A \emph{divide by zero} fault is raised if \sr{1} is zero.

       For integer operands, the \xrefinst{div} and \texttt{mod}
       instruction are related as follows:

       \begin{math}
         \begin{array}{lcl}
           x \bmod y & \equiv  & x - y \lfloor {x \bdiv y} \rfloor \\
           y > 0 & \rightarrow & 0 \leq x \bmod y < y \\
           y = 0 & \rightarrow & undefined \\
           y < 0 & \rightarrow & y < x \bmod y \leq 0 \\
         \end{array}
       \end{math}
     }
\end{instruction}

\begin{instruction}{mul}
  \gregtype{mul}{\amulopc}{$\dr \becomes \sr{0} \times \sr{1}$}{Multiplies
    \sr{0} by \sr{1} and stores the result in \dr.}
\end{instruction}

\begin{instruction}{nor}
  \rtype{nor}{\rnoropc}{$\dr \becomes \neg (\sr{0} \vee
    \sr{1})$}{Performs a bitwise \texttt{nor} of two registers and
    places the result in the destination register.}

  \begin{xnote}
    A bitwise \texttt{not} can be implemented using \texttt{nor} with
    the following:

\begin{verbatim}
  nor   R1, R0, R2
\end{verbatim}

This instruction sets \texttt{R2} to the bitwise \texttt{not} of
\texttt{R1}.
  \end{xnote}

\end{instruction}


\begin{instruction}{or}
  \rtype{or}
     {\roropc}
     {$\dr \becomes \sr{0} \vee \sr{1}$}
     {
       Performs a bitwise \texttt{or} of two registers and then
       places the result in the destination register.
     }
\end{instruction}


\begin{instruction}{pop}\label{inst:pop}
  \popstktype{pop}
     {\popopc}
     {
       \begin{math}
         \begin{array}{lcl}
           \dr & \becomes & \texttt{MEMORY[\textrm{\sr{30}}]} \\
           \sr{30} & \becomes & \sr{30} + 4 \\
         \end{array}
       \end{math}
     }
     {
       Pops a four byte integer value from the stack into \dr, then
       increments \asmreg{30} by four bytes.
     }
\end{instruction}

\begin{instruction}{popf}\label{inst:popf}
  \popstktype{popf}
     {\popfopc}
     {
       \begin{math}
         \begin{array}{lcl}
           \dr & \becomes & \texttt{MEMORY[\textrm{\sr{30}}]} \\
           \sr{30} & \becomes & \sr{30} + 4 \\
         \end{array}
       \end{math}
     }
     {
       Pops a four byte single precision real value from the stack
       into \dr, then increments \asmreg{30} by four bytes.
     }
\end{instruction}

\begin{instruction}{popd}\label{inst:popd}
  \popstktype{popd}
     {\popdopc}
     {
       \begin{math}
         \begin{array}{lcl}
           {\dr}_{31..00} & \becomes & \texttt{MEMORY[\textrm{\sr{30}}]} \\
           {\dr}_{63..32} & \becomes & \texttt{MEMORY[\textrm{\sr{30}} + 4]} \\
           \sr{30}       & \becomes & \sr{30} + 8 \\
         \end{array}
       \end{math}
     }
     {
       Pops an eight byte double precision  value from the stack
       into \dr, then increments \asmreg{30} by eight bytes.
     }
\end{instruction}

\begin{instruction}{push}\label{inst:push}
  \pushstktype{push}
     {\pushopc}
     {
       \begin{math}
         \begin{array}{lcl}
           \sr{30}                           & \becomes & \sr{30} - 4 \\
           \texttt{MEMORY[\textrm{\sr{30}}]} & \becomes & {\sr{0}} \\
         \end{array}
       \end{math}
     }
     {
       Decrements \asmreg{30} by four, then pushes the four byte integer
       value in \sr{0} onto the stack.
     }
\end{instruction}

\begin{instruction}{pushf}\label{inst:pushf}
  \pushstktype{pushf}
     {\pushfopc}
     {
       \begin{math}
         \begin{array}{lcl}
           \sr{30}                           & \becomes & \sr{30} - 4 \\
           \texttt{MEMORY[\textrm{\sr{30}}]} & \becomes & {\sr{0}} \\
         \end{array}
       \end{math}
     }
     {
       Decrements \asmreg{30} by four, then pushes the four byte single
       precision real value in \sr{0} onto the stack.
     }
\end{instruction}

\begin{instruction}{pushd}\label{inst:pushd}
  \pushstktype{pushd}
     {\pushdopc}
     {
       \begin{math}
         \begin{array}{lcl}
           \sr{30}                           & \becomes & \sr{30} - 8 \\
           \texttt{MEMORY[\textrm{\sr{30}} - 4]} & \becomes & {\sr{0}}_{31..00} \\
           \texttt{MEMORY[\textrm{\sr{30}}]} & \becomes & {\sr{0}}_{63..32} \\
         \end{array}
       \end{math}
     }
     {
       Decrements \asmreg{30} by eight, then pushes the eight byte double
       precision real value in \sr{0} onto the stack.
     }
\end{instruction}


\begin{instruction}{rot}
  \rtype{rot}
     {\rrotropc}
     {
       \begin{math}
         \begin{array}{lcl}
           \sr{1} >= 0 & \implies & \begin{array}{lcl}
             \textrm{shift} & \becomes & \sr{1} \bmod 32 \\
             \textrm{lo} & \becomes & {\sr{0}} \bdiv {2^{32 - \textrm{shift}}} \\
             \textrm{hi} & \becomes & \sr{0} \times 2^{\textrm{shift}} \\
             \dr & \becomes & \textrm{hi}\vee \textrm{lo} \\
           \end{array} \\
           & & \\
           \sr{1} < 0 & \implies & \begin{array}{lcl}
             \textrm{shift} & \becomes & \abs{\sr{1}} \bmod 32 \\
             \textrm{lo} & \becomes & \sr{0} \times 2^{\textrm{shift}} \\
             \textrm{hi} & \becomes & {\sr{0}} \bdiv {2^{32 - \textrm{shift}}} \\
             \dr & \becomes & \textrm{hi}\vee \textrm{lo} \\
           \end{array}
         \end{array}
       \end{math}
     }
     {
       This considers \sr{0} as an unsigned integer and rotates it by
       the number of bits specified in \sr{1} and stores the result in
       \dr.

       When \sr{1} is positive, a left shift is used, and when \sr{1}
       is negative, a right shift is used.

       When shifting left, bits shifted out of the highest bit appear
       in the lowest bit, When shifting right, bits shifted out of the
       lowest bit appear in the highest bit.
     }
\end{instruction}


\begin{instruction}{sb}
  \sregmemtype{sb}
     {\sbopc}
     {
       \begin{math}
         \begin{array}{lcl}
           \texttt{MEMORY[\textrm{\sr{base}} + \(\textrm{\sr{index}}
               \times 2^\texttt{S}\) +  \textrm{signed offset}]}
           & \becomes & {\sr{0}}_{7..0}
         \end{array}
       \end{math}
     }
     {
       This instruction stores the low-order byte
       of \sr{0} to the address specified in the instruction.
     }
\end{instruction}


\begin{instruction}{scr}\label{inst:scr}
  \scrtype{scr}
    {
      Stores \sr{0} into \acs{CR}.  The affects of the control
      register change will take place at the beginning of the next
      instruction.
    }
\end{instruction}


\begin{instruction}{sd}
  \sregmemtype{sd}
     {\sdopc}
     {
       \begin{math}
         \begin{array}{lcl}
           \texttt{MEMORY[\textrm{\sr{base}} + \(\textrm{\sr{index}}
               \times 2^\texttt{S}\) +  \textrm{signed offset}]}
           & \becomes & {\sr{0}}_{31..00} \\
           \texttt{MEMORY[\textrm{\sr{base}} + \(\textrm{\sr{index}}
               \times 2^\texttt{S}\) +  \textrm{signed offset} + 4]}
           & \becomes & {\sr{0}}_{63..32} \\
         \end{array}
       \end{math}
     }
     {
       Stores the eight byte double precision real value in
       \sr{0} to the address specified in the instruction.
     }
\end{instruction}


\begin{instruction}{seb}
  \exttype{seb}
     {\extb}
     {
       \begin{math}
         \begin{array}{lcl}
           {\dr}_{07..00} & \becomes & {\sr{0}}_{07..00} \\
           {\dr}_{31..08} & \becomes & {\sr{0}}_{07} \\
         \end{array}
       \end{math}
     }
     {
       Sign-extends the low-order byte in \sr{0} to a word
       and assigns the result to \dr.  Bits eight through 31
       will each contain a copy of bit seven from \sr{0}.
     }
\end{instruction}


\begin{instruction}{seh}
  \exttype{seh}
     {\exth}
     {
       \begin{math}
         \begin{array}{lcl}
           {\dr}_{15..00} & \becomes & {\sr{0}}_{15..00} \\
           {\dr}_{31..16} & \becomes & {\sr{0}}_{15} \\
         \end{array}
       \end{math}
     }
     {
       Sign-extends the low-order half-word in \sr{0} to a word and
       assigns the result to \dr.  Bits 16 through 31 will each
       contain a copy of bit 15 from \sr{0}.
     }
\end{instruction}


\begin{instruction}{seq}
  \ctype{seq}
     {\seqopc}
     {
       \begin{math}
         \begin{array}{lcl}
           \dr & \becomes & \sr{0}\texttt{.ZF} = 1 \\
         \end{array}
       \end{math}
     }
\end{instruction}


\begin{instruction}{sf}
  \sregmemtype{sf}
     {\sfopc}
     {
       \begin{math}
         \begin{array}{lcl}
           \texttt{MEMORY[\textrm{\sr{base}} + \(\textrm{\sr{index}}
               \times 2^\texttt{S}\) +  \textrm{signed offset}]}
           & \becomes & {\sr{0}}
         \end{array}
       \end{math}
     }
     {
       Stores the four byte single precision real value
       in \sr{0} into the address specified in the instruction.
     }
\end{instruction}


\begin{instruction}{sge}
  \ctype{sge}
     {\sgeopc}
     {
       \begin{math}
         \begin{array}{lcl}
           \dr & \becomes & \sr{0}.SF = \sr{0}.OF \\
         \end{array}
       \end{math}
     }
\end{instruction}


\begin{instruction}{sgeu}
  \ctype{sgeu}
     {\sgeuopc}
     {
       \begin{math}
         \begin{array}{lcl}
           \dr & \becomes & \sr{0}.CF = 0 \\
         \end{array}
       \end{math}
     }
\end{instruction}


\begin{instruction}{sgt}
  \ctype{sgt}
     {\sgtopc}
     {
       \begin{math}
         \begin{array}{lcl}
           \dr & \becomes & \sr{0}.ZF = 0 \wedge \sr{0}.SF = \sr{0}.OF \\
         \end{array}
       \end{math}
     }
\end{instruction}


\begin{instruction}{sgtu}
  \ctype{sgtu}
     {\sgtuopc}
     {
       \begin{math}
         \begin{array}{lcl}
           \dr & \becomes & \sr{0}.CF = 0 \wedge \sr{0}.ZF = 0 \\
         \end{array}
       \end{math}
     }
\end{instruction}


\begin{instruction}{sh}
  \sregmemtype{sh}
     {\shopc}
     {
       \begin{math}
         \begin{array}{lcl}
           \texttt{MEMORY[\textrm{\sr{base}} + \(\textrm{\sr{index}}
               \times 2^\texttt{S}\) +  \textrm{signed offset}]}
           & \becomes & {\sr{0}}_{15..00}
         \end{array}
       \end{math}
     }
     {
       Stores the low-order 16 bits of \sr{0} to the address
       specified in the instruction.
     }
\end{instruction}


\begin{instruction}{sin}
  \fprtype{sin}
     {\sinopc}
     {
       \begin{math}
         \begin{array}{lcl}
           \dr & \becomes & \sin(\sr{0})
         \end{array}
       \end{math}
     }
     {Takes the sine of the value in \sr{0}.  The domain of the value
     in \texttt{\sr{0}}, as well as the result, conform to
     the \texttt{ANSI C} \texttt{sin} library function.

     \sr{0} and \dr are both floating point registers.}
\end{instruction}


\begin{instruction}{sle}
  \ctype{sle}
     {\sleopc}
     {
       \begin{math}
         \begin{array}{lcl}
           \dr & \becomes & \sr{0}.ZF = 1 \vee \sr{0}.SF \neq \sr{0}.OF \\
         \end{array}
       \end{math}
     }
\end{instruction}


\begin{instruction}{sleu}
  \ctype{sleu}
     {\sleuopc}
     {
       \begin{math}
         \begin{array}{lcl}
           \dr & \becomes & \sr{0}.CF = 1 \vee \sr{0}.ZF = 1 \\
         \end{array}
       \end{math}
     }
\end{instruction}


\begin{instruction}{slt}
  \ctype{slt}
     {\sltopc}
     {
       \begin{math}
         \begin{array}{lcl}
           \dr & \becomes & \sr{0}.SF \neq \sr{0}.OF \\
         \end{array}
       \end{math}
     }
\end{instruction}

\begin{instruction}{sltu}
  \ctype{sltu}
     {\sltuopc}
     {
       \begin{math}
         \begin{array}{lcl}
           \dr & \becomes & \sr{0}.CF = 1 \\
         \end{array}
       \end{math}
     }
\end{instruction}

\begin{instruction}{sne}
  \ctype{sne}
     {\sneopc}
     {
       \begin{math}
         \begin{array}{lcl}
           \dr & \becomes & \sr{0}.ZF = 0 \\
         \end{array}
       \end{math}
     }
\end{instruction}


\begin{instruction}{sqrt}
  \fprtype{sqrt}
     {\sqrtopc}
     {
       \begin{math}
         \begin{array}{lcl}
           \dr & \becomes & \sqrt{\sr{0}}
         \end{array}
       \end{math}
     }
     {Takes the square root of the value in \sr{0}.  The domain of the value
     in \texttt{\sr{0}}, as well as the result, conform to
     the \texttt{ANSI C} \texttt{sqrt} library function.

     \sr{0} and \dr are both floating point registers.}
\end{instruction}


\begin{instruction}{sub}
  \gregtype{sub}{\asubopc}{$\dr \becomes \sr{0} - \sr{1}$}{Subtracts
    \sr{1} from \sr{0} and stores the result in \dr.}
\end{instruction}


\begin{instruction}{sw}
  \sregmemtype{sw}
     {\swopc}
     {
       \begin{math}
         \begin{array}{lcl}
           \texttt{MEMORY[\textrm{\sr{base}} + \(\textrm{\sr{index}}
               \times 2^\texttt{S}\) +  \textrm{signed offset}]}
           & \becomes & {\sr{0}}
         \end{array}
       \end{math}
     }
     {
       Stores the word in \sr{0} to the address specified in the instruction.
     }
\end{instruction}


\begin{instruction}{tan}
  \fprtype{tan} {\tanopc} { \begin{math} \begin{array}{lcl} \dr
     & \becomes & \tan(\sr{0}) \end{array} \end{math} } {This
     instruction produces the tangent of \texttt{\sr{0}}.  The value
     in \sr{0} is given in radians.  The domain of the value
     in \texttt{\sr{0}}, as well as the result, conform to
     the \texttt{ANSI C} \texttt{tan} library function.

     \sr{0} and \dr are both floating point registers.}
\end{instruction}


\begin{instruction}{traparray}\label{inst:traparray}
\systraptype{traparray}
     {13}
     {0}
     { % semantics
       \begin{math}
         \begin{array}{lcl}
           \neg (0 <= \sr{0} \wedge \sr{0} < \sr{1}) & \implies &
           \left\lbrace
           \begin{array}{lcl}
             \asmreg{1} & = & 13 \\
             \asmreg{31} & = & \pc \\
             \pc & = & \creg{5} \\
           \end{array}
           \right. \\

           (0 <= \sr{0} \wedge \sr{0} < \sr{1}) & \implies & \pc = \pc + 4 \\
         \end{array}
       \end{math}
     }
     {
       Tests that the value in \sr{0} is non-negative, and less than
       or equal to the array bound in \sr{1}.  If the value is within
       the specified range, execution continues with the next
       instruction.  If not within the specified range, software trap
       13 is raised (\xref{sect:CR5}).
     }
\end{instruction}


\begin{instruction}{trapnil}\label{inst:trapnil}
\niltraptype{trapnil}
     {C}
     {0}
     { % semantics
       \begin{math}
         \begin{array}{lcl}
           C & \in & \lbrace 19, 20 \rbrace \\
           \sr{0} = \textrm{NIL} & \implies &
           \left\lbrace
           \begin{array}{lcl}
             \asmreg{1} & = & C \\
             \asmreg{31} & = & \pc \\
             \pc & = & \creg{5} \\
           \end{array}
           \right. \\
           \sr{0} \neq \textrm{NIL} & \implies & \pc = \pc + 4 \\
         \end{array}
       \end{math}
     }
     {
       Checks that the pointer value in \sr{0} is not \texttt{NIL}.
       If it is \texttt{NIL}, software trap 20 (\xref{sect:CR5}) is
       raised.  If it is not \texttt{NIL}, execution continues at the
       next instruction.
     }
     % 20 must match SKLKernel.Mod HaltPointerCheck.
\end{instruction}


\begin{instruction}{traprange}\label{inst:traprange}
\rangetraptype{traprange}
     {12}
     {\textrm{C}}
     {
       \begin{math}
         \begin{array}{lcl}
           C = 1 & \implies &
           \begin{array}{lcl}
             (\sr{0} \in \lbrace 0, 1 \rbrace) & \implies & \pc = \pc + 4 \\
             (\sr{0} \notin \lbrace 0, 1 \rbrace) & \implies &
             \left\lbrace
             \begin{array}{lcl}
               \asmreg{1} & = & 12 \\
               \asmreg{31} & = & \pc \\
               \pc & = & \creg{5} \\
             \end{array}
             \right. \\
           \end{array} \\

           & & \\

           C = 4 & \implies &
           \begin{array}{lcl}
             (-2^7 + 1 <= \sr{0} <= 2^7) & \implies & \pc = \pc + 4 \\
             \neg (-2^7 + 1 <= \sr{0} <= 2^7) & \implies &
             \left\lbrace
             \begin{array}{lcl}
               \asmreg{1} & = & 12 \\
               \asmreg{31} & = & \pc \\
               \pc & = & \creg{5} \\
             \end{array}
             \right. \\
           \end{array} \\

           & & \\

           C = 5 & \implies &
           \begin{array}{lcl}
             (-2^{15} + 1 <= \sr{0} <= 2^{15}) & \implies & \pc = \pc + 4 \\
             \neg (-2^{15} + 1 <= \sr{0} <= 2^{15}) & \implies &
             \left\lbrace
             \begin{array}{lcl}
               \asmreg{1} & = & 12 \\
               \asmreg{31} & = & \pc \\
               \pc & = & \creg{5} \\
             \end{array}
             \right. \\
           \end{array} \\

           & & \\

           C = 6 & \implies & \pc = \pc + 4 \\

           & & \\

           C = 9 & \implies &
           \begin{array}{lcl}
             (\sr{0} \in \lbrace 0, 31 \rbrace) & \implies & \pc = \pc + 4 \\
             (\sr{0} \notin \lbrace 0, 31 \rbrace) & \implies &
             \left\lbrace
             \begin{array}{lcl}
               \asmreg{1} & = & 12 \\
               \asmreg{31} & = & \pc \\
               \pc & = & \creg{5} \\
             \end{array}
             \right. \\
           \end{array} \\

         \end{array}
       \end{math}
     }
     {
       Checks that the value in \sr{0} is within the allowable range
       of the type specified by \textrm{C}.  If the value is out of
       the range, software trap 12 (\xref{sect:CR5}) is raised.  If
       the value is within the allowed range, execution continues with
       the next instruction.

       \Figref{traprange-type} shows the Oberon type encoded by
       \emph{C}.

       \begin{figure}[H]
         \centering
           \begin{tabular}{l|l}
             1 & \texttt{BOOLEAN} \\
             4 & \texttt{SHORTINT} \\
             5 & \texttt{INTEGER} \\
             6 & \texttt{LONGINT} \\
             9 & \texttt{SET} \\
           \end{tabular}
         \caption{\texttt{traprange} type encoding}\label{fig:traprange-type}
       \end{figure}
     }
\end{instruction}

\begin{instruction}{vmsvc}\label{inst:vmsvc}
\mtype{vmsvc}
     {\mvmsvcopc}
     {\sr{0}}
     {0}
     {Host OS services executed}
     {
       This instruction provides an escape from the virtual machine
       that facilitates interaction with the host OS.  The input
       register \sr{0} must hold the address of a descriptor in Oberon
       memory that describes the service that should be executed.  The
       structure of the descriptor is defined by the \texttt{Kernel}
       module, and is shown in \figref{vmsvc-descriptor}.

       \begin{figure}[h]
         \centering
           \begin{bytefield}{32}
             \bitheader[endianness=big]{0-31}          \\
             \bitbox{32}{service id}
           \end{bytefield}
         \caption{Base \texttt{vmsvc} descriptor}\label{fig:vmsvc-descriptor}
       \end{figure}

       The \emph{service id} field indicates the service that is to be
       provided.

       See \xref{chap:vmsvc} for support VM Services.
     }
     {\sr{0}}
\end{instruction}


\begin{instruction}{xor}
  \rtype{xor}{\rxoropc}{$\dr \becomes \sr{0} \xor \sr{1}$}
  {Performs a bitwise \emph{exclusive or} of \sr{0} and \sr{1} and
   stores the result in \dr.}
\end{instruction}
