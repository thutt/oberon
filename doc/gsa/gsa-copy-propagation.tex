% Copyright (c) 2001-2022 Logic Magicians Software
\chapter{Copy Propagation}

\begin{quote}
  \ac{cp} replaces all uses of \gsainst{copy} by uses of the operand
  of the \gsainst{copy}.  This reduces the unnecessary movement of
  data from one location to another.

  See \refgsainst{copy}{Gmisc}.
\end{quote}

The translation phase from the \ac{ast} to the \ac{gsa} \ac{ir}, the
translator will introduce a \gsainst{copy} when a constant or variable
is assigned to another variable.  Since \ac{gsa} has a \emph{single
  assignment} property, the left-hand-side of a \gsainst{copy} cannot
be directly overwritten, so all uses of this \gsainst{copy} can be
replaced with direct use of the operand of the instruction.

The \ac{cp} algorithm operates by examining a \emph{global region} for
\gsainst{copy} instructions.  When found it replaces these
instructions with the source of the copy.

\begin{algorithm}[h!]
  \caption{Copy Propagation}
  \begin{algorithmic}[1]
    \REQUIRE{greg $\neq$ nil}

    \ENSURE{\FORALL{inst: inst $\in$ greg}
      \STATE{inst $\neq$ \gsainst{copy}}
      \ENDFOR}
    \STATE
    \FORALL{region: region $\in$ greg}
    \FORALL{inst: inst $\in$ region}
    \IF{inst $=$ \gsainst{copy}}
    \STATE replace-use-of inst with inst.operand
    \ENDIF
    \ENDFOR
    \ENDFOR
  \end{algorithmic}
\end{algorithm}

Since \gsainst{copy} can only be introduced during the translation to
\ac{gsa}, once all have been removed, the \ac{cp} algorithm never has
to be run again.
