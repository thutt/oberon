% Copyright (c) 2001-2023 Logic Magicians Software
\chapter{Type Hierarchy}

The implementation of \ac{gsa} for in this compiler uses type
extension to enable an easy-to-use and simple-to-understand framework
of data structures.

\begin{figure}[h!]
% 2000.07.20: There is a metapost equivalent of this diagram (which
% looks  much better), but a defect in the NEC SuperScript 1800N
% printer - or in metapost - causes the metapost diagram to hang the
% printer.  So, instead this xypic version will have to be used for
% now.
\begin{tiny}
  \xymatrix@R=23pt@C=5pt{
  *+[F]{ConstSet} & *+[F]{ConstIntRange} & *+[F]{ConstInt} &
  *+[F]{ConstBool} & *+[F]{ConstReal} & *+[F]{ConstLReal} &
  *+[F]{ConstString} \\
  *+[F]{Type} & *+[F]{Temporary} & *+[F]{Const} \ar[ull]\ar[ul]\ar[u]\ar[ur]\ar[urr]\ar[urrr]\ar[urrrr]   &         & *+[F]{CondMerge} \\
  & &                   & *+[F]{Guard}       & *+[F]{Merge}\ar[u]      &
  *+[F]{GlobalRegion} & *+[F]{Case}\\
  & *+[F]{Symbol}\ar[uu] &                    &                    & *+[F]{Region}\ar[ul]\ar[u]\ar[ur]\ar[urr]    \\
  & &                    & *+[F]{Gate}     \\
  & &                    & *+[F]{Instruction}\ar[u]\ar[uur]                   &  &  \\
  & *+[F]{Addressable}\ar[uuuuul]\ar[uuu]\ar[uuuuur] & & *+[F]{Result}\ar[u]                       &               &  \\
  & & *+[F]{Usable}\ar[ul]\ar[ur]      &                    & *+[F]{Operand}
  & *+[F]{SymLocation} \\
  &                    &                    & *+[F]{Node}\ar[ul]\ar[ur]              &              & *+[F]{Location}\ar[u] \\
  &                    &                    &                   & *+[F]{Info}\ar[ul]\ar[ur]
}
\end{tiny}
\caption{\ac{gsa} Type Hierarchy}\label{fig:typehierarchy}
\end{figure}

Figure~\ref{fig:typehierarchy} shows this, and the following sections
will touch briefly on how each of these sturctures functions in a
\ac{gsa}-based compiler.

\section{GSA Type Hierarchy}
\subsection{Info}

As the root of the entire type tree, the \code{Info} record ties
everything together and makes the system work seamlessly.  This type
allows any node in a graph to reference any other node in a graph; it
is quite beneficial to be able to do this for certain \ac{cia}.

For example, it becomes possible for a \code{Symbol} node to reference
a \code{Result} node to designate the symbol's \emph{current value}.

\subsection{Location}

The \code{Location} record is used to store information about the
\emph{location} of a definition of a node in the \ac{gsa} graph.

\subsection{SymLocation}

The \code{SymLocation} record is used to store information the
\emph{location} of a definition of a Symbol in the \ac{gsa} graph.

\subsection{Node}

The \code{Node} type is the root of items which can be placed, at
various positions, into a \ac{gsa} graph.  Each node has a symbol
table type and an \code{Info} field.  The \code{Info} field is to be
considered volatile over different \ac{cia} executions; in other
words, this field is available for use by a \ac{cia}.

\subsection{Operand}\label{th:operand}

An \code{Operand} type defines an \emph{operand} of an
\code{Instruction}.  Operands have the following attributes:

\begin{itemize}
\item Operands are joined to an instruction via a doubly-linked list of
  operands, with a dummy head.
\item Operands have a doubly linked list of \emph{uses}.  This
  implements a \emph{use chain}.

\item Operands refer back to the instruction that contains them.

\item Operands contain a field which indicates the definition of the
  operand's value.
\end{itemize}

\subsection{Usable}

Type \code{Usable} is the root of a tree of records which can be
\emph{used} by other records as operands or results.

\subsection{Addressable}

The \code{Addressable} tree describes things which can be
\emph{addressed} by the generated software.

The \code{Addressable} hierarchy introduces some complications for the
\acp{cia} since the hierarchy extended from \code{Addressable} is not
of the same lineage as those extended from \code{Result}; this would
naively result in a requirement of many special cases to distinguish
different nodes in the \ac{ir}.

This is important because the improvement algorithms work from the
assumption that an \code{Instruction} produces a \code{Result}, and
these \code{Instruction}/\code{Result} pairs are used to establish
\emph{code scheduling}, \emph{code live-ness}, and many other
details of \acp{cia}.

Consequently, the following rules must be applied uniformly when
working with \code{Addressable} items.

\begin{enumerate}
\item When used as an lvalue, the Addressable object (\code{Symbol},
  \code{Temporary}, \code{Const}) will be directly placed into the
  \code{Result.lhs} field.

\item When used as an rvalue, an \code{Addressable} must be rendered
  into a \code{Result}.  This will happen naturally through the use of
  the \code{Addressable.currvalue} field in all cases except the first
  rvalue use of an \code{Addressable} prior to its being set (this
  should only occur on uninitialized local variables (module variables
  are modeled through the \gsainst{enter} result
  list).\footnote{Maybe all local variables should be modeled on the
    \gsainst{enter} result list to prevent this breakdown?}
\end{enumerate}

\subsection{Type}

The \code{Type} record is used to model \emph{source language
  type-related} type symbols.

For example, the \ac{td} associated with a user-defined type would be
assigned a \code{Type} node.

\subsection{Symbol}
The \code{Symbol} record is used to model \emph{source language}
symbols.

For example, a user-defined procedure would be assigned a
\code{Symbol} node.

\subsection{Temporary}
The \code{Temporary} record is used to model \emph{compiler-generated}
temporaries.

\subsection{Const}

The \code{Const} type roots the class of constants used by the
compiler.  Each specific type of constant which can be used in the
\ac{be} will be represented by a type extended from \code{Const}.

\subsection{ConstSet}

\code{ConstSet} represents a set constant of the source langauge.

\subsection{ConstIntRange}

\code{ConstIntRange} represents a
\subsection{ConstInt}
\code{ConstInt} represents an integer constant of the source langauge.
The \code{type} field provides the information pertaining to size and
range of the constant value.
\subsection{ConstBool}
\code{ConstBool} represents a boolean constant of the source language.
\subsection{ConstReal}
\code{ConstReal} represents a real constant of the source language.
\subsection{ConstLReal}
\code{ConstLReal} represents a long real constant of the source language.
\subsection{ConstString}
\code{ConstString} represents a string literal of the source language.

\subsection{Result}
The \code{Result} record represents the \emph{result} of some
operation.  Normally, this takes the form of the \emph{result of an
  instruction}, but every \code{Result} is not an \code{Instruction}.

Each \code{Record} has the following properties:

\begin{itemize}
\item Each is maintained on a doubly-linked list rooted in the
  \code{Instruction} record.

\item A reference to the \code{Instruction} which produced the result.

\item A set of attributes which can be used by \acp{cia}.
\end{itemize}

\subsection{Instruction}

An \code{Instruction} record represents an \ac{ir} \emph{instruction}.
Each \code{Instruction} has the following attributes:

\begin{itemize}
\item Each is kept on a doubly-linked list.
\item An synthesizes one or more \code{Result}
  values\footnote{Consider a \emph{divide} instruction which produces
    both the \emph{quotient} and \emph{remainder}}.  The
  \code{Instruction} itself is considered a \code{Result}.
\item Each can be placed on a second doubly-linked list; this is
  intended for use by \acp{cia}.
\item Each can have zero or more operands on a doubly-linked list with
  a dummy head.
\item Each has a reference to the \code{Region} which contains it.
  \item Each can be numbered using a \emph{value numbering} scheme.
\end{itemize}

\subsection{Gate}
The \code{Gate} record implements the so-called $\phi$ nodes of
\ac{gsa}.  These nodes join values which are set through different
control paths in the program.

Each \code{Gate}, which is a specialized \code{Instruction}, has the
following attributes in addition to those of an \code{Instruction}:

\begin{itemize}
\item The \code{Addressable} item which is controlled by the gate.
\item The original value of the \code{Addressable} prior to the split
  of the control path; this allows the compiler to reset the value so
  that all control paths can be processed under the same starting conditions.
\end{itemize}

\subsection{Region}

A \code{Region} record is used to hold a sequence of instructions.
For example, the \code{THEN} and \code{ELSE} sections of an \code{IF}
statement would each be implemented as a \code{Region}.

Since a \code{Region} is also an \code{Instruction}, it has all the
attributes of an \code{Instruction}.  It also has the following
attributes:

Each \code{Region} references the \code{Region} in which it is
contained, but this relationship is not established until the region
is actually appended to the instruction stream.

\begin{itemize}
\item Each is held on a doubly-linked list.
\item Each has a reference to the \code{Instruction} items it
  contains.
\item Each has a reference to its \code{Merge} region.
\end{itemize}

\subsection{Merge}

A \code{Merge} region is used to bring split control flow back into a
single control path.  This is a base type for handling specific types
of merge control regions.

\subsection{Guard}

A \code{Guard} region is a specialization of the \code{Region} type
which controls access to a particular code path.  \code{Instruction}
items which are contained in the \code{Guard} region are executed
based on the truth value of the first operand.

\subsection{CondMerge}

The \code{CondMerge} is not currently used.

\subsection{GlobalRegion}
The \code{GlobalRegion} type is used to contain the instructions
generated for a source-level procedure.  It is essentially a
\code{Region} with the following additional attributes:

\begin{itemize}
\item The symbol table entry for the procedure.
\item A reference to \gsainst{enter} for the procdure.
\item A reference to \gsainst{exit} for the procedure.
\item A list of \code{Const} items used by the procedure.
\item A list of \code{Addressable} items used by the procedure.
\end{itemize}

\subsection{Case}
The \code{Case} type is used to model the source-level \code{CASE}
statement.  It contains zero or more regions which process the
individually coded sections of the \code{CASE} statement.vm
