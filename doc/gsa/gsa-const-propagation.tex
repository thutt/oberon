% Copyright (c) 2001-2022 Logic Magicians Software
\chapter{Constant Propagation}
\todo{Document which algorithm is implemented and give a bib ref}

Frequently a program will calculate the same value in different
sections of code, and while these extra expressions may not have been
intentional, recomputing them reduces the efficiency of the generated
code.  However, if the computed expressions yield constant values, the
constant, rather than the result, can be propagated to all uses of the
expression.  Very often this process of \ac{ccp} will cause further
computations to become constant, which will iteratively improve the
quality of the code produced by the compiler.

When the condition controlling a region evaluates to a constant value,
it can be determined at compile time if the region will \emph{always
  execute} or \emph{never execute}.  These properties can be derived
from the output of the algorithm described in this chapter.

Surprisingly, many of the opportunities for \ac{ccp} are produced by
the compiler as by-products of other transformation and improvement
algorithms, and it is the reason that this improvement algorithm is
included.

This chapter describes an algorithm which propagates constants and
identifies code which is not reached through any control path through
the region.

\section{Basic Algorithm}

\ac{ccp} iteratively examines the operands of each instruction in a
region, and when all the operands of an instruction can be proven to
evaluate to a constant value, the instruction will be marked as
\emph{constant}.  All uses of the newly produced result will be
propagated throughout the region.  This process is iterated until a
fix point is reached.

\subsection{Instruction State}\label{ccp:classification}

To easily determine the status of an instruction, it will be
classified into one of the basic categories shown in
\tableref{table:const-prop-class}.

\begin{table}[h!]
  \begin{tabularx}{\linewidth}{|l|X|}
    \hline State & Description \\

    \hline unclassified & This annotation indicates that nothing has
    yet been determined about the \emph{constness} of the instruction
    or its operands.  \\

    \hline constant & This annotation indicates that an instruction
    will always produce a constant value.  Each unique constant value
    will have a unique lattice node which contains the computed
    value. \\

    \hline non-constant & This annotation indicates that an
    instruction will never produce a constant value. \\

    \hline unreachable & This annotation indicates that the
    instruction is unreachable through any control path of the
    program.  It will not be set for \emph{regions}; they will instead
    be marked as \emph{false}. \\
    \hline
  \end{tabularx}
  \caption{Instruction Categories}\label{table:const-prop-class}
\end{table}

\begin{observation}
  These categories can be considered hierarchical: \emph{unclassified}
  is the highest category and \emph{unreachable} is the lowest.  In
  this implementation an instruction can only be reclassified with an
  equal or lower hierarachy classification.  For example, a value
  classified as \emph{constant} will never be reclassified as
  \emph{unclassified}, but it could be reclassified as
  \emph{unreachable}.  If this invariant is not satisfied, it cannot
  be shown that the algorithm will reach a fix point.
\end{observation}

The basic categories will be more fully described when applied to
specific instruction types.

\section{Initialization}

Initialization of the global region for this algorithm is fairly
simple, consisting of setting the \emph{backend data field} of the
\ac{ir} to an initial value.  Initializing the constant values used in
the global region is simplified since these constants are stored in a
list accessible through the global region; without this list, each
operand of all instructions would have to be examined and initialized
if constant.

\begin{algorithm}[h!]
  \label{algo:ccp-initialization}
  \caption{CCP (initialization)}
  \begin{algorithmic}[1]
    \REQUIRE{greg $\neq$ nil}

    \STATE{greg $\leftarrow$ \emph{non-constant}}
    \FORALL{i: i $\in$ \{result of \gsainst{enter}, access of data structure\}}
      \STATE i $\leftarrow$ \emph{non-constant}
    \ENDFOR

    \FORALL{i: i $\in$ \{constant literal used in greg\}}
      \STATE i $\leftarrow$ \emph{constant}
    \ENDFOR

    \FORALL{i: i $\in$ \{instructions not initialized already\})}
      \STATE i $\leftarrow$ \emph{unclassified}
    \ENDFOR

    \STATE{worklist $\leftarrow$ instructions in greg}
  \end{algorithmic}
\end{algorithm}

\section{Propagation}
Following initialization, the core of the algorithm is executed
iteratively until a fixpoint is reached.  \Algoref{ccp-propagation}
shows the basic steps for the core of \ac{ccp}, while further
subsections will show greater detail.

\begin{algorithm}[h!]
  \label{algo:ccp-propagation}
  \caption{CCP (core)}
  \begin{algorithmic}[1]
    \REQUIRE{greg $\neq$ nil}

    \WHILE{worklist $\neq$ nil}
      \STATE{inst = worklist head}
      \STATE{lattice = NewLattice(inst)}

      \COMMENT{inv: lattice.classification $\leq$ inst.classification}
      \IF{lattice.classification $<$ inst.classification}
      \STATE{inst.classifcation $\leftarrow$ lattice.classification}
      \COMMENT{instruction is now \emph{more} classified}
      \IF{(instruction is Region) and (lattice.classification $=$
      false)}
        \STATE{mark code in region as unreachable}
        \COMMENT{Region is never executed}
      \ENDIF
      \ELSIF{lattice.classficiation = unclassified}
      \STATE{}
      \COMMENT{instruction has not yet been classified; put it back on
      the worklist}
      \ENDIF
    \ENDWHILE
    \STATE{}
    \COMMENT{All instructions have been classified.  Now replace all
      constant values}
    \STATE{ReplaceWithConstants(greg);}
  \end{algorithmic}
\end{algorithm}

Since the transition from state to state can only progress in one
direction (\S\ref{ccp:classification}), iterating until a fix point is
reached is trivial.  The difficult portion of the algorithm,
classifying an \ac{ir} node, is abstracted through
\texttt{NewLattice}.

Once the fixpoint has been reached, instructions which yield a
constant value can be replaced with the actual value of the constant,
and that is abstracted by \texttt{ReplaceWithConstants}.

%-----------------------------------------------------------------------------
\section{Classfication}

The most complex portion of the \ac{ccp} algorithm is the
classification of each instruction in a region.  Since the task
encompasses all types of instructions which can be present in a
region, it will be approached with a \emph{divide \& conquer}
technique.  To that end, the basic classification algorithm will be
presented, and then the algorithm will be tailored to each type of
instruction which can be present.

\begin{algorithm}[h!]
  \label{algo:ccp-classification}
  \caption{CCP (classficiation)}
  \begin{algorithmic}[1]
    \REQUIRE{inst $\neq$ nil}
    \IF{inst.classification $=$ unreachable}
    \STATE{result is unreachable}
    \ELSIF{inst.classification $=$ non-constant}
    \STATE{result is non-constant}
    \ELSIF{inst is Gate}
    \STATE{result is classification-of-gate (\S\ref{const-prop:classification-of-gate})}
    \ELSIF{inst is Merge}
    \STATE{result is classification-of-merge (\S\ref{const-prop:classification-of-merge})}
    \ELSIF{inst is Guard}
    \STATE{result is classification-of-guard (\S\ref{const-prop:classification-of-guard})}
    \ELSIF{inst is Case}
    \STATE{result is classification-of-case (\S\ref{const-prop:classification-of-case})}
    \ELSE
    \STATE{}
    \COMMENT{This is a normal instruction, with one (1) or, at most,
    two (2) operands}
    \STATE{o0 $\leftarrow$ zeroth operand}
    \STATE{o1 $\leftarrow$ first operand}
    \STATE{}
    \COMMENT{inv: o0 $\neq$ nil}
    \COMMENT{inv: o1 $=$ nil or defined(o1)}
    \IF{o0.classification $=$ constant and o1.classification $\in$ \{nil,constant\}}
      \STATE{result $\leftarrow$ calculate-constant}
    \ELSIF{o0.classfication $=$ unreachable or o1.classfication $=$
    unreachable}
      \STATE{result $\leftarrow$ unreachable}
    \ELSIF{o0.classfication $=$ non-constant or o1.classfication $=$
    non-constant}
      \STATE{result $\leftarrow$ non-constant}
    \ELSE
      \STATE{result $\leftarrow$ unclassified}
    \ENDIF
    \ENDIF
    \STATE{result = non-constant}
  \end{algorithmic}
\end{algorithm}

Processing a normal instruction involves looking at each of its
operands, and when all are constant the constant value can be
calcualted by interpretting the instruction opcode.  Fortunately this
compiler has at most two (2) operands for all normal instructions, so
this process is fairly easy.

One bit of caution should be heeded before blindly folding all
instructions with constant operands: floating point instructions are
susecptible to the settings of the \ac{fpu} at runtime.  Consequently,
any interpretation of floating point instructions must produce the
same result as if the code to perform the calculation was produced by
the compiler.  Further caution is warranted when interpretting
instructions which are produced from \texttt{SYSTEM} module procedures
or functions.

%-----------------------------------------------------------------------------
\subsection{Guard}\label{const-prop:classification-of-guard}

The value determined for \refgsaclass{guards}{guard} is directly
related to the value calculated for the conditional which controls the
guard.  The controlling expression can evaluate to one of its
operands, or \emph{non-constant}.  Likewise, a guard will select from
one of its operands or become \emph{non-constant}\footnote{It is
  possible for a guard to also obtain the value \emph{unreachable}
  when it is contained in a reqion that is marekd as
  \emph{unrechable}.}

At the completion of this algorithm, all guards will have evaluated to
one of three lattice values: \emph{true}, \emph{false} and
\emph{non-constant}; these values directly correspond to
\emph{execute}, \emph{do not execute} and \emph{may execute}.

A guard which evaluates to \emph{false} will have all the instructions
in the guarded region set to \emph{unreachable}; such a region can be
deleted from the \ac{ir}.

\subsubsection{\gsainst{true}}

To evaluate the lattice value of \refgsainst{true}{Gguard}, the final
lattice value of the controlling condition must already have been
calculated.

\begin{algorithm}[h!]
  \label{algo:ccp-true-classification}
  \caption{\gsainst{true} (classficiation)}
  \begin{algorithmic}[1]
    \REQUIRE{inst $\neq$ nil $\wedge$ inst is \gsainst{true}}
    \STATE{op $\leftarrow$ zero-th operand}
    \IF{op.state $=$ constant}
    \STATE{}
    \COMMENT{op is true or false}
    \STATE{inst.state $\leftarrow$ op.state}
    \ELSIF{op.state $=$ operand}
    \STATE{}
    \COMMENT{inv: op is \gsainst{gate}}
    \STATE{}
    \COMMENT{inv: gate evaluates to \emph{operand-0} or \emph{operand-1}}
    \STATE{gate $\leftarrow$ definition of op}
    \STATE{op $\leftarrow$ \emph{operand-0} or \emph{operand-1} of gate}
    \STATE{inst.state $\leftarrow$ op.state}
    \ELSE
    \STATE{inst.state $\leftarrow$ non-constant}
    \ENDIF
  \end{algorithmic}
\end{algorithm}

\begin{observation}\label{obs:ccp-true-guard}
  The constness of \gsainst{true} is directly tied to the condition
  which controls the region.
\end{observation}

As shown in algorithm \ref{algo:ccp-true-classification},
\gsainst{true} is directly dependent upon the conditional expression
which dominates it.

Interestingly, if the conditional has not yet been fully classified,
the \gsainst{true} will evaluate to \emph{non-constant}.  This
actually turns out to be the correct action to take, as the dominating
conditional expression has only a limited set of lattice states it can
obtain, and the important ones are already covered in this algorithm;
only \emph{unclassified} and \emph{unreachable} are not handled.

A dominating conditional which is unreachable implies that the
\gsainst{true} is also unreachable, and so \emph{non-constant} is a
suitable value because the instruction will be deleted.  If the
dominating conditional is \emph{unclassified}, then all \ac{ir} nodes
which rely on the result will be scheduled for reclassification; once
the conditional is no longer \emph{unclassified}, it will be
appropriately handled by the algorithm.

\subsubsection{\gsainst{false}}
To evaluate the lattice value of \refgsainst{false}{Gguard}, the final
lattice value of the controlling condition must already have been
calculated.

\begin{algorithm}[h!]
  \label{algo:ccp-false-classification}
  \caption{\gsainst{false} (classficiation)}
  \begin{algorithmic}[1]
    \REQUIRE{inst $\neq$ nil $\wedge$ inst is \gsainst{false}}
    \STATE{op $\leftarrow$ zero-th operand}
    \IF{op.state $=$ constant}
    \STATE{}
    \COMMENT{op is true or false}
    \STATE{inst.state $\leftarrow$ $\neg$ op.state}
    \ELSIF{op.state $=$ operand}
    \STATE{}
    \COMMENT{inv: op is \gsainst{gate}}
    \STATE{}
    \COMMENT{inv: gate evaluates to \emph{operand-0} or \emph{operand-1}}
    \STATE{gate $\leftarrow$ definition of op}
    \STATE{op $\leftarrow$ \emph{operand-0} or \emph{operand-1} of gate}
    \STATE{inst.state $\leftarrow$ $\neg$ op.state}
    \ELSE
    \STATE{inst.state $\leftarrow$ non-constant}
    \ENDIF
  \end{algorithmic}
\end{algorithm}

\begin{observation}\label{obs:ccp-false-guard}
  The constness of \gsainst{false} is directly tied to the condition
  which controls the region.
\end{observation}

As shown in algorithm \ref{algo:ccp-false-classification},
\gsainst{false} is directly dependent upon the conditional expression
which dominates it.

Interestingly, if the conditional has not yet been fully classified,
the \gsainst{false} will evaluate to \emph{non-constant}.  This
actually turns out to be the correct action, as the dominating
conditional expression has only a limited set of lattice states it can
obtain, and the important ones are already covered in this algorithm;
only \emph{unclassified} and \emph{unreachable} are not handled.

A dominating conditional which is unreachable implies that the
\gsainst{false} is also unreachable, and so \emph{non-constant} is a
suitable value because the instruction will be deleted.  If the
dominating conditional is \emph{unclassified}, then all \ac{ir} nodes
which rely on the result will be scheduled for reclassification; once
the conditional is no longer \emph{unclassified}, it will be
appropriately handled by the algorithm.

\subsubsection{\gsainst{casesgl}}
To evaluate the lattice value of \refgsainst{casesgl}{Gguard}, the
final lattice value of the controlling \refgsainst{casereg}{Mregion}
must already have been calculated.

\begin{observation}\label{obs:ccp-casesgl-guard}
  When the \gsainst{casereg} evaluates to a constant, and the value
  matches one of the elements guarded by the \gsainst{casesgl}, the
  code in the region will be executed; the code in the
  \gsainst{casesgl} can be moved to the enclosing region and the
  \gsainst{casereg} can be deleted.
\end{observation}

\begin{algorithm}[h!]
  \label{algo:ccp-casesgl-classification}
  \caption{\gsainst{casesgl} (classficiation)}
  \begin{algorithmic}[1]
    \REQUIRE{inst $\neq$ nil $\wedge$ inst is \gsainst{casesgl}}
    \STATE{selector $\leftarrow$ zero-th operand}
    \STATE{inst.state $\leftarrow$ non-constant}
    \IF{selector $=$ constant}
      \STATE{inst.state $\leftarrow$ false}
      \STATE{value $\leftarrow$ selector.value}
      \FORALL{op: op $\in$ operands of inst}
        \IF{op is integer}
          \IF{op.value $=$ value}
            \STATE{inst.state $\leftarrow$ true}
          \ENDIF
        \ELSE
          \STATE{}
          \COMMENT{inv: op is an integer range}
          \IF{value.lo $<=$ op.value $\wedge$ op.value $<=$ value.hi}
            \STATE{inst.state $\leftarrow$ true}
          \ENDIF
        \ENDIF
      \ENDFOR
    \ENDIF
  \end{algorithmic}
\end{algorithm}

Determining the final state of a \gsainst{casereg} requires
examination of each of its \gsainst{casesgl} and possibly its
\gsainst{caseelse}.  While the processing of the \gsainst{casesgl} and
\gsainst{caseelse} are inextricably linked, it's handy to consider
them separately.

The \ac{fe} of the compiler, which ensures the source language rules,
guarantees that there will be no duplicate values in the
\gsainst{casesgl} entries; this, in turn, guarantees that, at most,
one \gsainst{casesgl} will evaluate to a constant value.  If the
controlling \gsainst{casereg} evaluates to a constant, but no
\gsainst{casesgl} instructions match the value, then the
\gsainst{caseelse} will be used.

\subsubsection{\gsainst{caseelse}}
To evaluate the lattice value of \refgsainst{caseelse}{Gguard}, the
final lattice value of the controlling \refgsainst{casereg}{Mregion}
must not match any \gsainst{casesgl} items contained in the region.

\begin{observation}\label{obs:ccp-casesgl-guard}
  When the \gsainst{casereg} evaluates to a constant, and the value
  matches none of the elements guarded by the \gsainst{casesgl}
  instructions, the code in the \gsainst{caseelse} region will be
  executed.
\end{observation}

\begin{algorithm}[h!]
  \label{algo:ccp-caseelse-classification}
  \caption{\gsainst{caseelse} (classficiation)}
  \begin{algorithmic}[1]
    \REQUIRE{inst $\neq$ nil $\wedge$ inst is \gsainst{caseelse}}
    \STATE{selector $\leftarrow$ zero-th operand}
    \STATE{inst.state $\leftarrow$ non-constant}
    \IF{region of inst is unclassified}
      \STATE{inst.state $\leftarrow$ true}
      \FORALL{reg: reg $\in$ instruction of \gsainst{casereg}}
        \STATE{}
        \COMMENT{Any \gsainst{casesgl} being true $\rightarrow$
    \gsainst{caseelse} is false}
        \IF{reg $=$ true}
          \STATE{inst.state $\leftarrow$ false}
        \ENDIF
      \ENDFOR
    \ELSE
      \STATE{}
      \COMMENT{\gsainst{casereg} already classified; this instruction
    also classified.}
    \ENDIF
  \end{algorithmic}
\end{algorithm}

The final state \gsainst{caseelse} can only be determined after all
the \gsainst{casesgl} instructions in the \gsainst{casereg} have been
fully processed.


%-----------------------------------------------------------------------------
\subsection{Merge}\label{const-prop:classification-of-merge}
Merges join control flow which was split by a conditional instruction.
When a merge is found to be controlled by a conditional that evaluates
to a constant value, the merge can be simplified to use only the
control path that will actually be executed, while all other paths
being merged will be considered \emph{unreachable}.

%    \hline $\textrm{const-op}_i$ & This annotation, used for
%    \emph{merge regions} and \emph{gates}, indicates that operand
%    \emph{i} ($0 \leq i < n$) has been selected as the only output
%    from the controlled
%    region. \\

\subsubsection{\gsainst{if}}

\begin{observation}\label{obs:ccp-if-merge}
  If the guard controlling the two (2) paths merged by
  \refgsainst{if}{Gif} evaluates to a constant value, one path will be
  taken and one path will not be taken.  However, if the guard does
  not evaluate to a constant, then it cannot be determined which path
  will be taken.
\end{observation}

\Refobs{obs:ccp-if-merge} lays the foundation for the possible lattice
values for an \gsainst{if} shown in table
\ref{table:const-prop-if-merge-lattice}.

\begin{table}[h!]
  \begin{tabularx}{\linewidth}{|l|X|}
    \hline Lattice Value & Description \\
    \hline unclassified & The initial value, but at the completion of
    the algorithm, all merges will have been more precisely classified
    into one of the remaining three classificiations. \\

    \hline $\textrm{const-op}_0$ & The conditional used to split the
    control flow has evaluated to a constant and that constant causes
    the \refgsainst{true}{Gguard} region to be executed; consequently
    the merge needs to only produce values associated with its first
    operand. \\

    \hline $\textrm{const-op}_1$ & The conditional used to split the
    control flow has evaluated to a constant and that constant causes
    the \refgsainst{false}{Gguard} region to be executed; consequently
    the merge needs to only produce values associated with its second
    operand.\\

    \hline non-constant & The conditional used to split the control flow
    has not evaluated to a constant; consequently the merge cannot be
    treated as a constant value. \\

    \hline unreachable & The merge is contained in a region which is
    \emph{unreachable}. \\
  \hline
  \end{tabularx}
  \caption{\gsainst{if} Lattice Values}\label{table:const-prop-if-merge-lattice}
\end{table}

\subsubsection{\gsainst{condmerge}}
\todo{Need more explanation here}
\begin{observation}
  When either of the operands of a \gsainst{condmerge} are
  \code{TRUE}, the code in the region will be executed.  If both
  operands evaluate to \code{FALSE}, the code in the region is
  unreachable.  Otherwise, the region is non-constant.
\end{observation}

\subsubsection{\gsainst{merge-loop}}

\todo{Need more explanation here}

A \gsainst{merge-loop} will always terminate in the
\emph{non-constant} state.  However, the loop body can be executed a
finite number of times\footnote{For the purposes of constant
  propagation, finite is one (1) or $\inf$.  Other analysis phases
  dealing with looping constructs may calculate the runtime
  characteristics of the looping construct.}.

\begin{table}[h!]
  \begin{tabularx}{\linewidth}{|l|l|X|}
    \hline top & backedge & Semantic \\
    \hline \code{FALSE} & \code{FALSE} & Internal compiler error \\
    \hline \code{FALSE} & \code{TRUE} & Internal compiler error \\
    \hline \code{TRUE} & \code{FALSE} & Loop executes only once \\
    \hline \code{TRUE} & \code{TRUE} & Infinite Loop \\
    \hline
  \end{tabularx}
  \caption{\gsainst{merge-loop} Constant Operand Semantics}\label{table:const-prop-loop-merge-lattice}
\end{table}

Table \ref{table:const-prop-loop-merge-lattice} shows the relationship
between constant operands of \gsainst{merge-loop} and the number of
iterations of the loop.

\begin{observation}
  A loop which only executes once can have the loop body moved to the
  region containing the \gsainst{merge-loop}.
\end{observation}


\begin{algorithm}[h!]
  \label{algo:ccp-merge-loop-once}
  \caption{\gsainst{merge-loop} (propagation)}
  \begin{algorithmic}[1]
    \REQUIRE{merge $\neq$ nil $\wedge$ merge is \gsainst{merge-loop}}
    \STATE{}
    \COMMENT{The first operand will always be true.}
    \STATE{}
    \COMMENT{The second operand indicates if the loop will continue executing.}
    \STATE{}
    \STATE{def $\leftarrow$ definition of second operand}
    \IF{def is ConstBool}
      \IF{def.value}
        \STATE{}
        \COMMENT{Infinite loop}
      \ELSE
        \STATE{}
        \COMMENT{Single iteration loop}
        \FORALL{inst: inst $\in$ instruction of \gsainst{merge}}
          \IF{inst is Gate}
            \STATE{op1 $\leftarrow$ first operand of inst}
            \STATE{op2 $\leftarrow$ second operand of inst}
            \STATE{Replace uses of inst \emph{in} merge with
    definition of op1.}
            \STATE{Replace uses of inst \emph{outside} merge with definition
    of op2.}
          \ENDIF
        \ENDFOR
      \ENDIF
    \ELSE
      \STATE{}
      \COMMENT{The loop is not constant.}
    \ENDIF
  \end{algorithmic}
\end{algorithm}

\subsubsection{\gsainst{case}}
\begin{observation}\label{obs:ccp-case-merge}
  If the selector controlling the \refgsainst{casesgl}{Gguard} and
  \refgsainst{caseelse}{Gguard} regions evaluates to a constant value,
  then only one of those regions will be executed.  If the selector
  does not evaluate to a constant value, then it cannot be determined,
  at compile time, which code path will be executed.

  When a constant selector is used, since only one region in a
  \code{CASE} statement can be active, the merge correspondingly will
  use only the results output from the active region.
\end{observation}

\Refobs{obs:ccp-case-merge} lays the foundation for the possible lattice
values for an \gsainst{case} shown in table
\ref{table:const-prop-case-merge-lattice}.

\begin{table}[h!]
  \begin{tabularx}{\linewidth}{|l|X|}
    \hline Lattice Value & Description \\
    \hline unclassified & The initial value, but at the completion of
    the algorithm, all merges will have been more precisely classified
    into one of the remaining three classificiations. \\

    \hline $\textrm{const-op}_n$ & The selector used has evaluated to
    a constant and that constant causes the \emph{n}-th region to be
    executed; consequently the merge needs to only produce values
    associated with its \emph{n}-th operand. \\

    \hline non-constant & The selector used has not evaluated to a
    constant; consequently the merge cannot be
    treated as a constant value. \\

    \hline unreachable & The merge is contained in a region which is
    \emph{unreachable}. \\
  \hline
  \end{tabularx}
  \caption{\gsainst{case} Lattice Values}\label{table:const-prop-case-merge-lattice}
\end{table}

%-----------------------------------------------------------------------------
\subsection{Gate}\label{const-prop:classification-of-gate}

Since gates only appear in merge regions, it is safe to say that the
classification produced for a gate is directly affected by the merge
region which immediately dominates the gate.  This property implies
that the final classification of a gate cannot be determined until the
final classification for the merge containing it has been computed.

However, once the classification of the merge has been determined, a
gate can easily be classified.

As can be seen from \refgsainst{gate}{Gmisc}, a gate has one operand
for each guarded region dominating the merge region and an operand
indicating the dominating merge region.

While there is only one real \refgsainst{gate}{Gmisc} instruction, the
number of operands which are present depends on the type of the
controlling merge instruction: \refgsainst{if}{Gmerge},
\refgsainst{case}{Gmerge}, \refgsainst{loop}{Gmerge},
\refgsainst{condmerge}{Gmerge}.  Because of this, each specialization of
the \gsainst{gate} will be treated separately in the following
sections.

\subsubsection{\gsainst{if}}

\begin{observation}
  An \refgsainst{if}{Gif} merge has two (2) operands: one for the true
  region and one for the false region being merged.
\end{observation}

\begin{observation}
  A gate controlled by an \refgsainst{if}{Gif} merge has three (3)
  operands: one referencing the controlling merge, one for the true
  region and one for the false region.
\end{observation}

Once all nodes in the \ac{ir} have been classified by this algorithm,
an \gsainst{if} merge can only have three (3) values:
\emph{non-constant}, \emph{$\textrm{const-op}_0$}, or
\emph{$\textrm{const-op}_1$}; knowing this, it is easy to show what a
gate controlled by an \gsainst{if} merge will produce.
\Tableref{table:const-prop-gate-if-lattice} shows the values which can
be produced.

\begin{table}[h!]
  \begin{tabularx}{\linewidth}{|l|l|X|}
    \hline Merge Value & Lattice Value & Description \\

    \hline $\textrm{const-op}_0$ & $\textrm{const-op}_0$ & When the
    controlling merge evaluates to its first operand, the gate
    also evaluates to its first operand. \\

    \hline $\textrm{const-op}_1$ & $\textrm{const-op}_1$ & When the
    controlling merge evaluates to its second operand, the gate also
    evaluates to its second operand. \\

    \hline non-constant & $\textrm{const-op}_0$ & If the merge does not
    evaluate to a constant value, but both operands of the gate are
    constant, \emph{and the same value}, the gate can be replaced with
    that constant value. \\

    \hline non-constant & non-constant & If the merge does not
    evaluate to a constant, and the gate is not constant, then it will
    be marked as non-constant. \\
    \hline
  \end{tabularx}
  \caption{Gate Controlled By \gsainst{if} Lattice Values}\label{table:const-prop-gate-if-lattice}
\end{table}

\todo{Provide three four (4) examples here which demonstrate all
  conditions of the if-gate.  Should the examples be graphic or just a
  gsa dump?}

\subsubsection{\gsainst{condmerge}}

A \gsainst{condmerge} instruction does not dominate gates the way that
other merge regions do; since there are no gates contained in such a
region, there is nothing requiring discussion.

\subsubsection{\gsainst{merge-loop}}

\begin{observation}
  \gsainst{merge-loop} handles the special needs of a loop's front-
  and back-edges.  It does not immediately dominate any gates, but
  does dominate the gates in the back-edge of the loop.
\end{observation}

\refgsainst{merge-loop}{Gmerge} is a special merge instruction which never
evaluates to a constant value.  Gates which are controlled by this
merge will never directly evaluate to a constant value during
\ac{ccp}.

\todo{This section (gates for \gsainst{loop}) is incomplete.}

\subsubsection{\gsainst{case}}

\subsection{Case}\label{const-prop:classification-of-case}

\refgsainst{casereg}{Mregion} is unique in the \ac{gsa} \ac{ir}
because one, and only one, control path will be executed each time the
construct is entered.  If the \emph{selector} evaluates to a constant,
then the control path can be selected at compile time and all other
control paths can be removed from the program.


\begin{table}[h!]
  \begin{tabularx}{\linewidth}{|l|X|}
    \hline Lattice Value & Description \\

    \hline unclassified & The initial state for the gate.  By the end
    of the execution of the algorithm, no guards will have this
    value. \\

    \hline $\textrm{const-op}_{n-1}$ & If the \emph{selector} evaluates to
    a constant, the \gsainst{casereg} construct will maintain an
    \emph{unclassified} state until the executed region has been
    determined.  Once the region to be executed has been determined,
    the \gsainst{casereg} construct will be attributed with this
    lattice value - which indicates which control region is to be
    executed.  A value of 0 indicates the first control region, while
    a value of \emph{n-1} indicates the \refgsainst{caseelse}{Gguard}
    region. \\

    \hline non-constant & If the selectore does not evaluate to a
    constant, the entire structure is considered to be
    non-constant.  In this case, none of the control paths can be
    removed from the program. \\

    \hline unreachable & The guard resides in a region that has
    evaluated to \emph{unreachable}. \\
    \hline
  \end{tabularx}
  \caption{\gsainst{casereg} Lattice Values}\label{table:const-prop-guard-casereg-lattice}
\end{table}

\begin{invariant}
  At the completion of the algorithm, no region in the
  \gsainst{casereg} will have the unclassified lattice value.
\end{invariant}

\begin{invariant}
  When the \gsainst{casereg} is reachable, and when the selector
  evaluates to a constant, a single region within the
  \gsainst{casereg} will have the true lattice value.  All other
  regions will have the false lattice value.
\end{invariant}

\begin{invariant}
  When the \gsainst{casereg} is reachable, and when the selector does
  not evaluate to a constant, every region in the \gsainst{casereg}
  will have the non-constant lattice value.
\end{invariant}

\begin{invariant}
  When the \gsainst{casereg} is not reachable, the \gsainst{casereg}
  and all regions contained in it will be attributed unreachable.
\end{invariant}

%-----------------------------------------------------------------------------
\subsection{Instruction}
