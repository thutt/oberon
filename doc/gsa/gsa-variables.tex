% Copyright (c) 2001-2022 Logic Magicians Software
\newenvironment{variable}[1]{\clearpage\section{\gsavar{#1}\label{var:#1}}\vspace{-\baselineskip}\rule{\textwidth}{.5pt}}{\clearpage}

\chapter{GSA Variables}

The \ac{gsa} \ac{ir} ses several \emph{pseudo-variables} internally.
Some of the variables are actually associated with the generation of
code, and others are simply used as a notational convenience to
simplify the obtaining a consistent view of memory at all times.

While none of these variables are directly accessible through a user
program, they are directly related to the original source code through
the process of translation.

The sections in this chapter describe the variable types created by
the compiler and their purpose.

\begin{variable}{nlm}
  \gsavar{nlm} is used to create a consistent view of non-local
  memory.  From a scoping view of the Oberon lanuage, \emph{non-local
    memory} means:
  \begin{enumerate}
  \item global variables
  \item variables from an outer procedure
  \item \byref parameters
  \end{enumerate}.

  Any instruction which accesses or modifies non-local memory will be
  provided a copy of \gsavar{nlm}.  The purpose of this is to enable
  simple algorithms to ensure that all non-local memory is written to
  their home memory locations at appropriate times.

  Further, since updates to nonlocal memory are modeled through the
  operand list of \gsainst{exit} (see \refgsainst{exit}{Mregion}), it
  is quite simple to handle alias analysis.  Also, the instructions
  which update non-local memory produce a new value for
  \gsavar{nlm}, which consequently links all access and updates to
  non-local memory through the \duc of \gsavar{nlm}.

  \gsainst{createnlm} (see \refgsainst{createnlm}{Mmemory})
  creates the initial value for the variable.

  \gsainst{deletenlm} (see \refgsainst{deletenlm}{Mmemory})
  finalizes \gsavar{nlm} at the end of the procedure.
\end{variable}

\begin{variable}{mem}
  This variable has not yet been implemented.
\end{variable}

\begin{variable}{result}
  \gsavar{result} is assigned the value of the \code{RETURN}
  instruction.

  This variable used used to model the actual return value through an
  operand of \gsainst{exit} (see \refgsainst{exit}{Mregion}).
\end{variable}

\begin{variable}{return}
  This variable is used in conjuntion with \gsavar{result} to model
  the \code{RETURN} instruction.  Since \ac{gsa} does not handle
  unstructured control flow, it is necessary to model \code{RETURN}
  statements specially.

  Each \code{RETURN} statement in a procedure body assigns to
  \gsavar{return}, and all statements following the \code{RETURN}
  statement are guarded by checking if the variable has been
  assigned.

  \todo{More description is required here}.

\end{variable}

\begin{variable}{exit}
  \gsavar{exit} is used to model the \code{EXIT} instruction.  Each
  \code{LOOP} strucutre is assigned a \gsavar{exit}, and each
  \code{EXIT} assigns to that variable.

  All instructions in the loop, following the \code{EXIT}, are guarded
  by the value of the appropriate \gsavar{exit}; if it is not true,
  then the \code{EXIT} statement has not been executed.

  \todo{Need a better description}
\end{variable}
