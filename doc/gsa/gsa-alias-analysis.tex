% Copyright (c) 2001-2023 Logic Magicians Software
\chapter{Alias Analysis}

\ac{aa} is a process to determine if access to memory through two or
more different data paths yields the same memory location.  For a
compiler which does not invoke any \ac{cia}, aliases are generally not
a problem since all access to data will always access data through a
variable's home memory location.

However, for a compiler that performs \ac{cia}, it may turn out that
the value contained in aliased memory is currently held in a register
- in which case access to the physical memory can yield the wrong
value for the variable.  Consider, for example, the code presented in
\figref{alias-example}.  For the invocation of procedure
\texttt{P0(z,~z)}, both \texttt{x} and \texttt{y} refer to the same
variable, \texttt{z}.  A write to either \texttt{x} or \texttt{y} will
change the value of the other variable.  Without tracking when
multiple variables can reference the same memory location, an compiler
can generate incorrect machine code by enregistering one or more of
the aliased variables.


\begin{figure}[h]
\begin{alltt}
PROCEDURE P0(VAR x, y : INTEGER);
BEGIN
  y := 10;
  INC(x);
  IF y = 10 THEN
    x := 0;
  ELSE
    x := 1;
  END;
END P0;

PROCEDURE P1;
VAR
  z : INTEGER;
BEGIN
  z := 0;
  P0(z, z);
  ASSERT(z = 1);
END P1;
\end{alltt}
  \caption{Example with aliased variables}\label{fig:alias-example}
\end{figure}


\section{Definition of Terms}

\begin{enumerate}
  \item Alias State

    The determination if two different variables refer to the same
    memory location will result in one of three states: \emph{do not
      alias}, \emph{may alias}, \emph{alias}.

    \begin{description}
    \item{do not alias}

      A \emph{do not alias} indicates that two memory references
      definitely do not refer to the same memory location.

    \item{may alias}

      A \emph{may alias} occurs when two memory references \emph{might}
      refer to the same memory location.

    \item{alias}

      An \emph{alias} occurs when two accesses to memory definitely
      refer to the same memory location.
    \end{description}

  \item Alias Compatible

    Two variables are \emph{alias compatible} when one or more of the
    following holds:

    \begin{enumerate}
      \item The type of the first variable is identical to the type of
        the second variable.

      \item The type of the first variable is a \texttt{RECORD} and is
        a base of the second variable's type.

      \item The type of the first variable is a \texttt{RECORD} and is
        an extension of the second variable's type.

      \item The type of the first variable is a \texttt{RECORD} and
        the second variable is a base of the first variable's type.

      \item The type of the first variable is a \texttt{RECORD} and
        the second variable is an extension of the first variable's
        type.

      \item The type of the first variable is a \emph{pointer to
        record} and the second variable is a pointer to an alias
        compatible type.

    \end{enumerate}

\end{enumerate}

\section{Oberon-2 Aliasing}

Aliasing in Oberon-2 is possible through the following venues:

\begin{enumerate}
\item \texttt{VAR} parameters can can alias to alias compatible global
  variables.

\item \texttt{VAR} parameters can alias to other, alias compatible,
  \texttt{VAR} parameters.

\item \texttt{VAR} parameters of a nested procedure can alias to alias
  compatible variables from an enclosing procedure.

\item \texttt{VAR} parameters can alias to alias compatible variables
  via a pointer dereference.

\item A pointer to an array type can alias a pointer to an
  identically-sized, and typed, array.
\end{enumerate}

%% \begin{algorithm}[h!]
%%   \label{algo:aa-gather-non-local}
%%   \caption{Create worklist of non-local memory accesses}
%%   \begin{algorithmic}[1]
%%     \REQUIRE{greg $\neq$ nil}


%%     \FORALL{rgn: rgn $\in$ greg}
%%       \FORALL{inst: inst $\in$ rgn}
%%         \IF{$\neg$(i is Region)}
%%           \FORALL{opnd: opnd $\in$ inst}
%%             \IF{NonLocalAccess(opnd.def)}
%%             \STATE{Add opnd to worklist}
%%             \ENDIF
%%           \ENDFOR
%%         \ENDIF
%%       \ENDFOR
%%     \ENDFOR
%%   \end{algorithmic}
%% \end{algorithm}


\section{Basic Algorithm}

Performing \ac{aa} is very straightforward given the small number of
major steps.

\begin{enumerate}
\item Create a worklist of all \emph{non-local} memory access in a
  region by examining the definition of each operand in the region
  (\xref{th:operand}).  Refer to \algoref{aa-determine-non-local}

\begin{algorithm}[h!]
  \caption{Determining non-local access}\label{algo:aa-determine-non-local}
  \begin{algorithmic}[1]
    \REQUIRE{usable $\neq$ nil}

    %%%% This Instruction test might need to be a Result test.  See
    %%%% O3MGSA.AssignThruInstruction.

    \IF{usable is Instruction}
      \STATE{inst $=$ usable(Instruction)}
      \IF{inst.opcode $\in$ \{ Hget, Hput, Hmemr, Hmemw, Gsystem\}}
      \STATE{}\COMMENT{Accesses via SYSTEM always alias.}
      \STATE{}\COMMENT{Hget (\xref{hardware:get}), Hput (\xref{hardware:put})}
      \STATE{}\COMMENT{Hmemr (\xref{hardware:memr}), Hmemw (\xref{hardware:memw})}
      \STATE{}\COMMENT{Gsystem (\xref{class:system})}
      \ELSIF {inst.opcode $\in$ \{Maccess, Mupdate\}}
        \STATE{}\COMMENT{Maccess, Mupdate \xref{class:memory-access}}
        \IF{inst.subcl $\in$ \{Mfield, Melement\}}
        \STATE{}\COMMENT{Mfield (\xref{memory-access:field}), Melement (\xref{memory-access:element})}
          \STATE{operand $=$ first operand of inst}
          \IF{operand is Instruction}
            \STATE{}\COMMENT{Referencing a non-local object.}
          \ELSE
            \STATE{}\COMMENT{Referencing a local object.}
          \ENDIF
        \ELSE
          \STATE{}\COMMENT{All other memory access opcodes are non-local.}
        \ENDIF

      \ELSIF {inst.opcode $=$ Mmemory}
        \STATE{}\COMMENT{Marraycopy (\xref{memory:arraycopy}), Mrecordcopy (\xref{memory:recordcopy})}
        \STATE{}\COMMENT{Copying an array or record aliases.}
      \ENDIF
    \ELSE
      \STATE{}\COMMENT{Not an alias}
    \ENDIF
  \end{algorithmic}
\end{algorithm}

\item Permute pairs of memory access from the worklist.  The items
  will be called \emph{left} and \emph{right}.

  \begin{enumerate}
    \item If both \emph{left} and \emph{right} refer to the same
      address (determinable from the operands of the instructions
      accessing memory), they \emph{do alias}.

    \item If \emph{left} and \emph{right} are \emph{alias compatible},
      then they two memory accesses \emph{may alias}.

    \item If \emph{left} and \emph{right} are not \emph{alias
      compatible}, then they two memory accesses \emph{do not alias}.
  \end{enumerate}

  \item For each pair, \emph{left} and \emph{right} that \emph{may
    alias}:
    \begin{enumerate}
      \item For \emph{left}, create a \gsainst{mayalias} instruction
        with two operands.  The first operand is the address
        accessed by \emph{left}, and the second operand is the address
        accessed by \emph{right}.  Replace the second operand of the
        \emph{left} instruction (the computed address) with the
        \gsainst{mayalias}.  See \xref{class:memory-access}.

      \item For \emph{right}, create a \gsainst{mayalias} instruction
        with two operands.  The first operand is the address accessed
        by \emph{right}, and the second operand is the address
        accessed by \emph{left}.  Replace the second operand of the
        \emph{right} instruction (the computed address) with the
        \gsainst{mayalias}.  See \xref{class:memory-access}.

    \end{enumerate}
\end{enumerate}
