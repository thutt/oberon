% Copyright (c) 2001-2023 Logic Magicians Software
\newcommand{\inv}[1] {invariant: #1\xspace}
\newcommand{\bit}[1] {bit #1\xspace}

% \note: for writing notes in the text which will not appear
% in the printed output.
\newcommand{\note}[1] {}

% multisubcl{subclass-list}
% provides a uniform way to provide a list of subclasses as the
% subclass operand of \gsadefn (and avoids underfull boxes)
\newcommand{\multisubcl}[1]{$\left\{\begin{minipage}{.60in}{#1}\end{minipage}\right\}$}


\newenvironment{instruction}[1]{\clearpage\subsection{\gsainst{#1}}\vspace{-\baselineskip}\rule{\textwidth}{.5pt}}{\clearpage}
\newenvironment{results}{\subsubsection{Results}\begin{enumerate}}{\end{enumerate}}
\newenvironment{dyntype}{\subsubsection{Dynamic Type}\begin{enumerate}}{\end{enumerate}}

% \operand{runtime-type}{description}
\newcommand{\operand}[2]{[#1] #2}
\newenvironment{operands}{\subsubsection{Operands}\begin{enumerate}}{\end{enumerate}}
\newenvironment{notes}{\subsubsection{Notes}}{}
\newcommand{\nresults}{\subsubsection{Results} There are no usable
  results produced.}

\newcommand{\voperands}[1]{\subsubsection{Operands} #1}
\newcommand{\noperands}{\voperands{There are no operands.}}

\newenvironment{seealso}{\subsubsection{See Also}}{}

% \info{datatype}{class}{subclass}{description}{example}{figure-name}{figure-class}
\newcommand{\info}[7]{
  \label{inst:#7-#6}
  \vspace{-3\baselineskip}\begin{flushright}\small{#5}\end{flushright}
  \vspace{-.30\baselineskip}
  \begin{minipage}[t]{.55\textwidth}#4\end{minipage}\hfill
  \begin{minipage}[t]{.35\textwidth}
    \begin{small}
      \begin{flushright}
        [#2, #3] {\linebreak}
        #1 {\linebreak}
      \end{flushright}
    \end{small}
  \end{minipage}
  \begin{figure}[h!]
    \includegraphics{gen-gsa-instruction-#6.1}
    \caption{\gsainst{#6}}\label{fig:instruction-#7-#6}
  \end{figure}
}


\chapter{GSA Instructions}

The \ac{gsa} \ac{ir} is a conglomeration of simple data structures
which can be combined into many complex configurations.  This chapter
serves as the definitive reference of the \ac{gsa} data structures.

The \ac{gsa} system implicitly uses the results of instructions in the
operands of other instructions, but this may not be entirely clear
from the type defintions in the source modules.

\begin{figure}[h!]
\includegraphics{gsa-instruction-simple.1}
\caption{\ac{gsa} Simplified Instruction Sample}\label{fig:instruction-simple}
\end{figure}

As an example of how the \ac{gsa} data structures function, figure
\ref{fig:instruction-simple} shows a greatly simplified sample of how
instruction sequences are represented in \ac{gsa} format, and how
results synthesized by instructions can be used as operands of other
instructions.

The remaining sections of this chapter document each of the \ac{gsa}
instructions.


%%%%%%%%%%%%%%%%%%%%%%%%%%%%%%%%%%%%%%%%%%%%%%%%%%%%%%%%%%%%%%%%%%%%%%%%%%%%%%
\section{Arithmetic}\label{class:arithmetic}

Arithmetic instructions perform calculations based on one or more
input operands and produce one or more results.  The results produced
can be used by any other instruction as input.

Figure \ref{tab:instruction-arithmetic} shows all the arithmetic
instructions generated by this compiler.

\begin{table}[h!]
  \begin{tabularx}{\linewidth}{|l|X|}
    \hline Instruction & Description \\
    \hline \gsainst{abs} & Absolute value \\
    \hline \gsainst{neg} & Arithmetic negation \\
    \hline \gsainst{add} & Addition \\
    \hline \gsainst{sub} & Subtraction \\
    \hline \gsainst{div} & Division \\
    \hline \gsainst{mul} & Multiplication \\
    \hline \gsainst{mod} & Modulus \\
    \hline \gsainst{odd} & Tests if the operand is an odd value. \\
    \hline \gsainst{rol} & Rotate left \\
    \hline \gsainst{ror} & Rotate right \\
    \hline \gsainst{lsl} & Logical shift left \\
    \hline \gsainst{lsr} & Logical shift right \\
    \hline \gsainst{asl} & Arithmetic shift left \\
    \hline \gsainst{asr} & Arithmetic shift right \\
    \hline \gsainst{convert} & Type conversion \\
    \hline \gsainst{not} & Boolean negation \\
    \hline \gsainst{get} & Direct read of \gsavar{memory}. \\
    \hline \gsainst{put} & Direct write of \gsavar{memory}. \\
    \hline \gsainst{getreg} & Direct read of CPU register. \\
    \hline \gsainst{putreg} & Direct write of CPU register. \\
    \hline \gsainst{memr} & Direct read of \gsavar{memory}. \\
    \hline \gsainst{memw} & Direct write of \gsavar{memory}. \\
    \hline
  \end{tabularx}
\caption{Arithmetic Instructions}\label{tab:instruction-arithmetic}
\end{table}

The \ac{fe} guarantees that the operands to all arithmetic
instructions are of the same size, but it is still the responsibility
of the \ac{be} to be able to handle operands of various sizes.  This
\ac{ir} accomplishes this by using special subclasses for the
arithmetic operations - these subclasses indicate the size of the
operands.

\begin{table}[h!]
  \begin{tabularx}{\linewidth}{|l|c|X|}
    \hline Subclass & Value & Description \\
    \hline \code{asS1} &  0 & 1-byte signed  \\
    \hline \code{asU1} &  1 & 1-byte unsigned  \\
    \hline \code{asS2} &  2 &  2-byte signed \\
    \hline \code{asU2} &  3 &  2-byte unsigned \\
    \hline \code{asS4} &  4 &  4-byte signed \\
    \hline \code{asU4} &  5 &  4-byte unsigned \\
    \hline             & 6 $\ldots$ 21 & reserved \\
    \hline \code{asR8} & 22 &  8-byte floating point \\
    \hline \code{asR4} & 23 &  4-byte floating point \\
    \hline              & 24 $\ldots$ 28 & reserved \\
    \hline \code{asST} & 31 &  ASCIIZ string  \\
    \hline
  \end{tabularx}
\caption{Arithmetic Subclasses}\label{tab:instruction-arithmetic-subclasses}
\end{table}

The values for the subclasses have been set up to be used with the
Oberon \code{SET} type, and therefore the values have been picked to
allow more integer types or more floating point types to be added
without impacting the arrangement of the existing values.

For integer calculations, even-valued subclasses handle signed
arithmetic and odd-valued subclasses handle unsigned arithmetic.  The
integer subclasses have been arranged so that as the value of the
subclass increases so does the numerical range of the data type.

For floating point calculations, the values have been arranged so that
as the value of the subclass decreases the numerical range of the data
type increases.

%-----------------------------------------------------------------------------
\begin{instruction}{abs}
  \info{Instruction}{Aabs}{\multisubcl{asS1, asS2, asS4, asR8,
      asR4}}{Models the absolute value of signed integers and real
    numbers.}{Aabs.Mod}{abs}{arithmetic}

  \begin{results}
  \item The result is the absolute value of the operand.
  \end{results}

  \begin{operands}
  \item \operand{Usable}{The value for which the absolute value should
      be produced.}
  \end{operands}
\end{instruction}

%-----------------------------------------------------------------------------
\begin{instruction}{add}
  \info{Instruction}{Aadd}{\multisubcl{asS1, asS2, asS4, asR8,
      asR4}}{Models addition of integer and real
    numbers.}{Aadd.Mod}{add}{arithmetic}

  \begin{notes}
    The \ac{fe} guarantees the two operands are of the size
    represented by the instruction subclass.
  \end{notes}

  \begin{results}
  \item ${op}_{0} + {op}_{1}$
  \end{results}

  \begin{operands}
  \item \operand{Usable}{The left addend.}
  \item \operand{Usable}{The right addend.}
  \end{operands}

  \begin{seealso}
    \refgsainst{sub}{arithmetic}, \refgsainst{mul}{arithmetic},
    \refgsainst{mod}{arithmetic}, \refgsainst{div}{arithmetic}
  \end{seealso}
\end{instruction}

%-----------------------------------------------------------------------------
\begin{instruction}{convert}
  \info{Instruction}{Aconvert}{\multisubcl{asS1, asU1, asS2, asU2,
      asS4, asU4, asR8, asR4}}{Models a checked type
    conversion.}{Aconvert.Mod}{convert}{arithmetic}

  \todo{There is something wrong with the typesetting of the
    subclasses of this instruction.}
  \begin{notes}
    The \ac{fe} guarantees that the conversion specified is legal with
    respect to the \ac{olid}\todo{Econvert in
      O3ImplementationDetails.tex}.

    It would be nice to institute a rule that \gsainst{convert} never
    has a constant as its operand, and to enforce this rule by
    applying it stringently to the \ac{fe} such that no \ac{ast} will
    be created with convert nodes with a constant operand.  However
    nice it may seem at first, it's a rule that is hard to implement.
    Forcing the \ac{fe} to fold all expressions is possible, but since
    Oberon only \emph{requires} constant expressions in certain
    constructs, it would be an extra burden on the semantics of the
    tree builder.  Further, it's a task that should not be taken
    lightly if cross compilation is desired; the evaluation of a
    constant expression on the host machine may not yield the the same
    results on the target machine, especially in the realm of floating
    point numbers\footnote{This argument is a bit of a red herring: if
      the constant expressions do not evaluate the same on the host
      and the target, then it is not possible to correctly create
      floating point constants from expressions, and therefore the
      compiler is a broken cross compiler.}.

    However, even if the \ac{fe} ensures that the \ac{ast} does not
    create nonconformant convert nodes, the \ac{be} is an entirely
    different beast.  Consider the following code:

\begin{verbatim}
VAR
  i : SHORTINT;
  j : LONGINT;

i := 10;
j := 5 + i;
\end{verbatim}

    The \ac{fe} is not written to determine that the result of the
    addition is a constant value, but the \ac{be} will detect that.
    Consequently, what seems like a good idea to simplify the work of
    the compiler actually ends up making more work and causing more
    potential problems due to other parts of the compiler naively
    violating the data structure rules.

    The bottom line is that if an operand must be a constant value,
    the processing for that operand must be prepared to process
    \gsainst{convert}, or special care must be taken to avoid convert
    nodes when building the \ac{ast}.

    The method used to perform the conversion must conform to the type
    inclusion rules of Oberon.
  \end{notes}

  \begin{results}
  \item The value of the input operand converted to the destination type.
  \end{results}

  \begin{operands}
  \item \operand{Usable}{The operand is the value to be converted.}
  \end{operands}
\end{instruction}

%-----------------------------------------------------------------------------
\begin{instruction}{div}
  \info{Instruction}{Adiv}{\multisubcl{asS1, asS2, asS4, asR8,
      asR4}}{Models division of integers and real
    numbers.}{Adiv.Mod}{div}{arithmetic}

  \begin{notes}
    The \ac{fe} guarantees the two operands are of the size
    represented by the instruction subclass.
  \end{notes}

  \begin{results}
  \item $\textrm{dividend}_0 \div \textrm{divisor}_1$.
  \end{results}

  \begin{operands}
  \item \operand{Usable}{The value of the dividend.}
  \item \operand{Usable}{The value of the divisor.}
  \end{operands}

  \begin{seealso}
    \refgsainst{add}{arithmetic}, \refgsainst{sub}{arithmetic},
    \refgsainst{mul}{arithmetic}, \refgsainst{mod}{arithmetic}
  \end{seealso}
\end{instruction}

%-----------------------------------------------------------------------------
\begin{instruction}{mod}
  \info{Instruction}{Amod}{\multisubcl{asS1, asS2, asS4}}{Models
    modulo arithmetic on integers.}{Amod.Mod}{mod}{arithmetic}

  \begin{notes}
    The \ac{fe} guarantees the two operands are of the size
    represented by the instruction subclass.

    The code generated for \gsainst{mod} must conform to the following
    relations:

    \[ x \bmod y \equiv x - y \lfloor {x \div y} \rfloor \]

\todo{This equation needs to be cleaned up}
    \[x \bmod y = \left\{
      \begin{array}{l|l}
        0 \leq x \bmod y < y & y > 0 \\
        \mbox{undefined} & y = 0 \\
        y < x \bmod y \leq 0 & y < 0
      \end{array}
      \right. \]
  \end{notes}

  \begin{results}
  \item $\textrm{dividend} \bmod \textrm{divisor}$.
  \end{results}

  \begin{operands}
  \item \operand{Usable}{The value of the dividend.}
  \item \operand{Usable}{The value of the divisor.}
  \end{operands}

  \begin{seealso}
    \refgsainst{add}{arithmetic}, \refgsainst{sub}{arithmetic},
    \refgsainst{mul}{arithmetic}, \refgsainst{div}{arithmetic}
  \end{seealso}
\end{instruction}

%-----------------------------------------------------------------------------
\begin{instruction}{mul}
  \info{Instruction}{Amul}{\multisubcl{asS1, asS2, asS4, asR8,
      asR4}}{Models multiplication of integers and real
    numbers.}{Amul.Mod}{mul}{arithmetic}

  \begin{notes}
    The \ac{fe} guarantees the two operands are of the size
    represented by the instruction subclass.
  \end{notes}

  \begin{results}
  \item ${op}_0 \times {op}_1$
  \end{results}

  \begin{operands}
  \item \operand{Usable}{The value of the first multiplicand.}
  \item \operand{Usable}{The value of the second multiplicand.}
  \end{operands}

  \todo{There is no discussion about out-of-range values or results,
    but there should be.}

  \begin{seealso}
    \refgsainst{add}{arithmetic}, \refgsainst{sub}{arithmetic},
    \refgsainst{mod}{arithmetic}, \refgsainst{div}{arithmetic}
  \end{seealso}
\end{instruction}

%-----------------------------------------------------------------------------
\begin{instruction}{neg}
  \info{Instruction}{Aneg}{\multisubcl{asS1, asS2, asS4, asR8,
      asR4}}{Negates signed integers and real
    numbers.}{Aneg.Mod}{neg}{arithmetic}

  \begin{results}
  \item The result is the negative value of the operand.
  \end{results}

  \begin{operands}
  \item \operand{Usable}{The value which is to be negated.}
  \end{operands}

  \begin{seealso}
    \refgsainst{not}{arithmetic}
  \end{seealso}
\end{instruction}

%-----------------------------------------------------------------------------
\begin{instruction}{not}
  \info{Instruction}{Anot}{asU1}{Models boolean
    negation.}{Anot.Mod}{not}{arithmetic}

  \begin{results}
  \item The result of this instruction is the boolean converse of the
    operand.
  \end{results}

  \begin{operands}
  \item \operand{Usable}{The operand is a boolean value which is to be
      negated.}
  \end{operands}

  \begin{seealso}
    \refgsainst{neg}{arithmetic},
    \refgsainst{eql}{conditional},
    \refgsainst{neq}{conditional},
    \refgsainst{lss}{conditional},
    \refgsainst{leq}{conditional},
    \refgsainst{gtr}{conditional},
    \refgsainst{geq}{conditional}
  \end{seealso}
\end{instruction}

%-----------------------------------------------------------------------------
\begin{instruction}{odd}
  \info{Instruction}{Aodd}{\multisubcl{asS1, asS2, asS4}}{Determines
    if the input operand is \emph{odd}.}{Aodd.Mod}{odd}{arithmetic}

  \begin{results}
  \item $\textrm{value} \bmod 2 = 0$.
  \end{results}

  \begin{operands}
  \item The value which should be tested for \emph{odd-ness}.
  \end{operands}
\end{instruction}

%-----------------------------------------------------------------------------
\begin{instruction}{sub}
  \info{Instruction}{Asub}{\multisubcl{asS1, asS2, asS4, asR8,
      asR4}}{Models subtraction of integers and real
    numbers.}{Asub.Mod}{sub}{arithmetic}

  \begin{notes}
    The \ac{fe} guarantees the two operands are of the size
    represented by the instruction subclass.
  \end{notes}

  \begin{results}
  \item ${op}_{0} - {op}_{1}$.
  \end{results}

  \begin{operands}
  \item \operand{Usable}{The value of the minuend.}
  \item \operand{Usable}{The value of the subtrahend.}
  \end{operands}

  \begin{seealso}
    \refgsainst{add}{arithmetic}, \refgsainst{mul}{arithmetic},
    \refgsainst{mod}{arithmetic}, \refgsainst{div}{arithmetic}
  \end{seealso}
\end{instruction}

%%%%%%%%%%%%%%%%%%%%%%%%%%%%%%%%%%%%%%%%%%%%%%%%%%%%%%%%%%%%%%%%%%%%%%%%%%%%%%
\section{Conditional}\label{class:conditional}
Conditional instructions allow values to be tested for specific
properties and for the control path of the program to subsequently
split.

\begin{table}[h!]
  \begin{tabularx}{\linewidth}{|l|X|}
    \hline Instruction & Description \\
    \hline \gsainst{eql} & $=$ \\
    \hline \gsainst{neq} & $\neq$ \\
    \hline \gsainst{lss} & $<$ \\
    \hline \gsainst{leq} & $\leq$ \\
    \hline \gsainst{gtr} & $>$\\
    \hline \gsainst{geq} & $\geq$\\
    \hline
  \end{tabularx}
\caption{Conditional Instructions}\label{tab:instruction-conditional}
\end{table}

Table \ref{tab:instruction-conditional} lists all the
conditional instructs supported.

%-----------------------------------------------------------------------------
\begin{instruction}{eql}
  \info{Instruction}{Ceql}{\multisubcl{asS1, asU1, asS2, asU2, asS4,
      asU4, asR8, asR4, asST}}{Models the $=$
    test.}{Ceql.Mod}{eql}{conditional}

  \begin{notes}
    \inv{The \ac{fe} guarantees that the generated \ac{ast} is
      type-correct so that mismatches between \ac{gsa} instructions
      will not occur when generating \ac{gsa} form.}
  \end{notes}

  \begin{results}
  \item ${op}_0 = {op}_1$
  \end{results}

  \begin{operands}
  \item \operand{Usable}{The value of the left operand from the
      source.}
  \item \operand{Usable}{The value of the right operand from the
      source.}
  \end{operands}

  \begin{seealso}
    \refgsainst{not}{arithmetic}
    \refgsainst{neq}{conditional},
    \refgsainst{lss}{conditional},
    \refgsainst{leq}{conditional},
    \refgsainst{gtr}{conditional},
    \refgsainst{geq}{conditional}
  \end{seealso}
\end{instruction}

%-----------------------------------------------------------------------------
\begin{instruction}{geq}
  \info{Instruction}{Cgeq}{\multisubcl{asS1, asU1, asS2, asU2, asS4,
      asU4, asR8, asR4, asST}}{Models the $\geq$
    test.}{Cgeq.Mod}{geq}{conditional}

  \begin{notes}
    \inv{The \ac{fe} guarantees that the generated \ac{ast} is
      type-correct so that mismatches between \ac{gsa} instructions
      will not occur when generating \ac{gsa} form.}
  \end{notes}

  \begin{results}
  \item ${op}_0 \geq {op}_1$
  \end{results}

  \begin{operands}
  \item \operand{Usable}{The value of the left operand from the source.}
  \item \operand{Usable}{The value of the right operand from the source.}
  \end{operands}

  \begin{seealso}
    \refgsainst{not}{arithmetic}
    \refgsainst{eql}{conditional},
    \refgsainst{neq}{conditional},
    \refgsainst{lss}{conditional},
    \refgsainst{leq}{conditional},
    \refgsainst{gtr}{conditional},
  \end{seealso}
\end{instruction}

%-----------------------------------------------------------------------------
\begin{instruction}{gtr}
  \info{Instruction}{Cgtr}{\multisubcl{asS1, asU1, asS2, asU2, asS4,
      asU4, asR8, asR4, asST}}{Models the $>$
    test.}{Cgtr.Mod}{gtr}{conditional}

  \begin{notes}
    \inv{The \ac{fe} guarantees that the generated \ac{ast} is
      type-correct so that mismatches between \ac{gsa} instructions
      will not occur when generating \ac{gsa} form.}
  \end{notes}

  \begin{results}
  \item ${op}_0 > {op}_1$
  \end{results}

  \begin{operands}
  \item \operand{Usable}{The value of the left operand from the source.}
  \item \operand{Usable}{The value of the right operand from the source.}
  \end{operands}

  \begin{seealso}
    \refgsainst{not}{arithmetic}
    \refgsainst{eql}{conditional},
    \refgsainst{neq}{conditional},
    \refgsainst{lss}{conditional},
    \refgsainst{leq}{conditional},
    \refgsainst{geq}{conditional}
  \end{seealso}
\end{instruction}

%-----------------------------------------------------------------------------
\begin{instruction}{leq}
  \info{Instruction}{Cleq}{\multisubcl{asS1, asU1, asS2, asU2, asS4,
      asU4, asR8, asR4, asST}}{Models the $\leq$
    test.}{Cleq.Mod}{leq}{conditional}

  \begin{notes}
    \inv{The \ac{fe} guarantees that the generated \ac{ast} is
      type-correct so that mismatches between \ac{gsa} instructions
      will not occur when generating \ac{gsa} form.}
  \end{notes}

  \begin{results}
  \item ${op}_0 \leq {op}_1$
  \end{results}

  \begin{operands}
  \item \operand{Usable}{The value of the left operand from the source.}
  \item \operand{Usable}{The value of the right operand from the source.}
  \end{operands}

  \begin{seealso}
    \refgsainst{not}{arithmetic}
    \refgsainst{eql}{conditional},
    \refgsainst{neq}{conditional},
    \refgsainst{lss}{conditional},
    \refgsainst{gtr}{conditional},
    \refgsainst{geq}{conditional}
  \end{seealso}
\end{instruction}

%-----------------------------------------------------------------------------
\begin{instruction}{lss}
  \info{Instruction}{Clss}{\multisubcl{asS1, asU1, asS2, asU2, asS4,
      asU4, asR8, asR4, asST}}{Models the $<$
    test.}{Clss.Mod}{lss}{conditional}

  \begin{notes}
    \inv{The \ac{fe} guarantees that the generated \ac{ast} is
      type-correct so that mismatches between \ac{gsa} instructions
      will not occur when generating \ac{gsa} form.}
  \end{notes}

  \begin{results}
  \item ${op}_0 < {op}_1$
  \end{results}

  \begin{operands}
  \item \operand{Usable}{The value of the left operand from the source.}
  \item \operand{Usable}{The value of the right operand from the source.}
  \end{operands}

  \begin{seealso}
    \refgsainst{not}{arithmetic}
    \refgsainst{eql}{conditional},
    \refgsainst{neq}{conditional},
    \refgsainst{leq}{conditional},
    \refgsainst{gtr}{conditional},
    \refgsainst{geq}{conditional}
  \end{seealso}
\end{instruction}

%-----------------------------------------------------------------------------
\begin{instruction}{neq}
  \info{Instruction}{Cneq}{\multisubcl{asS1, asU1, asS2, asU2, asS4,
      asU4, asR8, asR4, asST}}{Models the $\neq$
    test}{Cneq.Mod}{neq}{conditional}

  \begin{notes}
    \inv{The \ac{fe} guarantees that the generated \ac{ast} is
      type-correct so that mismatches between \ac{gsa} instructions
      will not occur when generating \ac{gsa} form.}
  \end{notes}

  \begin{results}
  \item ${op}_0 \neq {op}_1$
  \end{results}

  \begin{operands}
  \item \operand{Usable}{The value of the left operand from the
      source.}
  \item \operand{Usable}{The value of the right operand from the
      source.}
  \end{operands}

  \begin{seealso}
    \refgsainst{not}{arithmetic}
    \refgsainst{eql}{conditional},
    \refgsainst{lss}{conditional},
    \refgsainst{leq}{conditional},
    \refgsainst{gtr}{conditional},
    \refgsainst{geq}{conditional}
  \end{seealso}
\end{instruction}


%%%%%%%%%%%%%%%%%%%%%%%%%%%%%%%%%%%%%%%%%%%%%%%%%%%%%%%%%%%%%%%%%%%%%%%%%%%%%%
\section{Guard}\label{class:guard}

A guard, in \ac{gsa}, is a specialization of the \ref{class:region}
whichs predicates the instructions contained in its region.  If the
guard evalutes to true, the instructions are executed, and if the
guard evalutes to false, the instructions are not executed.

\begin{table}[h!]
  \begin{tabularx}{\linewidth}{|l|X|}
    \hline Instruction & Description \\
    \hline \gsainst{true} & Code to be executed when the guarded
    condition evaluates is true. \\
    \hline \gsainst{false} & Code to be executed when the guarded
    condition evaluates is false. \\
    \hline \gsainst{casesgl} & A guard which contains code for a
    single selection of a \code{CASE} statement. \\
    \hline \gsainst{caseelse} & A guard which contains code
    corresponding to the \code{ELSE} section of an Oberon \code{CASE} statement.\\
    \hline
  \end{tabularx}
\caption{Guard Instructions}\label{tab:instruction-guard}
\end{table}

%-----------------------------------------------------------------------------
\begin{instruction}{caseelse}
  \info{Guard}{Gguard}{Gcaseelse}{Models the \code{ELSE}
  clause of a \code{CASE} statement.}{Gcase.Mod}{caseelse}{Gguard}

  \begin{notes}
    If no match for the \emph{selection} expression for a \code{CASE}
    statement is found in the \gsainst{casesgl} regions, then the
    control flow passes to the \gsainst{caseelse} region.

    The \ac{fe} ensures that all trees generated for \code{CASE}
    statements will have an \code{ELSE} clause.  In the absence of a
    user-supplied \code{ELSE} clause, the compiler will insert a
    \emph{trap} instruction which corresponds to \emph{no selection
      found}.
  \end{notes}

  \nresults

  \begin{operands}
  \item \operand{Usable}{The current value of selector used in the
      \code{CASE} statement.}
  \end{operands}

  \begin{seealso}
    \refgsainst{casereg}{Mregion}, \refgsainst{casesgl}{Gguard},
    \refgsainst{case}{Gmerge}
  \end{seealso}
\end{instruction}

%-----------------------------------------------------------------------------
\begin{instruction}{casesgl}
  \info{Guard}{Gguard}{Gcasesgl}{Models a single selection of an
    Oberon \code{CASE} statement.}{Gcase.Mod}{casesgl}{Gguard}

  \begin{notes}
    If the value of the \emph{selector} matches any of the remaining
    operands, then the code in the region will be executed.  If no
    match is found, control transfers to next \gsainst{casesgl} guard
    in the \gsainst{casereg} region.  If no match is found for all the
    \gsainst{casesgl} guards in the region, then control transfers to
    the \gsainst{caseelse} region.

    Integer \code{CASE} statements can have \emph{integer range}
    operands.
  \end{notes}

  \nresults

  \begin{operands}
  \item \operand{Usable}{The value of the selector used in the
      \code{CASE} statement.}
  \item \operand{Usable}{The remaining number of operands for this
      guard varies directly with the number elements and ranges
      specified in the source.  The elements will have the type of the
      selector of the \code{CASE} statement.}
  \end{operands}

  \begin{seealso}
    \refgsainst{casereg}{Mregion}, \refgsainst{caseelse}{Gguard},
    \refgsainst{case}{Gmerge}
  \end{seealso}
\end{instruction}

%-----------------------------------------------------------------------------
\begin{instruction}{false}
  \info{Guard}{Gguard}{Gfalse}{Models a region to be executed only
    when the guarded condition is false.}{Gfalse.Mod}{false}{Gguard}

  \begin{notes}
    If the condition specified by the operand list evaluates to
    \code{FALSE}, then the code in this region is executed.  When the
    condition evalutes to \code{TRUE}, the code is not executed.

    This region goes hand-in-hand with \refgsainst{true}{Gguard}: when
    the \gsainst{true} region is executed, the \gsainst{false} region
    is not and when the \gsainst{true} region is not executed, the
    \gsainst{false} region is.
  \end{notes}

  \nresults

  \begin{operands}
  \item \operand{Usable}{This value of the condition which guards the
      region.}
  \end{operands}

  \begin{seealso}
    \refgsainst{true}{Gguard}
  \end{seealso}
\end{instruction}

%-----------------------------------------------------------------------------
\begin{instruction}{true}
  \info{Guard}{Gguard}{Gtrue}{Models a region to be executed only when
    the guarded condition is true.}{Gtrue.Mod}{true}{Gguard}

  \begin{notes}
    When the condition specified by the operand list evaluates to
    \code{TRUE}, then the code in this region is executed.  When it
    evalutes to \code{FALSE}, the code is not executed.

    This region goes hand-in-hand with \refgsainst{false}{Gguard}: when
    the \gsainst{true} region is executed, the \gsainst{false} region
    is not and when the \gsainst{true} region is not executed, the
    \gsainst{false} region is.
  \end{notes}

  \begin{results}
  \item THere are no program-usable results produced by this
    instruction, but its result is used by \emph{merge} statements to
    merge control flow.  A guard with a value of \code{TRUE} indicates
    that the guarded region is executed.  Conversely, a guard with a
    value of \code{FALSE} indicates that the guarded region is not
    executed.
  \end{results}

  \begin{operands}
  \item \operand{Usable}{The value of the condition which guards the
      region.}
  \end{operands}

  \begin{seealso}
    \refgsainst{false}{Gguard}
  \end{seealso}
\end{instruction}


%%%%%%%%%%%%%%%%%%%%%%%%%%%%%%%%%%%%%%%%%%%%%%%%%%%%%%%%%%%%%%%%%%%%%%%%%%%%%%
\section{Hardware}\label{class:hardware}
%-----------------------------------------------------------------------------
\begin{instruction}{get}\label{hardware:get}
  \info{Instruction}{Hget}{\multisubcl{asS1, asU1, asS2, asU2, asS4,
      asU4}}{Models reading directly from
    memory.}{Hget.Mod}{get}{hardware}

  \begin{notes}
    \inv{The result of the instruction is of the size specified by the
      instruction subclass.  The \ac{fe} guarantees that the generated
      \ac{ast} is type-correct so that mismatches between \ac{gsa}
      instructions will not occur.}

    \inv{This instruction cannot be moved to a different region.}
    \inv{This instruction cannot be removed.}
  \end{notes}

  \begin{results}
  \item The result of this instruction is the value contained in the
    specified memory location.
  \end{results}

  \begin{operands}
  \item \operand{Usable}{The operand is a value which represents the
      memory address to read.}
  \end{operands}
\end{instruction}

%-----------------------------------------------------------------------------
\begin{instruction}{getreg}
  \info{Instruction}{Hgetreg}{\{asS1, asS2, asS4\}}{Models reading a
    physical cpu register.}{Hgetreg.Mod}{getreg}{hardware}

  \begin{notes}
    \inv{The result of the instruction is of the size specified by the
      instruction subclass.  The \ac{fe} guarantees that the generated
      \ac{ast} is type-correct so that mismatches between \ac{gsa}
      instructions will not occur.}

    \inv{This instruction cannot be moved to a different region.}

    \inv{This instruction cannot be removed.}
  \end{notes}

  \begin{results}
  \item The result is the value contained in the requested register.
  \end{results}

  \begin{operands}
  \item \operand{Usable}{The operand is the number of the physical cpu
      register which should be read.}
  \end{operands}
\end{instruction}

%-----------------------------------------------------------------------------
\begin{instruction}{memr}\label{hardware:memr}
  \info{Instruction}{Hmemr}{asS4}{Models reading directly from
    memory.}{Hmemr.Mod}{memr}{hardware}

  \begin{notes}
    This instruction reads memory at $\textrm{base-memory} +
    \textrm{offset}$ and returns the appropriately sized value.

    \inv{The result of the instruction is of the size specified by the
      instruction subclass.  The \ac{fe} guarantees that the generated
      \ac{ast} is type-correct so that mismatches between \ac{gsa}
      instructions will not occur.}

    \inv{This instruction cannot be moved to a different region.}

    \inv{This instruction cannot be removed.}
  \end{notes}

  \begin{results}
  \item The result of this instruction is the data at the specified
    memory location.
  \end{results}

  \begin{operands}
  \item \operand{Usable}{An integer representing the value of the base
      memory location.}
  \item \operand{Usable}{A constant offset.

      Although this value is constant, it may be contain
      \gsainst{convert} instructions.}
  \end{operands}
\end{instruction}

%-----------------------------------------------------------------------------
\begin{instruction}{memw}\label{hardware:memw}
  \info{Instruction}{Hmemw}{asS4}{Models writing directly to
    memory.}{Hmemw.Mod}{memw}{hardware}

  \begin{notes}
    This instruction writes the value specified to memory at
    $\textrm{base-memory} + \textrm{offset}$.

    \inv{The result of the instruction is of the size specified by the
      instruction subclass.  The \ac{fe} guarantees that the generated
      \ac{ast} is type-correct so that mismatches between \ac{gsa}
      instructions will not occur.}

    \inv{This instruction cannot be moved to a different region.}

    \inv{This instruction cannot be removed.}
  \end{notes}

  \nresults

  \begin{operands}
  \item \operand{Usable}{An integer representing the value of the base
      memory location.}
  \item \operand{Usable}{A constant offset.

      Although this value is constant, it may be contain
      \gsainst{convert} instructions.}
  \item \operand{Usable}{The value which should be written to memory.}
  \end{operands}
\end{instruction}

%-----------------------------------------------------------------------------
\begin{instruction}{put}\label{hardware:put}
  \info{Instruction}{Hput}{\multisubcl{asS1, asU1, asS2, asU2, asS4,
      asU4}}{Models writing directly to memory.}{Hput.Mod}{put}{hardware}

  \begin{notes}
    \inv{The result of the instruction is of the size specified by the
      instruction subclass.  The \ac{fe} guarantees that the generated
      \ac{ast} is type-correct so that mismatches between \ac{gsa}
      instructions will not occur.}

    \inv{This instruction cannot be moved to a different region.}

    \inv{This instruction cannot be removed.}
  \end{notes}

  \nresults

  \begin{operands}
  \item \operand{Usable}{The address to which data should be written.}
  \item \operand{Usable}{The value which should be written to memory.}
  \end{operands}
\end{instruction}

%-----------------------------------------------------------------------------
\begin{instruction}{putreg}
  \info{Instruction}{Hputreg}{\{asS1, asS2, asS4\}}{Models writing a
    physical cpu register.}{Hputreg.Mod}{putreg}{hardware}

  \begin{notes}
    \inv{The result of the instruction is of the size specified by the
      instruction subclass.  The \ac{fe} guarantees that the generated
      \ac{ast} is type-correct so that mismatches between \ac{gsa}
      instructions will not occur.}

    \inv{This instruction cannot be moved to a different region.}

    \inv{This instruction cannot be removed.}
  \end{notes}

  \nresults

  \begin{operands}
  \item \operand{Usable}{The number of the physical cpu register which
      should be written.}
  \item \operand{Usable}{The value to be written to the cpu register.}
  \end{operands}
\end{instruction}


%%%%%%%%%%%%%%%%%%%%%%%%%%%%%%%%%%%%%%%%%%%%%%%%%%%%%%%%%%%%%%%%%%%%%%%%%%%%%%
\section{Logical}\label{class:logical}
%-----------------------------------------------------------------------------
%-----------------------------------------------------------------------------
\begin{instruction}{asl}
  \info{Instruction}{Lasl}{asU4}{Models \emph{arithmetic shift
      left}}{Aasl.Mod}{asl}{logical}

  \begin{notes}
    Consider the following bit pattern:

    \begin{bytefield}{32}
      \bitheader[b]{0,8,16,24,31} \\
      \bitbox{8}{aaaaaaaa} \bitbox{8}{bbbbbbbb}
      \bitbox{8}{cccccccc} \bitbox{8}{dddddddd}
    \end{bytefield}

    Executing a shift left of three (3) bits would produce:

    \begin{bytefield}{32}
      \bitheader[b]{0,8,16,24,31} \\
      \bitbox{8}{aaaaabbb} \bitbox{8}{bbbbbccc}
      \bitbox{8}{cccccddd} \bitbox{8}{ddddd000}
    \end{bytefield}
  \end{notes}

  \begin{results}
  \item The result of this instruction is the value obtained after
    performing the shift.
  \end{results}

  \begin{operands}
  \item \operand{Usable}{The value to be shifted.}
  \item \operand{Usable}{The number of bits to shift.

    \inv{$\textrm{bits-to-shift} \in \{0..31\}$}}
  \end{operands}
\end{instruction}

%-----------------------------------------------------------------------------
\begin{instruction}{asr}
  \info{Instruction}{Lasr}{asU4}{Models \emph{arithmetic shift
      right}}{Aasr.Mod}{asr}{logical}

  \begin{notes}
    Consider the following bit patterns:

    \begin{bytefield}{32}
      \bitheader[b]{0,8,16,24,31} \\
      \bitbox{8}{1aaaaaaa} \bitbox{8}{bbbbbbbb} \bitbox{8}{cccccccc}
      \bitbox{8}{dddddddd} \\
      \bitbox{8}{0aaaaaaa} \bitbox{8}{bbbbbbbb} \bitbox{8}{cccccccc}
      \bitbox{8}{dddddddd}
    \end{bytefield}

    Executing a shift right of three (3) bits would produce:

    \begin{bytefield}{32}
      \bitheader[b]{0,8,16,24,31} \\
      \bitbox{8}{111aaaaa} \bitbox{8}{aaabbbbb} \bitbox{8}{bbbccccc}
      \bitbox{8}{cccddddd} \\
      \bitbox{8}{000aaaaa} \bitbox{8}{aaabbbbb} \bitbox{8}{bbbccccc} \bitbox{8}{cccddddd}
    \end{bytefield}
  \end{notes}

  \begin{results}
  \item The result of this instruction is the value obtained after
    performing the shift.
  \end{results}

  \begin{operands}
  \item \operand{Usable}{The value to be shifted.}
  \item \operand{Usable}{The number of bits to shift.

    \inv{$\textrm{bits-to-shift} \in \{0..31\}$}}
  \end{operands}
\end{instruction}

%-----------------------------------------------------------------------------
\begin{instruction}{lsl}
  \info{Instruction}{Llsl}{asU4}{Models
  \emph{logical shift left}}{Alsl.Mod}{lsl}{logical}

  \begin{notes}
    Consider the following bit pattern:

    \begin{bytefield}{32}
      \bitheader[b]{0,8,16,24,31} \\
      \bitbox{8}{aaaaaaaa} \bitbox{8}{bbbbbbbb}
      \bitbox{8}{cccccccc} \bitbox{8}{dddddddd}
    \end{bytefield}

    Executing a shift left of three (3) bits would produce:

    \begin{bytefield}{32}
      \bitheader[b]{0,8,16,24,31} \\
      \bitbox{8}{aaaaabbb} \bitbox{8}{bbbbbccc}
      \bitbox{8}{cccccddd} \bitbox{8}{ddddd000}
    \end{bytefield}

    \todo{Ensure this only works with unsigned values}
  \end{notes}

  \begin{results}
  \item The result of this instruction is the value obtained after
    performing the shift.
  \end{results}

  \begin{operands}
  \item \operand{Usable}{The value to be shifted.}
  \item \operand{Usable}{The number of bits to shift.

    \inv{$\textrm{bits-to-shift} \in \{0..31\}$}}
  \end{operands}
\end{instruction}

%-----------------------------------------------------------------------------
\begin{instruction}{lsr}
  \info{Instruction}{Llsr}{asU4}{Models \emph{logical shift
      right}}{Alsr.Mod}{lsr}{logical}

  \begin{notes}
    Consider the following bit pattern:

    \begin{bytefield}{32}
      \bitheader[b]{0,8,16,24,31} \\
      \bitbox{8}{aaaaaaaa} \bitbox{8}{bbbbbbbb}
      \bitbox{8}{cccccccc} \bitbox{8}{dddddddd}
    \end{bytefield}

    Executing a shift right of three (3) bits would produce:

    \begin{bytefield}{32}
      \bitheader[b]{0,8,16,24,31} \\
      \bitbox{8}{000aaaaa} \bitbox{8}{aaabbbbb}
      \bitbox{8}{bbbccccc} \bitbox{8}{cccddddd}
    \end{bytefield}
  \end{notes}

  \begin{results}
  \item The result is the value produced by the shift.
  \end{results}

  \begin{operands}
  \item \operand{Usable}{The value to be shifted.}
  \item \operand{Usable}{The number of bits to shift.

    \inv{$\textrm{bits-to-shift} \in \{0..31\}$}}
  \end{operands}
\end{instruction}

\begin{instruction}{rol}
  \info{Instruction}{Lrol}{asU4}{Models \emph{rotate
      left.}}{Arol.Mod}{rol}{logical}

  \begin{notes}

    This instruction shifts the value of the first operand such that
    bits shifted out of \bit{31} will appear in \bit{0}.  Consider the
    following bit pattern:

    \begin{bytefield}{32}
      \bitheader[b]{0,8,16,24,31} \\
      \bitbox{8}{aaaaaaaa} \bitbox{8}{bbbbbbbb} \bitbox{8}{cccccccc} \bitbox{8}{dddddddd}
    \end{bytefield}

    Executing a rotate left of three (3) bits would produce:

    \begin{bytefield}{32}
      \bitheader[b]{0,8,16,24,31} \\
      \bitbox{8}{aaaaabbb} \bitbox{8}{bbbbbccc} \bitbox{8}{cccccddd} \bitbox{8}{dddddaaa}
    \end{bytefield}

    \todo{Ensure this only works with unsigned values}
  \end{notes}

  \begin{results}
  \item The result of the instruction is the value obtained after
    performing the rotate.
  \end{results}

  \begin{operands}
  \item \operand{Usable}{The value to be rotated.}
  \item \operand{Usable}{The number of bits to rotate.

      \inv{$\textrm{operand} \in \{0..31\}$}}
  \end{operands}
\end{instruction}

%-----------------------------------------------------------------------------
\begin{instruction}{ror}
  \info{Instruction}{Lror}{asU4}{Models \emph{rotate
      right}}{Aror.Mod}{ror}{logical}

  \begin{notes}
    This instruction shifts the value of the first operand such that
    bits shifted out of \bit{0} will appear in \bit{31}.  Consider the
    following bit pattern:

    \begin{bytefield}{32}
      \bitheader[b]{0,8,16,24,31} \\
      \bitbox{8}{aaaaaaaa} \bitbox{8}{bbbbbbbb}
      \bitbox{8}{cccccccc} \bitbox{8}{dddddddd}
    \end{bytefield}

    Executing a rotate right of three (3) bits would produce:

    \begin{bytefield}{32}
      \bitheader[b]{0,8,16,24,31} \\
      \bitbox{8}{dddaaaaa} \bitbox{8}{aaabbbbb}
      \bitbox{8}{bbbccccc} \bitbox{8}{cccddddd}
    \end{bytefield}

    \todo{Ensure this only works with unsigned values}
  \end{notes}

  \begin{results}
  \item The result of this instruction is the value obtained after
    performing the rotate.
  \end{results}

  \begin{operands}
  \item \operand{Usable}{The value to be rotated.}
  \item \operand{Usable}{The number of bits to rotate.

    \inv{$\textrm{operand} \in \{0..31\}$}}
  \end{operands}
\end{instruction}


%%%%%%%%%%%%%%%%%%%%%%%%%%%%%%%%%%%%%%%%%%%%%%%%%%%%%%%%%%%%%%%%%%%%%%%%%%%%%%
\section{Memory Access}\label{class:memory-access}

The instructions described in this section provide structured R/W
access to memory declared in the source program.  The structure and
linking of the instructions has been designed to make it possible for
the compiler to hold values in registers during its entire \emph{live
  range}.

Table \ref{tab:instruction-memory} shows the instructions described in
the following sections.


\begin{table}[h!]
  \begin{tabularx}{\linewidth}{|l|X|}
    \hline Instruction & Description \\
    \hline \gsainst{deref} & Provides access to the heap through
    dereferencing Oberon pointers.\\
    \hline \gsainst{element} & Oberon \code{ARRAY} element access. \\
    \hline \gsainst{field} & Oberon \code{RECORD} field access. \\
    \hline \gsainst{mem} & Provides direct access to memory through
    the \code{SYSTEM} module.\footnote{Not implemented.}\\
    \hline \gsainst{nonlocal} & Models access to nonlocal variables
    (variables contained in outer lexical scopes).\\
    \hline \gsainst{varparm} & Models access to \code{VAR} parameters.\\
    \hline
  \end{tabularx}
\caption{Memory Access Instructions}\label{tab:instruction-memory}
\end{table}

%-----------------------------------------------------------------------------
\begin{instruction}{access-deref}
  \info{Instruction}{Maccess}{Mheap}{Models the dereference of
    pointer variables.}{Mheap.Mod}{access-deref}{Maccess}

  \begin{notes}
    The \gsainst{nlmctor} operand allows a consistent view of
    memory: it becomes possible to easily ensure that variables are
    written to their home memory locations when necessary.

    Since this deference deals with variables, the address returned is
    that of the first byte of program-accessible memory.  Accessing
    the type tag should be performed through
    \refgsainst{heaptag}{Mmemory}.
  \end{notes}

  \begin{results}
  \item The address of the memory referenced by the pointer variable.
  \end{results}

  \begin{dyntype}
  \item The type of the instruction will be set to the type of the
    data which was dereferenced.

    For example, dereferencing a \emph{pointer to an array} will cause
    the instruction to have an \emph{array} type.  Similarly
    dereferencing a \emph{pointer to a record} will cause the
    instruction to have a \emph{record} type.
  \end{dyntype}
    \todo{The dynamic type of each instruction must be documented.}

  \begin{operands}
  \item \operand{Usable}{The current value of the variable being accessed.}
  \item \operand{Usable}{The address of the variable to dereference.}
  \item \operand{Usable}{The current value of \gsavar{nlm}.}
  \end{operands}

  \begin{seealso}
    \refgsainst{update-deref}{Mupdate}, \refgsainst{heaptag}{Mmemory}
  \end{seealso}
\end{instruction}

%-----------------------------------------------------------------------------
\begin{instruction}{access-element}\label{memory-access:element}
  \info{Instruction}{Maccess}{Melement}{Accesses an element from an
    array; for use when the array access appears as an
    \ac{rv}.}{Melement.Mod}{access-element}{Maccess}

  \begin{notes}
    The \emph{variable} is required by several analysis steps.  The
    \emph{index} value is helpful in distinguishing between accesses
    to different elements.

    Since the \emph{address} of the index must be calculated at some
    point to provide physical access, associating tightly it with the
    instruction which will eventually require it provides for better
    code improvement opportunities.
  \end{notes}

  \begin{results}
  \item The value of the accessed element.
  \end{results}

  \begin{operands}
  \item \operand{Usable}{The current value of the variable to be
    indexed.  If this operand is an \texttt{Instruction}, it
    indicates a non-local variable has been used as the base.  If it
    is not an \texttt{Instruction}, then it is a local variable.}
  \item \operand{Usable}{The computed address of the indexed element.}
  \item \operand{Usable}{The unscaled index value.}
  \end{operands}

  \begin{seealso}
    \refgsainst{update-element}{Mupdate}
  \end{seealso}
\end{instruction}

%-----------------------------------------------------------------------------
\begin{instruction}{access-field}\label{memory-access:field}
  \info{Instruction}{Maccess}{Mfield}{Accesses a field from a
    structured variable; for use when the field reference appears as
    an \ac{rv}.}{Mfield.Mod}{access-field}{Maccess}


  \begin{notes}
    The \emph{variable} is required by analysis phases, such as
    \ac{cse}.  The \emph{offset} value is helpful in distinguishing
    between accesses to different fields.

    Since the \emph{address} of the field must be calculated at some
    point to provide physical access, associating tightly it with the
    instruction which will eventually require it provides for better
    code improvement opportunities.
  \end{notes}

  \begin{results}
  \item The value of the accessed field.
  \end{results}

  \begin{operands}
  \item \operand{Usable}{The current value of the variable to be
    referenced.  If this operand is an \texttt{Instruction}, it
    indicates a non-local variable has been used as the base.  If it
    is not an \texttt{Instruction}, then it is a local variable.}
  \item \operand{Usable}{The computed address of the referenced item.}
  \item \operand{ConstInt}{The offset, in bytes, of the item in the
      record.}
  \end{operands}

  \begin{seealso}
    \refgsainst{update-field}{Mupdate}
  \end{seealso}
\end{instruction}

%-----------------------------------------------------------------------------
\begin{instruction}{access-mem}
  \info{Instruction}{Maccess}{Mmem}{Models direct read access to
    memory through the Oberon \code{SYSTEM} module}{}{nop}{Mmem}
  \begin{notes}
    This instruction is not yet implemented.
  \end{notes}
\end{instruction}

%-----------------------------------------------------------------------------
\begin{instruction}{access-nonlocal}
  \info{Instruction}{Maccess}{Mnonlocal}{Provides read access to
    nonlocal items.}{Mnonlocal.Mod}{access-nonlocal}{Maccess}

  \begin{notes}
    Each access to a nonlocal variable will generate a
    \gsainst{nonlocal} instruction.  \ac{cse} will later find most of
    them to be redundant and eliminate most of them.

    \gsavar{nlm} is present in the operand list so that a consistent
    view of non-local memory can be used.  This enables data to be
    written to the home memory location of non-local variables at
    appropriate times.
  \end{notes}

  \begin{results}
  \item The result of this instruction is the current value of the
    accessed nonlocal variable.
  \end{results}

  \begin{operands}
  \item \operand{Usable}{The current value of the variable being
      accessed.}
  \item \operand{Usable}{The address of the variable being accessed.}
  \item \operand{Usable}{The current value of \gsavar{nlm}.}
  \end{operands}

  \begin{seealso}
    \refgsainst{update-nonlocal}{Mupdate}
  \end{seealso}
\end{instruction}

%-----------------------------------------------------------------------------
\begin{instruction}{access-varparm}
  \info{Instruction}{Maccess}{Mvarparm}{Models read access to \byref
    variables.}{Mvarparm.Mod}{access-varparm}{Maccess}

  \begin{notes}
    This instruction provides uniform access to \byref procedure
    parameters so that a consistent view of non-local memory is
    maintained throughout the compilation of a procedure body.
  \end{notes}

  \begin{results}
  \item The current value of the accessed \byref variable.
  \end{results}

  \begin{operands}
  \item \operand{Usable}{The current value of the  variable being accessed.}
  \item \operand{Usable}{The address of the variable being accessed.}
  \item \operand{The current value of \gsavar{nlm}.}
  \end{operands}

  \begin{seealso}
    \refgsainst{update-varparm}{Mupdate}
  \end{seealso}
\end{instruction}

%-----------------------------------------------------------------------------
\begin{instruction}{update-deref}
  \info{Instruction}{Mupdate}{Mheap}{Models the update of dereferenced pointer
    values.}{Mheap.Mod}{update-deref}{Mupdate}

  \begin{notes}
    The addition of \gsainst{nlmctor} as an operand allows a
    consistent view of memory: it becomes possible to easily ensure
    that variables are written to their home memory locations when
    necessary.
  \end{notes}

  \begin{results}
  \item A new value for \gsavar{nlm}.
  \end{results}

  \begin{operands}
  \item \operand{Usable}{The current value of the variable being accessed.}
  \item \operand{Usable}{The address of the variable to dereference.}
  \item \operand{Usable}{The current value of \gsavar{nlm}.}
  \item \operand{Usable}{The value which is to be written to the
      derferenced memory.}
  \end{operands}

  \begin{seealso}
    \refgsainst{access-deref}{Maccess}
  \end{seealso}
\end{instruction}


%-----------------------------------------------------------------------------
\begin{instruction}{update-element}
  \info{Instruction}{Mupdate}{Melement}{Writes a single element into
    an array; for use when the array access appears as an
    \ac{lv}.}{Melement.Mod}{update-element}{Mupdate}

  \begin{notes}
    The \emph{variable} is required by several analysis steps.  The
    \emph{index} value is helpful in distinguishing between accesses
    to different elements.

    Since the \emph{address} of the index must be calculated at some
    point to provide physical access, associating tightly it with the
    instruction which will eventually require it provides for better
    code improvement opportunities.
  \end{notes}

  \begin{results}
  \item A new copy of \gsavar{nlm}.
  \end{results}

  \begin{operands}
  \item \operand{Usable}{The current value of the variable to be
    indexed.  If this operand is an \texttt{Instruction}, it
    indicates a non-local variable has been used as the base.  If it
    is not an \texttt{Instruction}, then it is a local variable.}
  \item \operand{Usable}{The computed address of the indexed element.}
  \item \operand{Usable}{The unscaled index value.}
  \item \operand{Usable}{The value to store into the array element.}
  \end{operands}

  \begin{seealso}
    \refgsainst{access-element}{Maccess}
  \end{seealso}
\end{instruction}

%-----------------------------------------------------------------------------
\begin{instruction}{update-field}
  \info{Instruction}{Mupdate}{Mfield}{Writes a field value into a
    structured variable; for use when the field reference appears as
    an \ac{lv}.}{Mfield.Mod}{update-field}{Mupdate}

  \begin{notes}
    The \emph{variable} is required by several analysis steps.  The
    \emph{offset} value is helpful in distinguishing between accesses
    to different fields.

    Since the \emph{address} of the field must be calculated at some
    point to provide physical access, associating tightly it with the
    instruction which will eventually require it provides for better
    code improvement opportunities.
  \end{notes}

  \begin{results}
  \item A new value for \gsavar{nlm}.
  \end{results}

  \begin{operands}
  \item \operand{Usable}{The current value of the variable to be
    referenced.  If this operand is an \texttt{Instruction}, it
    indicates a non-local variable has been used as the base.  If it
    is not an \texttt{Instruction}, then it is a local variable.  This
    can be a direct variable reference, \gsainst{access} or
    \gsainst{deref} instruction.}
  \item \operand{Usable}{The computed address of the referenced item.}
  \item \operand{ConstInt}{The offset, in bytes, of the item in the
      record.}
  \item \operand{Usable}{The value to which the field should be set.}
  \end{operands}

  \begin{seealso}
    \refgsainst{access-field}{Maccess}
  \end{seealso}
\end{instruction}

%-----------------------------------------------------------------------------
\begin{instruction}{update-mem}
  \info{Instruction}{Mupdate}{Mmem}{Models direct write access to
  memory through the Oberon \code{SYSTEM} module}{}{nop}{Mmem}

  \begin{notes}
    This instruction is not yet implemented.
  \end{notes}
\end{instruction}

%-----------------------------------------------------------------------------
\begin{instruction}{update-nonlocal}
  \info{Instruction}{Mupdate}{Mnonlocal} {Provides write access to
    nonlocal items.}{Mnonlocal.Mod}{update-nonlocal}{Mupdate}


\begin{notes}
  The current value, rather than the variable, \gsavar{nlm} is used in
  the parameter list for a very specific reason: because two
  assignments to the same nonlocal memory location should not be
  considered congruent if they occur over a change to \gsavar{nlm}.

  Consider:
\begin{verbatim}
VAR
  global : INTEGER;

PROCEDURE P;
BEGIN
  global := 10;
  global := 10;
END P;
\end{verbatim}

  In this case, even with \gsavar{nlm}, these instructions would be
  considered congruent because \gsavar{nlm} is not changed between the
  two assignments.  However, consider:

\begin{verbatim}
VAR
  global : INTEGER;

PROCEDURE P(VAR x : INTEGER);
BEGIN
  global := 10;
  x := 100;
  global := 10;
END P;
\end{verbatim}

  If the two assignments to \code{global} are determined to be
  congruent, the second assignment will be superflous and removed.
  But, if \code{x} aliases \code{global}, the improved code produced
  by the compiler will not be semantically identical to the code
  generated without \acs{cia} being applied, and it will therefore be
  incorrect.\todo{Alias analysis should detect this; the congruence
    should be able to be determined.}
\end{notes}

  \begin{results}
  \item The result of this instruction is a new value for
    \gsavar{nlm}.
  \end{results}

  \begin{operands}
  \item \operand{Usable}{The current value of the variable which is to
      be accessed.}
  \item \operand{Usable}{The address of the variable being written.}
  \item \operand{Usable}{The current value of \gsavar{nlm}.}
  \item \operand{Usable}{The value being written to the variable.}
  \end{operands}

  \begin{seealso}
    \refgsainst{access-nonlocal}{Maccess}
  \end{seealso}
\end{instruction}


%-----------------------------------------------------------------------------
\begin{instruction}{update-varparm}
  \info{Instruction}{Mupdate}{Mvarparm}{Models write access to \byref
    variables.}{Mvarparm.Mod}{update-varparm}{Mupdate}

  \begin{notes}
    This instruction provides uniform access to \byref procedure
    parameters so that a consistent view of non-local memory is
    maintained throughout the compilation of a procedure body.
  \end{notes}

  \begin{results}
  \item A new value for \gsavar{nlm}.
  \end{results}

  \begin{operands}
  \item \operand{Usable}{The current value of the variable being updated.}
  \item \operand{Usable}{The address of the variable being accessed.}
  \item \operand{Usable}{The current value of \gsavar{nlm}.}
  \item \operand{Usable}{The data being written to the \byref variable.}
  \end{operands}

  \begin{seealso}
    \refgsainst{access-varparm}{Maccess}
  \end{seealso}
\end{instruction}


%%%%%%%%%%%%%%%%%%%%%%%%%%%%%%%%%%%%%%%%%%%%%%%%%%%%%%%%%%%%%%%%%%%%%%%%%%%%%%
\section{Memory Operations}\label{class:memory-operations}

\begin{table}[h!]
  \begin{tabularx}{\linewidth}{|l|X|}
    \hline Instruction & Description \\
    \hline \gsainst{adr} & Synthesizes the address of data.\\
    \hline \gsainst{arraycopy} &  Provides a byte-wise copy of a
    statically sized array onto another compatible array. \\
    \hline \gsainst{dynarrlen} & Calculates the length of a heap-based
    dynamic array. \\
    \hline \gsainst{initialize} & Models the initialization of
    structured data to binary zero (0). \\
    \hline \gsainst{heaptag} & Synthesizes the \emph{tag} value of a
    heap-based memory block. \\
    \hline \gsainst{newarray} & Models the heap allocation of a new
    statically-sized array. \\
    \hline \gsainst{newdynarray} & Models the heap allocation of a new
    dynamically-sized array. \\
    \hline \gsainst{newrecord} & Models the heap allocation of a new
    record. \\
    \hline \gsainst{nlmctor} & Creates the pseudo-variable \gsavar{nlm}.\\
    \hline \gsainst{nlmdtor} & Finalizes the pseudo-variable \gsavar{nlm}. \\
    \hline \gsainst{recordcopy} & Provides a byte-wise copy of a
    record onto another compatible record. \\
    \hline \gsainst{stringcopy} & Copies a string literal or an array
    of characters to another array of characters. \\
    \hline \gsainst{tbpadr} & Synthesizes the address of an Oberon \ac{tbp}.\\
    \hline
  \end{tabularx}
\caption{Memory Operations}\label{tab:instruction-memory-operations}
\end{table}

%-----------------------------------------------------------------------------
\begin{instruction}{adr}
  \info{Instruction}{Mmemory}{Madr}{Models the address of its
    operand.}{Madr.Mod}{adr}{Mmemory}

  \begin{notes}
    This instruction can be generated by using the \code{SYSTEM.ADR}
    function, however it is most often generated by the compiler
    rather than directly from source.

    This instruction does not synthesize the address of a \ac{tbp}.
  \end{notes}

  \todo{Results should have their physical attributes denoted.  For
    example, this instruction produces a signed 4-byte value.}
  \begin{results}
  \item The physical address of the operand.
  \end{results}

  \begin{operands}
  \item \operand{Addressable}{The addressable item for which the
      address should be calculated.}
  \end{operands}

  \begin{seealso}
    \refgsainst{tbpadr}{Mmemory}, \refgsainst{heaptag}{Mmemory},
    \refgsainst{mayalias}{Palias}
  \end{seealso}
\end{instruction}

%-----------------------------------------------------------------------------
\begin{instruction}{arraycopy}\label{memory:arraycopy}
  \info{Instruction}{Mmemory}{Marraycopy}{Models the
    copying of a source array to a destination
    array.}{Marraycopy.Mod}{arraycopy}{Mmemory}

  \begin{notes}
    The \ac{fe} ensures that the arrays are of the same size and are
    assignment compatible.

    When presented a \code{RECORD} with no fields, the \ac{fe} will
    create \code{RECORD} and \code{ARRAY} types which are zero (0)
    bytes in size; in this case the parameter indicating the number of
    bytes in the destination will, of course, be zero (0).

    There are two situations which cause this instruction to be
    generated:
    \begin{enumerate}
    \item To copy \byval parameters into a local temporary.
    \item To handle source-level array assignments.
    \end{enumerate}
  \end{notes}

  \begin{results}
  \item The result is a copy of the source.  The current value of the
    destination will be assigned the copy of the source.
  \end{results}

  \begin{operands}
  \item \operand{Usable}{The current value of the destination
      variable.}
  \item \operand{Usable}{The address of the destination.}
  \item \operand{ConstInt}{\code{LEN(dest)}.}
  \item \operand{Usable}{The current value of the source array.}
  \item \operand{Usable}{The address of the current value of the
      source array.}
  \end{operands}

  \begin{seealso}
    \refgsainst{stringcopy}{Mmemory}, \refgsainst{recordcopy}{Mmemory}
  \end{seealso}
\end{instruction}

%-----------------------------------------------------------------------------
\begin{instruction}{dynarrlen}
  \todo{Must indicate the size of the result for all instructions}
  \info{Instruction}{Mmemory}{Mdynarrlen}{Models the calculated length
    of a source-level heap-based open
    array}{Mdynarrlen.Mod}{dynarrlen}{Mmemory}

  \begin{notes}

    In terms of the generation of code, there are three different
    types of array length requests in the Oberon language:

    \begin{itemize}
    \item Statically-sized

      The length of the statically-sized array is known at compile
      time, and therefore requires no code to be generated.

    \item Open array parameter

      The open array parameter provides access to the length of each
      dimension through the result list of \gsainst{enter} (see
      \refgsainst{enter}{Mregion}.

    \item Heap-based open array

      A heap-based open array, when created with

      $$\code{NEW}(\textrm{array}, \textrm{dim}_0, \textrm{dim}_1,
      \ldots, \textrm{dim}_{n-1})$$

      is allocated space on the heap and the allocator stores the
      specified dimension lengths into the \emph{array block preamble}
      which is part of the allocated memory (see \ac{rmld}).
      Consequently, it becomes necessary to use this instruction to
      access those array lengths.
    \end{itemize}
  \end{notes}

  \begin{results}
  \item The result is a \code{LONGINT} representing the number of
    elements contained in the array dimension.
  \end{results}

\todo{\gsainst{newdynarray} and its kin return the address of the
  array block.}
  \begin{operands}
  \item \operand{Usable}{The address of the array block in the heap.}
  \item \operand{ConstInt}{The dimension for which the length should
      be calculated.}
  \item \operand{ConstInt}{The offset into the heap block of the array
      length for the specified dimension.}

    This offset is calculated based on the static type of the array;
    it can be determined by examining the \ac{rmld} document.
  \end{operands}
\end{instruction}

%-----------------------------------------------------------------------------
\begin{instruction}{heaptag}
  \info{Instruction}{Mmemory}{Mheaptag}{Models the address of a
    heap-based \ac{td}.}{Mheaptag.Mod}{heaptag}{Mmemory}

  \begin{notes}
    \gsainst{heaptag} is used for accessing \ac{tbp} addresses and for
    performing runtime type checks.  Because the value synthesized by
    this instruction resides in read-only memory, it is not necessary
    to include \gsavar{nlm} as an operand.

    It is not necessary to have a special instruction to access the
    tag of static records because the \ac{td} symbol can be determined
    directly from the static type of the variable.  It is also not
    necessary to have a special instruction for obtaining the \ac{td}
    of a \byref record parameter since the \ac{td} can be obtained
    directly from the result list of \gsainst{enter}.
  \end{notes}

  \begin{results}
  \item The address of a \ac{td} is the result.
  \end{results}

  \begin{operands}
  \item \operand{Usable}{The current value of the variable for which
      the \ac{td} is to be obtained.}
  \item \operand{Usable}{A \gsainst{deref} which dereferences the
      desired variable.}
  \end{operands}
\end{instruction}

%-----------------------------------------------------------------------------
\begin{instruction}{initialize}
  \info{Instruction}{Mmemory}{Minitialize}{Models the initialization
    of structured variables to binary zero
    (0).}{Minitialize.Mod}{initialize}{Mmemory}

  \begin{notes}
    The code generator for this procedure will use the \emph{type}
    field of the instruction to determine how many bytes need
    initialization, and if the code should be emitted inline or if a
    helper function should be called.

    When presented a \code{RECORD} with no fields, the \ac{fe} will
    create \code{RECORD} and \code{ARRAY} types which are zero (0)
    bytes in size; in this case the parameter indicating the number of
    bytes to intialize will, of course, be zero (0).

    This initializes data to binary zero (0); this may not be
    \emph{zero} as reflected by the type of the data\footnote{It could
      be possible to have a floating point value for zero which does
      not correspond to binary 0.  \texttt{NIL} may also not
      correspond to binary zero.}.  A properly functioning program
    should initialize all data to a known value before use.

    Initializing pointers in records and arrays to the runtime value
    of \texttt{NIL} is not supported by this instruction\footnote{In
      fact, it would require another instruction, additional
      information in the compiler-generated \ac{td} describing the
      fields, and runtime support which will traverse the \ac{td}
      information to initialize the data.}.
  \end{notes}

  \begin{results}
  \item The result is the new current value of the initialized
    variable.
  \end{results}

  \begin{operands}
  \item \operand{Usable}{The current value of the variable which is to
      be initialized.}
  \item \operand{Usable}{The address of the variable which is to be
      initialized.}
  \item \operand{ConstInt}{The number of bytes to be initialized.}
  \end{operands}

  \begin{seealso}
    \refgsainst{initrec}{Gtypedesc}, \refgsainst{initarr}{Gtypedesc},
    \refgsainst{initdarr}{Gtypedesc}
  \end{seealso}

\end{instruction}

%-----------------------------------------------------------------------------
\begin{instruction}{newarray}
  \info{Instruction}{Mmemory}{Mnewarray}{Models the creation of a new
    statically-sized array on the
    heap.}{Mnewarray.Mod}{newarray}{Mmemory}

  \begin{notes}
    This instruction will always return a unique value.  \todo{This
      instruction does not take the address of the pointer being
      allocated.  This requires a change in the allocators in
      \code{Kernel}.}

    When presented a \code{RECORD} with no fields, the \ac{fe} will
    create \code{RECORD} and \code{ARRAY} types which are zero (0)
    bytes in size.  The memory allocator must ensure that each
    allocation request for zero (0) bytes will return a unique
    value\footnote{It is possible for an allocation to return a value
      that was previously returned, but this will only occur after the
      \ac{gc} has reclaimed the memory.}  In practice, this is not a
    problem since each memory allocation request\footnote{Except for
      \code{SYSTEM.NEW}} must include space for a pointer to the block
    descriptor; as a consequence of this, each allocation will be
    rounded up to the nearest \emph{heap block size}.
  \end{notes}

  \begin{results}
  \item The result is the address of the newly allocated array.
  \end{results}

  \begin{operands}
  \item \operand{Usable}{The address of the \ac{td} for the array.}
  \end{operands}
\end{instruction}

%-----------------------------------------------------------------------------
\begin{instruction}{newdynarray}
  \info{Instruction}{Mmemory}{Mnewdynarray}{Models the allocation of a
    new dynamically-sized array on the
    heap.}{Mnewdynarray.Mod}{newdynarray}{Mmemory}

  \begin{notes}
    This instruction will always return a unique value.  \todo{This
      instruction does not take the address of the pointer being
      allocated.  This requires a change in the allocators in
      \code{Kernel}.}

    When presented a \code{RECORD} with no fields, the \ac{fe} will
    create \code{RECORD} and \code{ARRAY} types which are zero (0)
    bytes in size.  The memory allocator must ensure that each
    allocation request for zero (0) bytes will return a unique
    value\footnote{It is possible for an allocation to return a value
      that was previously returned, but this will only occur after the
      \ac{gc} has reclaimed the memory.}  In practice, this is not a
    problem since each memory allocation request\footnote{Except for
      \code{SYSTEM.NEW}} must include space for a pointer to the block
    descriptor; as a consequence of this, each allocation will be
    rounded up to the nearest \emph{heap block size}.
  \end{notes}

  \begin{results}
  \item The result is the address of the newly allocated dynamic
    array.
  \end{results}

  \begin{operands}
  \item \operand{Usable}{The address of the \ac{td} for the dynamic
      array.}
  \item \operand{ConstInt}{A length value for each dimension of the
      array.}
  \end{operands}
\end{instruction}

%-----------------------------------------------------------------------------
\begin{instruction}{newrecord}
  \info{Instruction}{Mmemory}{Mnewrecord}{Models the allocation of a
    new record on the heap}{Mnewrecord.Mod}{newrecord}{Mmemory}

  \begin{notes}
    This instruction will always return a unique value.  \todo{This
      instruction should be changed to not take an address; that would
      allow more efficient use of local variables for allocation,
      since their address would not need to be taken.  However, this
      will require a change in the allocators in \code{Kernel}.}

    When presented a \code{RECORD} with no fields, the \ac{fe} will
    create \code{RECORD} and \code{ARRAY} types which are zero (0)
    bytes in size.  The memory allocator must ensure that each
    allocation request for zero (0) bytes will return a unique
    value\footnote{It is possible for an allocation to return a value
      that was previously returned, but this will only occur after the
      \ac{gc} has reclaimed the memory.}  In practice, this is not a
    problem since each memory allocation request\footnote{Except for
      \code{SYSTEM.NEW}} must include space for a pointer to the block
    descriptor; as a consequence of this, each allocation will be
    rounded up to the nearest \emph{heap block size}.
  \end{notes}

  \begin{results}
  \item The result is the address of the newly allocated record.
  \end{results}

  \begin{operands}
  \item \operand{Usable}{The address of the \ac{td} for the record.}
  \end{operands}
\end{instruction}

%-----------------------------------------------------------------------------
\begin{instruction}{nlmctor}
  \info{Instruction}{Mmemory}{Mnlmctor}{Models read and write
    access to non-local
    variables.}{Mnlmctor.Mod}{nlmctor}{Mmemory}

  \begin{notes}
    \gsavar{nlm} models \emph{nonlocal variables}, \emph{\byref
      parameters} and \emph{heap-based objects}.  This instruction,
    coupled with \refgsainst{nlmdtor}{Mmemory} ensures that all
    such items are written back to their home memory location at the
    appropriate times.

    Every \ac{greg} will have a \gsainst{nlmctor} instruction as
    part of the \gsainst{enter} region.
  \end{notes}

  \begin{results}
  \item The current value of the the pseudo-variable
    \gsavar{nlm}.
  \end{results}

  \noperands

  \begin{seealso}
    \refgsainst{nlmdtor}{Mmemory}, \refgsavar{nlm},
    \refgsainst{enter}{Mregion}
  \end{seealso}
\end{instruction}

%-----------------------------------------------------------------------------
\begin{instruction}{nlmdtor}
  \info{Instruction}{Mmemory}{Mnlmdtor}{Ensures that all non-local
    variables are written back to their home memory
   location}{Mnlmdtor.Mod}{nlmdtor}{Mmemory}

  \begin{notes}
    \todo{The semantics of this instruction have not yet been
      implemented.}

    In conjunction with \refgsainst{nlmctor}{Mmemory}, this
    instruction ensures that non-local variables are written back to
    their home memory location at appropriate times.

    Every \ac{greg} will have a \gsainst{nlmdtor} in the
    \gsainst{exit} region; each such instruction will be marked as
    \emph{always live}.

  \end{notes}

  \begin{results}
  \item A consistent view of \gsavar{nlm} is implied by this
    instruction.
  \end{results}

  \noperands

  \begin{seealso}
    \refgsainst{nlmctor}{Mmemory}, \refgsavar{nlm},
    \refgsainst{exit}{Mregion}
  \end{seealso}
\end{instruction}

%-----------------------------------------------------------------------------
\begin{instruction}{recordcopy}\label{memory:recordcopy}
  \info{Instruction}{Mmemory}{Mrecordcopy}{Models the assignment of
    one record to another
    record}{Mrecordcopy.Mod}{recordcopy}{Mmemory}

  \begin{notes}
    The \ac{fe} ensures that the source and destination are assignment
    compatible with static type tests and with dynamically-typed
    runtime tests generated into the \ac{gsa} \ac{ir}.

    If the control flow of the program reaches a \gsainst{recordcopy},
    the types are entirely compatible.

    When presented a \code{RECORD} with no fields, the \ac{fe} will
    create \code{RECORD} and \code{ARRAY} types which are zero (0)
    bytes in size; in this case the parameter indicating the number of
    bytes in the destination will, of course, be zero (0).

    There are two situations which cause this instruction to be
    generated:
    \begin{enumerate}
    \item To copy \byval parameters into a local temporary.
    \item To handle source-level record assignments.
    \end{enumerate}
  \end{notes}

  \begin{results}
  \item The result is a copy of the, truncated to fit the size of the
    destination.  The copy will be assigned to the destination variable.
  \end{results}

  \begin{operands}
  \item \operand{Usable}{The current value of the destination
      variable.}
  \item \operand{Usable}{The address of the destination.}
  \item \operand{ConstInt}{The number of bytes available in the
      destination variable.}
  \item \operand{Usable}{The current value of the source.}
  \item \operand{Usable}{The address of the source.}
  \end{operands}

  \begin{seealso}
    \refgsainst{stringcopy}{Mmemory}, \refgsainst{arraycopy}{Mmemory}
  \end{seealso}
\end{instruction}

%-----------------------------------------------------------------------------
\begin{instruction}{stringcopy}
  \info{Instruction}{Mmemory}{Mstringcopy}{Models the copy of an
    ASCIIZ array of characters, or a string constant, to an array of
    characters.}{Mstringcopy}{stringcopy}{Mmemory}

  \begin{notes}
    \gsainst{stringcopy} is only generated when the standard procedure
    \code{COPY} is used.

    \gsainst{stringcopy} is a high level enough instruction that it
    can be implemented with hardware constructs such as string
    instructions, or small loops; the actual selection of machine
    instructions will be performed when the \ac{ir} is lowered to the
    \ac{mr}.
  \end{notes}

  \begin{results}
  \item The result is a copy of the source.  The current value of the
    destination will be assigned the copy of the source.
  \end{results}

  \begin{operands}
  \item \operand{Usable}{The current value of the destination
      variable.}
  \item \operand{Usable}{The address of the destination.}
  \item \operand{ConstInt}{\code{LEN(dest)}.}
  \item \operand{Usable}{The current value of the source.}
  \item \operand{Usable}{The address of the source.}
  \end{operands}

  \begin{seealso}
    \refgsainst{arraycopy}{Mmemory}, \refgsainst{recordcopy}{Mmemory}
  \end{seealso}
\end{instruction}

%-----------------------------------------------------------------------------
\begin{instruction}{tbpadr}
  \info{Instruction}{Mmemory}{Mtbpadr}{Models the address of a
    \ac{tbp}.}{Mtbpadr.Mod}{tbpadr}{Mmemory}

  \begin{notes}
    This static type of the self symbol parameter prevents the taking
    of the address of two different methods with identical names, from
    two different type hierarchies extended from the same base, and
    having the same \ac{mt} index from being considered
    congruent\note{Consider two type hierarchies with the
      characterstics stated above, then write this instruction without
      this operand and notice that they would be considered
      congruent}.

    Consider:

\begin{verbatim}
TYPE
  Base = POINTER TO BaseDesc;
  BaseDesc = RECORD END;

  Ext0 = POINTER TO Ext0Desc;
  Ext0Desc = RECORD (BaseDesc) END;
  Ext1 = POINTER TO Ext1Desc;
  Ext1Desc = RECORD (BaseDesc) END;

PROCEDURE (x : Ext0) Method(y : INTEGER);
BEGIN END Method;

PROCEDURE (x : Ext1) Method(y : INTEGER);
BEGIN END Method;

PROCEDURE NotCongruent;
  VAR b : Base;
BEGIN
  b{Ext0}.Method(0);
  b{Ext1}.Method(0);
END NotCongruent;
\end{verbatim}

    Without the addition of the static type of the self parameter, the
    two invocations of \code{Method} would be considered congruent.
  \end{notes}

  \begin{results}
  \item The result of this instruction is the address of the specified
    \ac{tbp}.
  \end{results}

  \todo{Ensure the fact that the static type descriptor symbol and the
    output of \gsainst{heaptag} are identical in layout.}

  \begin{operands}
  \item \operand{Usable}{The address of the \ac{td} for record which
      owns the specified method.

      This instruction will be synthesized from either
      \refgsainst{heaptag}{Mmemory}, or, in the case of statically
      typed method references, by a direct reference to the address of
      the \ac{td} symbol.}

  \item The static type of the \emph{self} symbol being referenced.

  \item The global symbol of the method; the compiler-generated symbol
    which is used for relocations involving the method.  This can be
    used to transform some method invocations into direct procedure
    calls\todo{2001.08.03: The Oberon compiler does not generate
      global symbols for methods; as of now, the symbol is only the
      method symbol}.

  \item The ordinal number of the method.
  \item The physical offset, in bytes, from the beginning of the
    \ac{mt} for the method.
  \end{operands}

  \begin{seealso}
    \refgsainst{adr}{Mmemory}
  \end{seealso}
\end{instruction}



%%%%%%%%%%%%%%%%%%%%%%%%%%%%%%%%%%%%%%%%%%%%%%%%%%%%%%%%%%%%%%%%%%%%%%%%%%%%%%
\section{Merge}\label{class:merge}

A merge in \ac{gsa} form is used to join a control flow which has been
split using guards (see \S\ref{class:guard}).  Essentially, a merge is
a conjunctive-or of the output of several different regions.  Typically
a merge will contain only \ac{phi} nodes.  However, since
\gsainst{merge-cond} coalesces all the false regions of conjunctive and
disjunctive control into a single region, it contains actual code.

\begin{table}[h!]
  \begin{tabularx}{\linewidth}{|l|X|}
    \hline Instruction & Description \\
    \hline \gsainst{merge-case} & A conditional guard for modeling the
    closing of \code{CASE} statements. \\
    \hline \gsainst{merge-cond} & A conditional guard which models the
    closing of complex boolean expressions involving the
    conditional operators \code{\&} and \code{OR}. \\
    \hline \gsainst{merge-if} & A conditional guard which models the closing
    of an Oberon \code{IF} statement. \\
    \hline \gsainst{merge-loop} & A conditional guard for modeling the
    closing of looping constructs. \\
    \hline
  \end{tabularx}
\caption{Merge Instructions}\label{tab:instruction-merge}
\end{table}

\begin{instruction}{merge-case}
  \info{Merge}{Gmerge}{Gcase}{Models the closing of a \code{CASE}
    statement.}{Gcase.Mod}{merge-case}{Gmerge}

  \begin{notes}
    This region merges the control flow which was split over the
    various conditions of the \code{CASE} statement.  For each
    selection criteria in the \code{CASE} body, there will be an
    operand of this merge.
  \end{notes}

  \nresults

  \begin{operands}
  \item \operand{Case}{The region which controls the whole \code{CASE}
      statement.}
  \item \operand{Usable}{$\gsainst{casesgl}_0$} Initially this will be
    a \gsainst{casesgl}, but \ac{ccp} can replace it with a constant
    value of \code{TRUE} or \code{FALSE}.
  \item \operand{Usable}{$\gsainst{casesgl}_1$} Initially this will be
    a \gsainst{casesgl}, but \ac{ccp} can replace it with a constant
    value of \code{TRUE} or \code{FALSE}.
  \item \operand{Usable}{$\vdots$}
  \item \operand{Usable}{$\gsainst{casesgl}_{n-1}$} Initially this
    will be a \gsainst{casesgl}, but \ac{ccp} can replace it with a
    constant value of \code{TRUE} or \code{FALSE}.
  \item \operand{Usable}{$\gsainst{else}$} Initially this will be an
    \gsainst{caseelse}, but \ac{ccp} can replace it with a constant
    value of \code{TRUE} or \code{FALSE}.
  \end{operands}

  \begin{seealso}
    \refgsainst{casereg}{Mregion}, \refgsainst{caseelse}{Gguard},
    \refgsainst{casesgl}{Gguard}
  \end{seealso}
\end{instruction}


%-----------------------------------------------------------------------------
\begin{instruction}{merge-cond}
  \info{CondMerge}{Gmerge}{Gcond}{Models a control flow merge
  from a code path which failed a logical (\code{\&} and \code{OR})
  operation}{Gif.Mod}{merge-cond}{Gmerge}

  \begin{notes}
    The conditional merge region is used to ensure that the control
    flow nicely joins in the case of conjunctive and disjunctive
    logical tests, for example:

\begin{verbatim}
IF a & b THEN (* do something *)
ELSE (* do something else *)
END;
\end{verbatim}

    In this example, the conditional merge region will contain the
    code corresponding to \code{(* do something else *)}.  In this
    sense, the \gsainst{merge-cond} is different than other merge
    instructions: it merges the control flow for all false regions of
    a complex conditional test into a single region.
  \end{notes}

  \nresults

  \begin{operands}
  \item \operand{Usable}{A boolean value, when \code{TRUE}, indicates
      that the left expression of the boolean operator failed.}
  \item \operand{Usable}{boolean value, when \code{TRUE}, indicates
      that the right expression of the boolean operator failed.}
  \end{operands}

  \begin{seealso}
    \refgsainst{if}{Gif}
  \end{seealso}
\end{instruction}

%-----------------------------------------------------------------------------
\begin{instruction}{merge-if}
  \info{Merge}{Gmerge}{Gif}{Models the merging of the control regions
    for the \code{THEN} and \code{ELSE} sections of an \code{IF}
    statement.}{Gif.Mod}{merge-if}{Gif}

  \begin{notes}
    This instruction merges control flow which was conditionally split
    into only two (2) paths - a \refgsainst{true}{Gguard} and a
    \refgsainst{false}{Gguard}, and with conjunctive and disjunctive
    conditional, an additional \refgsainst{merge-cond}{Gmerge} can be
    present.

    Instructions placed into this region are executed through all
    control paths merged by this instruction.

    \code{ELSIF} constructs are modeled via nested \code{IF}
    statements in the \code{ELSE} region of the corresponding
    \code{IF}.
  \end{notes}

  \nresults

  \begin{operands}
  \item \operand{Usable}{A boolean value which indicates the execution
      state of the \code{THEN} region of the associated \code{IF}:
      when it evaluates to \code{TRUE}, the \code{THEN} region has
      been executed and the \code{ELSE} region has not been executed.}
  \item \operand{Usable}{A boolean value which indicates the execution
      state of the \code{ELSE} region of the associated \code{IF}:
      when it evaluates to \code{TRUE}, the \code{ELSE} region has
      been executed and the \code{THEN} region has not been executed.}
  \end{operands}

  \begin{seealso}
    \refgsainst{merge-cond}{Gmerge}, \refgsainst{true}{Gguard},
    \refgsainst{false}{Gguard}
  \end{seealso}
\end{instruction}

%-----------------------------------------------------------------------------
\begin{instruction}{merge-loop}
  \info{Merge}{Gmerge}{Gloop}{Models the top and backedges of the
    looping construct.}{Gloop.Mod}{merge-loop}{Gmerge}

  \begin{notes}
    This compiler implements a single type of loop, one in which the
    body will be executed at least once (\code{REPEAT}-\code{UNTIL},
    \code{LOOP}).  Implmentation of \code{WHILE} loops require a
    translation into an equivalent \code{REPEAT}-\code{UNTIL} looping
    construct.  For example,

\begin{verbatim}
  WHILE condition DO <body> END;
\end{verbatim}

    will be translated into the equivalent

\begin{verbatim}
  IF condition THEN
    REPEAT <body> UNTIL ~condition;
  END;
\end{verbatim}

    This translation is done so that loop-invariant code for
    \code{WHILE} statements can be hoisted out of the loop to the
    containing region (the \code{IF} condition region); without the
    translation there is no guard blocking execution of hoisted
    instructions, and the semantic equivalence of the translated loop
    will not be the same as the original.
  \end{notes}

  \nresults

  \begin{operands}
  \item \operand{Usable}{This operand is a boolean value which
      controls entry into the loop.  Semantically, when \code{TRUE},
      the loop will be entered, when \code{FALSE}, the loop will not
      be entered.  However, due to the invariant that all loops are
      executed at least once, this parameter always evaluates to the
      constant \code{TRUE}.}
  \item \operand{Usable}{This operand is a boolean value which
      controls the iterative property of the loop.  When \code{TRUE},
      the loop will continue iterating, when \code{FALSE}, the loop
      construct will exit.}
  \end{operands}
\end{instruction}

%%%%%%%%%%%%%%%%%%%%%%%%%%%%%%%%%%%%%%%%%%%%%%%%%%%%%%%%%%%%%%%%%%%%%%%%%%%%%%
\section{Miscellaneous}\label{class:misc}
\begin{table}[h!]
  \begin{tabularx}{\linewidth}{|l|X|}
    \hline Instruction & Description \\
    \hline \gsainst{call} & Procedure invocation. \\
    \hline \gsainst{copy} & Enables the use of a \emph{copy} of data. \\
    \hline \gsainst{cap} & Convert character to uppercase. \\
    \hline \gsainst{gate} & Coalesces the value of a variable set
    through different control paths in the program. \\
    \hline \gsainst{import} & Import module. \\
    \hline
  \end{tabularx}
\caption{Miscellaneous Instructions}\label{tab:instruction-misc}
\end{table}
%-----------------------------------------------------------------------------
\begin{instruction}{call}
  \info{Instruction}{Gmisc}{Gcall}{Models the invocation of a
    procedure or a \ac{tbp}}{Gcall.Mod}{call}{Gmisc}

  \begin{results}
  \item The results of this instruction are modeled by the
    \emph{result list} of the instruction.
  \end{results}

  \todo{The results of the call instruction are dependent upon several
    things: 1) the declared/defined result type, 2) the global
    variables which are known to be modified and 3) by-reference
    parameters.  As of 2002.10.07, it's unclear to me exactly how the
    value of var parameters should be correctly modeled.  oo2c seems
    to model them by assigning them to a local varible, and updating
    that local variable after the call which takes the by-reference
    parameter (for scalar values).  I'm inclined to create a result
    for each var paremter and assign it to the parameter -- taking the
    pessimistic assumption that the parameter is updated.  However, if
    the value is known not to be updated, then this is unnecessary.}

    \voperands{The number of operands
      of a \gsainst{call} varies directly with the number of
      \emph{formal parameters} declared in the source.  For each
      operand provided to the callee, there are two semantic classes
      which must be handled:

    \todo{Better description of all types of parameters: open array,
      record, integral, \byval, \byref needed.}
    \begin{description}
    \item[by-value] The current value of the expression is used.
    \item[by-reference] The address of the parameter is passed, followed
      immediately by the current value of the parameter.

      If the current value of the parameter is not used in
      \gsainst{call}, the algorithms may determine that an assignment
      to that variable is not used and subsequently delete it.  Or,
      they may move the assignment to follow the procedure invocation.
    \end{description}
      }
\end{instruction}

%-----------------------------------------------------------------------------
\begin{instruction}{cap}
  \info{Instruction}{Gmisc}{Gcap}{Models the conversion of
  characters to their ASCII uppercase equivalent.}{Gcap.Mod}{cap}{Gmisc}

  \begin{results}
  \item
    $
    \left.
      \begin{array}{r}
        \{\textrm{'a'..'z'}\} \\
        -\{\textrm{'a'..'z'}\}
      \end{array}
    \right\}
    \mapsto
    \left\{
      \begin{array}{r}
        \{\textrm{'A'..'Z'}\} \\
        -\{\textrm{'A'..'Z'}\}
      \end{array}
    \right.
    $

    The result is the uppercase ASCII value of the character.  The
    result of using this on a non-lowercase character is undefined.
  \end{results}

  \begin{operands}
  \item \operand{Usable}{The character value to capitalize.}
  \end{operands}
\end{instruction}

%-----------------------------------------------------------------------------
\begin{instruction}{copy}
  \info{Instruction}{Gmisc}{Gcopy}{Models the use of a \emph{copy} of
    the operand.}{Gcopy.Mod}{copy}{Gmisc}

  \begin{notes}
    The \gsainst{copy} is only used during the translation to \ac{gsa}
    phase, and the \ac{cia} \ac{cp} will remove all these copies.
  \end{notes}

  \begin{results}
  \item The result is a copy of the input operand.
  \end{results}

  \begin{operands}
  \item \operand{Usable}{The value to be copied.}
  \end{operands}
\end{instruction}

%-----------------------------------------------------------------------------
\begin{instruction}{gate}
  \info{Instruction}{Gmisc}{Ggate}{Models \ac{gsa} \ac{phi} nodes in
    the \ac{ir}.}{Ggate.Mod}{gate}{Gmisc}

  \begin{notes}
    The number of \emph{selection} operands is dependent upon the type
    of merge region controlling the gate.  Table
    \ref{tab:gate-operands} shows the relationship of the controlling
    merge region to the number of selection operands.

    \begin{table}[h!]
      \begin{tabularx}{\linewidth}{|l|X|}
        \hline Merge Region & Operand Information \\

        \hline \gsainst{merge-if} & The second and third operands
        correspond to the \emph{true} and \emph{false} control paths
        of the controlling conditional expression.\\

        \hline \gsainst{merge-cond} & A gate will never be controlled
        by a condition merge.  This is because a condition merge
        region splits the control path for a compound conditional of
        an \code{IF} statement.  Since the control path is not
        merging, there is no possibility of a gate. \\

        \hline \gsainst{merge-loop} & A gate will never be controlled
        by a loop merge. This is because a loop merge region simply
        models the entry point of the loop (the loop top and the
        backedge) for a looping construct.  Since the control path is
        not merging, there is no possibility of a gate. \\

        \hline \gsainst{merge-case} & A gate controlled by this type
        of merge will have an operand for each selection block of the
        \code{CASE} statement. \\

        \hline
      \end{tabularx}
      \caption{Gate Operands}\label{tab:gate-operands}
    \end{table}

    During GSA construction the operands of a gate controlled by an if
    are as follows:

    inv: operand-1 IS Instruction

    inv: operand-2 IS Instruction

    The reason for this is that during if-gate creation the compiler
    needs to know the Region of the operand and some things from the
    Usable hierarchy do not have a 'region' field.

    The invariant that the second and third operands are instructions
    can be enforced by having the compiler introduce 'copy'
    instructions for the cases where constants and variables are
    assigned directly (i.e.  things which are not expressions) to
    other variables.

    This restriction only applies when the GSA is being created.
    After the entire structure is created, copy-propogation is free to
    replace these instructions with direct variable references.

    After the gate is constructed, it should be noted that nothing
    prohibits the operands of a \gsainst{gate} from being the result
    of a \gsainst{gate}.  This can have implications on the \ac{cia},
    \ac{ccp} for example.

  \end{notes}

  \begin{results}
  \item The result is the new value for the variable, coalesced over
    the output from several control paths.
  \end{results}

  \begin{operands}
  \item \operand{Merge}{The merge which controls the gate.}
  \item \operand{Usable}{An operand for each region controlled by the
      merge.}
  \end{operands}
\end{instruction}

%-----------------------------------------------------------------------------
\begin{instruction}{import}
  \info{Instruction}{Gmisc}{Gimport}{Models the invocation of the
    initialization code of an imported
    module.}{Gimport.Mod}{import}{Gmisc}

  \nresults

  \begin{operands}
  \item \operand{Symbol}{The module which should be initialized.}
  \end{operands}
\end{instruction}


%%%%%%%%%%%%%%%%%%%%%%%%%%%%%%%%%%%%%%%%%%%%%%%%%%%%%%%%%%%%%%%%%%%%%%%%%%%%%%
\section{Pseudo}\label{class:pseudo}

The instructions described in this section provide...

Table \ref{tab:pseudo} shows the instructions described in the
following sections.

\begin{table}[h!]
  \begin{tabularx}{\linewidth}{|l|X|}
    \hline Instruction & Description \\
    \hline \gsainst{noalias} & Unused. Models when there is no
    aliasing found between non-local memory accesses. \\
    \hline \gsainst{mayalias} &  Models variable aliasing in non-local
    memory. \\
    \hline \gsainst{doesalias} & Unused.  Models when two variables
    are known to alias the same memory.  \\
    \hline
  \end{tabularx}
\caption{Pseudo Instructions}\label{tab:pseudo}
\end{table}


%-----------------------------------------------------------------------------
\begin{instruction}{mayalias}
  \info{Pseudo Instruction}{Palias}{Pmayalias}{Models memory aliasing}{Pmayalias.Mod}{mayalias}{Palias}

  \begin{notes}
    \gsainst{mayalias} models memory aliasing by replacing the
    \emph{address} operand of instructions that access non-local
    memory.
  \end{notes}

  \begin{results}

  \item The instruction is a single result.

    The result is to be treated as the same as the result of the first
    operand -- a computed memory address.

    The result instructs the compiler to never enregister data read
    from, or written to, this address.
  \end{results}

  \begin{operands}
  \item The first operand is a computed memory address that may
    alias to the computed memory addess in the second operand.

    The instruction using the result of this \gsainst{mayalias} will
    have referred to the value of this operand before \emph{alias
      analysis} was executed.

  \item The second operand is a computed memory address that may
    alias to the computed memory addess in the first operand.
  \end{operands}

  \begin{seealso}
    \refgsainst{adr}{Mmemory},
    \refgsainst{access-nonlocal}{Maccess},
    \refgsainst{access-varparm}{Maccess},
    \refgsainst{update-varparm}{Mupdate},
    \refgsainst{update-nonlocal}{Mupdate}
  \end{seealso}
\end{instruction}



%%%%%%%%%%%%%%%%%%%%%%%%%%%%%%%%%%%%%%%%%%%%%%%%%%%%%%%%%%%%%%%%%%%%%%%%%%%%%%
\section{Region}\label{class:region}

In \ac{gsa} form, a \emph{region} is a data structure into which
instructions can be written.  These regions model such things as:
\begin{itemize}
\item procedure prologue
\item procedure epilogue
\item the body of an \code{IF} statement
\end{itemize}

Every \ac{gsa} instruction emitted by the compiler will be placed into
a region; the type of region is dependent upon the context of the
original source.

\begin{table}[h!]
  \begin{tabularx}{\linewidth}{|l|X|}
    \hline Instruction & Description \\
    \hline \gsainst{greg} & Global region; all code of every procedure
    is ultimately contained in a global region. \\

    \hline \gsainst{reg} & A region in which instructions can be emitted. \\

    \hline \gsainst{casereg} & A region which models Oberon
    \code{CASE} statements. \\

    \hline \gsainst{enter} & A region which models the procedure
    prologue.  Any instructions which must be executed as part of the
    prologue will be placed into this region. \\
    \hline \gsainst{exit} & A region which models the procedure
    epilogue.  Any instructions which must be executed as part of the
    epilogue will be placed into this region. \\
    \hline
  \end{tabularx}
\caption{Region Instructions}\label{tab:instruction-region}
\end{table}

Table \ref{tab:instruction-region} shows all the instructions which
are regions.

%-----------------------------------------------------------------------------
\begin{instruction}{casereg}
  \info{Case}{Mregion}{Mcasereg}{Models an Oberon \code{CASE}
    statement}{Mcasereg.Mod}{casereg}{Mregion}
  \begin{notes}
    \gsainst{casereg} is the controlling region for all the selector
    regions for a \code{CASE} statement.  It will contain
    \refgsainst{casesgl}{Gguard}, \refgsainst{caseelse}{Gguard} and
    \refgsainst{case}{Gmerge} instructions.

    No other instructions are allowed to appear in the body of a
    \gsainst{casereg}.
  \end{notes}

  \nresults

  \begin{operands}
  \item \operand{Usable}{The value of the selector expression for the
      \code{CASE} statement.}
  \end{operands}

  \begin{seealso}
    \refgsainst{casesgl}{Gguard}, \refgsainst{caseelse}{Gguard},
    \refgsainst{case}{Gmerge}.
  \end{seealso}
\end{instruction}

%-----------------------------------------------------------------------------
\begin{instruction}{enter}
  \info{Instruction}{Mregion}{Menter}{Models the procedure prolog and
    will be present as the first instruction of any
    procedure.}{Menter.Mod}{enter}{Mregion}

  \begin{notes}
    The enter instruction synthesizes the initial values for all
    parameters, nonlocal and local variables through the list of
    results.  The code generator can easily provide initial values for
    local variables by a traversal of the result list.

    The \gsainst{nlmctor} is modeled as an operand for two main
    reasons:
    \begin{enumerate}
    \item It is \emph{bad} to place the result of an instruction onto
      the result list of another instruction.  In this case, the
      \gsainst{nlmctor} would appear on two result lists.
    \item It provides a fast and reliable method to determine if a
      procedure which is to be call actually requires nonlocal
      variables to be placed back into their home memory locations.
      That is, if \gsavar{nlm} is present on the callee's
      \gsainst{enter} operand list, then nonlocals must be written to
      their home memory locations.
    \end{enumerate}

    Modeling all initial values of variables through the
    \gsainst{enter} instruction happens to be a convenient way of
    ensuring that most \code{Addressable} items are converted into
    \code{Result} values prior to their first use.

    It is convenient to say that \gsainst{enter} produces all the
    result values.

    All instructions which are part of the procedure prologue will be
    placed into the \gsainst{enter} region.

    The results are in no particular order.
  \end{notes}

  \begin{results}
  \item By-Value Integral Parameter
    \begin{enumerate}
    \item A result for each by-value integral parameter.
    \end{enumerate}

  \item By-reference Integral Parameter
    \begin{enumerate}
    \item A result modeling the current value of each \byref formal
      parameter.
    \item A result modeling the address of the formal parameter.
    \end{enumerate}

  \item By-reference Record Parameter
    \begin{enumerate}
    \item A result modeling the value of the formal parameter.
    \item A result modeling the address of the formal parameter.
    \item A result modeling the value of the type descriptor for the
      formal parameter.
    \item A result modeling the address of the type descriptor for the
      formal parameter.
    \end{enumerate}

  \item Open Array Parameter
    \begin{enumerate}
    \item A result modeling the value of the formal parameter
    \item A result modeling \code{LEN(${dim}_0$)}
    \item A result modeling \code{LEN(${dim}_1$)}
    \item A result modeling \code{LEN(${dim}_2$)}
    \item $\vdots$
    \item A result modeling \code{LEN(${dim}_{n-1}$)}
    \end{enumerate}

  \item Non-local Data
    \begin{enumerate}
    \item A result modeling the value of the non-local data
    \item A result modeling the address of the non-local data.
    \end{enumerate}
  \end{results}

  \begin{operands}
  \item An optional initial value of \gsavar{nlm} for the procedure;
    if the procedure does not need to model non-local variable memory,
    then this operand will not be present.
  \end{operands}

  \begin{seealso}
    \refgsainst{greg}{Mregion}, \refgsainst{exit}{Mregion}
  \end{seealso}
\end{instruction}

%-----------------------------------------------------------------------------
\begin{instruction}{exit}
  \info{Instruction}{Mregion}{Mexit}{Models the procedure epilogue and
    will be present as the last instruction of any procedure.}{Mexit.Mod}{exit}{Mregion}

  \begin{notes}
    The instructions contained in \gsainst{exit} model the procedure
    epilogue, and with no exceptions, are always to be considered
    live.

    The operands of the \gsainst{exit} instruction model assignments
    to items which have \emph{non-local effect}, such as globals,
    \byref paramters and non-local variables.  The code generated for
    \gsainst{exit} will ensure that the current value of all non-local
    items will be stored back to their home memory location.
    \todo{There is no code which adds NLE operands to the exit
      instruction.}
  \end{notes}

  \nresults

  \voperands{
    \begin{enumerate}
    \item The result of \refgsainst{nlmdtor}{Mmemory}.
    \end{enumerate}

    The remaining number of operands for \gsainst{exit} varies with
    the number of assignments which have non-local effect.}

  \begin{seealso}
    \refgsainst{greg}{Mregion}, \refgsainst{enter}{Mregion}
  \end{seealso}
\end{instruction}

%-----------------------------------------------------------------------------
\begin{instruction}{greg}
  \info{GlobalRegion}{Mregion}{Mgreg}{Models an entire
    procedure}{Mgreg.Mod}{greg}{Mregion}

  \begin{notes}
    \gsainst{greg} is a specialization of \gsainst{reg}: it contains
    references to the procedure prologue (\gsainst{enter}) and
    epilogue (\gsainst{exit}) regions.  However, in all other aspects
    it is identical to \gsainst{reg}.

    \gsainst{greg} models the entire body of each Oberon procedure
    emitted by the compiler.
  \end{notes}

  \nresults

  \begin{operands}
  \item There are no operands for a \gsainst{greg}
  \end{operands}

  \begin{seealso}
    \refgsainst{reg}{Mregion}, \refgsainst{enter}{Mregion},
    \refgsainst{exit}{Mregion}
  \end{seealso}
\end{instruction}

%-----------------------------------------------------------------------------
\begin{instruction}{reg}
  \info{Region}{Mregion}{Mreg}{Models a region into which \ac{gsa}
    instructions can be placed and
    manipulated.}{none}{reg}{Mregion}

  \begin{notes}
    \gsainst{reg} implements a base class for use as a foundation for
    other types of regions.  This type is not used explicitly in the
    compiler for anything other than type extension or specialization.
  \end{notes}

  \nresults

  \begin{operands}
  \item There are no operands for \gsainst{reg}.
  \end{operands}

  \begin{seealso}
    \refgsainst{greg}{Mregion}
  \end{seealso}
\end{instruction}



%%%%%%%%%%%%%%%%%%%%%%%%%%%%%%%%%%%%%%%%%%%%%%%%%%%%%%%%%%%%%%%%%%%%%%%%%%%%%%
\section{Set Arithmetic}\label{class:set-arithmetic}

The set arithmetic instructions model operations on bitsets, which
consists of bitwise operations.  For Oberon, this entails only 32-bit
unsigned quantities.

\note{
    (* 2000.09.22:
     * SET-specific instructions
     * On machines without bit instructions, these are too
     * high level to be useful, since it involves shifting 1 left
     * a number of places and providing the proper bitwise instruction
     * to operate on that particular bit.
     * Further, if bitsets are extended to be larger than the wordsize
     * of the machine, these two instructions can hide other instructions
     * (through their high-level-ness) which the improvement algorithms
     * may be able to improve (by finding congruent instructions elsewhere)
     *
     * For now, the instruction sequence will assume the precense of bit
     * instructions, and if it is ported to another architecture or if
     * SET sizes altered, then these code sequences should be revisited.
     *)
}
\begin{table}[h!]
  \begin{tabularx}{\linewidth}{|l|X|}
    \hline Instruction & Description \\
    \hline \gsainst{set-convert} & Set constructor; individual element \\
    \hline \gsainst{set-diff} & Symmetric Difference \\
    \hline \gsainst{set-element} & $\in$ test\\
    \hline \gsainst{set-excl} & Exclusion \\
    \hline \gsainst{set-incl} & Inclusion \\
    \hline \gsainst{set-intersect} & Intersection \\
    \hline \gsainst{set-neg} & Negation \\
    \hline \gsainst{set-range} & Set constructor; integer range \\
    \hline \gsainst{set-sub} & Subtraction \\
    \hline \gsainst{set-union} & Union \\
    \hline
  \end{tabularx}
\caption{Set Arithmetic Instructions}\label{tab:instruction-set}
\end{table}

%-----------------------------------------------------------------------------
\begin{instruction}{set-convert}
  \info{Instruction}{Sconvert}{asU4}{Models set construction from a
    single integer value.}{Sconvert.Mod}{set-convert}{sarithmetic}

  \begin{notes}
    \inv{$bit \in \{0..31\}$}
  \end{notes}

  \begin{results}
  \item $2^\textrm{bit}$
  \end{results}

  \begin{operands}
  \item \operand{Usable}{The integer to convert into a \code{SET} value.}
  \end{operands}

  \begin{seealso}
    \refgsainst{set-element}{sarithmetic}
  \end{seealso}
\end{instruction}

%-----------------------------------------------------------------------------
\begin{instruction}{set-diff}
  \info{Instruction}{Sdiff}{asU4}{Models \emph{symmetric set
      difference}.}{Sdiff.Mod}{set-diff}{sarithmetic}

  \begin{notes}
    Symmetric set difference can be represented in logical notation as
    $\code{a / b} \equiv (a \wedge \neg b) \vee (\neg a \wedge b)$
    which is nothing more than a fancy representation for
    \emph{exclusive or} $\code{a / b} \equiv a \xor b$.
  \end{notes}

  \begin{results}
  \item $\textrm{op}_0 \xor \textrm{op}_1$
  \end{results}

  \begin{operands}
  \item \operand{Usable}{The value of the left operand from the source.}
  \item \operand{Usable}{The value of the right operand from the source.}
  \end{operands}

  \begin{seealso}
    \refgsainst{set-sub}{sarithmetic}
    \refgsainst{set-union}{sarithmetic}
    \refgsainst{set-intersect}{sarithmetic}
  \end{seealso}
\end{instruction}

%-----------------------------------------------------------------------------
\begin{instruction}{set-element}
  \info{Instruction}{Selement}{asU4}{Models a \emph{set existence
      test}.}{Selement.Mod}{set-element}{sarithmetic}

  \begin{notes}
    \inv{$bit \in \{0..31\}$}
  \end{notes}

  \begin{results}
  \item $\textrm{bit} \in \textrm{set}$
  \end{results}

  \begin{operands}
  \item \operand{Usable}{The element to test for existence.}
  \item \operand{Usable}{The set which is to be tested.}
  \end{operands}

  \begin{seealso}
    \refgsainst{set-incl}{sarithmetic}
    \refgsainst{set-excl}{sarithmetic}
    \refgsainst{set-convert}{sarithmetic}
  \end{seealso}
\end{instruction}

%-----------------------------------------------------------------------------
\begin{instruction}{set-excl}
  \info{Instruction}{Sexcl}{asU4}{Models \emph{set
      exclusion}.}{Sexcl.Mod}{set-excl}{sarithmetic}

  \begin{results}
  \item $\textrm{set} \wedge \neg (2^\textrm{value})$
  \end{results}

  \begin{operands}
  \item \operand{Usable}{The set value which should be modified.}
  \item \operand{Usable}{The element number which should be removed
      from the set.}
  \end{operands}

  \begin{seealso}
    \refgsainst{set-sub}{sarithmetic}
    \refgsainst{set-incl}{sarithmetic}
    \refgsainst{set-element}{sarithmetic}
    \refgsainst{set-neg}{sarithmetic}
  \end{seealso}
\end{instruction}

%-----------------------------------------------------------------------------
\begin{instruction}{set-incl}
  \info{Instruction}{Sincl}{asU4}{Models \emph{set
      inclusion}.}{Sincl.Mod}{set-incl}{sarithmetic}

  \begin{notes}
    \inv{$bit \in \{0..31\}$}
  \end{notes}

  \begin{results}
  \item $\textrm{set} \vee {2}^\textrm{value}$
  \end{results}

  \begin{operands}
  \item \operand{Usable}{The set which is to be modified.}
  \item \operand{Usable}{The element number which should be included
      in the set.}
  \end{operands}

  \begin{seealso}
    \refgsainst{set-excl}{sarithmetic}
    \refgsainst{set-union}{sarithmetic}
    \refgsainst{set-element}{sarithmetic}
  \end{seealso}
\end{instruction}

%-----------------------------------------------------------------------------
\begin{instruction}{set-intersect}
  \info{Instruction}{Sintersection}{asU4}{Models \emph{set
      intersection}.}{Sintersection.Mod}{set-intersect}{sarithmetic}

  \begin{notes}
    $$\code{a * b} \equiv (a \cap b) \equiv (a \wedge b)$$
  \end{notes}

  \begin{results}
  \item $\textrm{op}_0 \wedge \textrm{op}_1$
  \end{results}

  \begin{operands}
  \item \operand{Usable}{The value of the left operand from the source.}
  \item \operand{Usable}{The value of the right operand from the source.}
  \end{operands}

  \begin{seealso}
    \refgsainst{set-sub}{sarithmetic}
    \refgsainst{set-union}{sarithmetic}
    \refgsainst{set-diff}{sarithmetic}
  \end{seealso}
\end{instruction}

%-----------------------------------------------------------------------------
\begin{instruction}{set-neg}
  \info{Instruction}{Sneg}{asU4}{Models \emph{set
      negation}.}{Sneg.Mod}{set-neg}{sarithmetic}

  \begin{results}
  \item $\neg \textrm{op}$
  \end{results}

  \begin{operands}
  \item \operand{Usable}{The value which should be negated.}
  \end{operands}

  \begin{seealso}
    \refgsainst{set-sub}{sarithmetic}
    \refgsainst{set-excl}{sarithmetic}
  \end{seealso}
\end{instruction}

%-----------------------------------------------------------------------------
\begin{instruction}{set-range}
  \info{Instruction}{Srange}{asU4}{Models \emph{set construction} from
    a range of integer values.}{Srange.Mod}{set-range}{sarithmetic}

  \begin{notes}
    This instruction will construct a set low..high.

    \inv{$\textrm{low} <= \textrm{high}$}

    \inv{$\textrm{low} \in \{0..31\}$}

    \inv{$\textrm{high} \in \{0..31\}$}
  \end{notes}

  \begin{results}
  \item $\{2^\textrm{low}..2^\textrm{high}\}$
  \end{results}

  \begin{operands}
  \item \operand{Usable}{The low bound of the range to construct.}
  \item \operand{Usable}{The high bound of the range to construct.}
  \end{operands}
\end{instruction}

%-----------------------------------------------------------------------------
\begin{instruction}{set-sub}
  \info{Instruction}{Ssub}{asU4}{Models \emph{set
      difference}.}{Ssub.Mod}{set-sub}{sarithmetic}

  \begin{notes}
    $$\code{a - b} \equiv (a \cap (-b)) \equiv (a \wedge (-b))$$
  \end{notes}

  \begin{results}
  \item $\textrm{op}_0 \wedge (-\textrm{op}_1)$
  \end{results}

  \begin{operands}
  \item \operand{Usable}{The value of the left operand from the source.}
  \item \operand{Usable}{The value of the right operand from the source.}
  \end{operands}

  \begin{seealso}
    \refgsainst{set-diff}{sarithmetic}
    \refgsainst{set-union}{sarithmetic}
    \refgsainst{set-intersect}{sarithmetic}
    \refgsainst{set-neg}{sarithmetic}
    \refgsainst{set-excl}{sarithmetic}
  \end{seealso}
\end{instruction}

%-----------------------------------------------------------------------------
\begin{instruction}{set-union}
  \info{Instruction}{Sunion}{asU4}{Models \emph{set
      union}.}{Sunion.Mod}{set-union}{sarithmetic}

  \begin{notes}
    $$\code{a * b} \equiv (a \cup b) \equiv (a \vee b)$$
  \end{notes}

  \begin{results}
  \item $\textrm{op}_0 \vee \textrm{op}_1$
  \end{results}

  \begin{operands}
  \item \operand{Usable}{The value of the left operand from the source.}
  \item \operand{Usable}{The value of the right operand from the source.}
  \end{operands}

  \begin{seealso}
    \refgsainst{set-sub}{sarithmetic}
    \refgsainst{set-incl}{sarithmetic}
    \refgsainst{set-intersect}{sarithmetic}
    \refgsainst{set-diff}{sarithmetic}
  \end{seealso}
\end{instruction}



%%%%%%%%%%%%%%%%%%%%%%%%%%%%%%%%%%%%%%%%%%%%%%%%%%%%%%%%%%%%%%%%%%%%%%%%%%%%%%
\section{System}\label{class:system}
\begin{table}[h!]
  \begin{tabularx}{\linewidth}{|l|X|}
    \hline Instruction & Description \\
    \hline \gsainst{bit} & \code{SYSTEM.BIT} \\
    \hline \gsainst{newblock} & \code{SYSTEM.NEW} \\
    \hline \gsainst{move} & \code{SYSTEM.MOVE}\\
    \hline \gsainst{resetbit} & \code{SYSTEM.BITR}\\
    \hline \gsainst{setbit} & \code{SYSTEM.BITS} \\
    \hline \gsainst{finalize} & \code{SYSTEM.FINALIZE} \\
    \hline \gsainst{condcode} & \code{SYSTEM.CC} \\
    \hline
  \end{tabularx}
\caption{System Instructions}\label{tab:instruction-system}
\end{table}

%-----------------------------------------------------------------------------
\begin{instruction}{bit}
  \info{Instruction}{Gsystem}{Gbit}{Models
    \code{SYSTEM.BIT}}{Gbit.Mod}{bit}{Gsystem}

  \begin{results}
  \item The result is the value of the bit tested.
  \end{results}

  \begin{operands}
  \item \operand{Usable}{The address of the memory to test.}
  \item \operand{Usable}{The constant value of the bit to test.
      \inv{$\textrm{bit} \in {0..8}$}}
  \end{operands}
\end{instruction}

%-----------------------------------------------------------------------------
\begin{instruction}{condcode}
  \info{Instruction}{Gmisc}{Gcondcode}{Models
    \code{SYSTEM.CC}}{Gcondcode.Mod}{condcode}{Gsystem}

  \begin{results}
  \item The result is the value of the specified condition code.
  \end{results}

  \begin{operands}
  \item \operand{ConstInt}{The hardware condition code which is to be
      tested.}
  \end{operands}
\end{instruction}

%-----------------------------------------------------------------------------
\begin{instruction}{finalize}
  \info{Instruction}{Gsystem}{Gfinalize}{Models
    \code{SYSTEM.FINALIZE}}{Gfinalize.Mod}{finalize}{Gsystem}

  \nresults

  \begin{operands}
  \item \operand{Symbol}{The \ac{td} for the record type in which a
      finalization procedure is being installed.}
  \item \operand{Result}{The address of the finalization procedure.}
  \end{operands}
\end{instruction}

%-----------------------------------------------------------------------------
\begin{instruction}{move}
  \info{Instruction}{Gsystem}{Gmove}{Models
    \code{SYSTEM.MOVE}}{Gmove.Mod}{move}{Gsystem}

  \begin{notes}
    This instruction does not handle overlapping memory regions.

    All the operands are \code{LONGINT} values.
  \end{notes}

  \nresults

  \begin{operands}
  \item \operand{Usable}{The source address.}
  \item \operand{Usable}{The destination address.}
  \item \operand{Usable}{The number of bytes to move.}
  \end{operands}
\end{instruction}

%-----------------------------------------------------------------------------
\begin{instruction}{newblock}
  \info{Instruction}{Gsystem}{Gnewblock}{Models
    \code{SYSTEM.NEW}}{Gnewblock.Mod}{newblock}{Gsystem}

  \begin{notes}
    This instruction always returns a unique result.

    The number of bytes to allocate will be a \code{LONGINT} value.
  \end{notes}

  \begin{results}
  \item The result is the address of a newly allocated block which
    will not participate in \ac{gc}.
  \end{results}

  \begin{operands}
  \item \operand{Usable}{The number of bytes to allocate.}
  \end{operands}
\end{instruction}

%-----------------------------------------------------------------------------
\begin{instruction}{resetbit}
  \info{Instruction}{Gsystem}{Gresetbit}{Models
    \code{SYSTEM.BITR}}{Gresetbit.Mod}{resetbit}{Gsystem}

  \nresults

  \begin{operands}
  \item \operand{Usable}{The address of the memory containing the bit to reset.}
  \item \operand{Usable}{The constant value of the bit to reset.
        \inv{$\textrm{bit} \in {0..8}$}}
  \end{operands}
\end{instruction}

%-----------------------------------------------------------------------------
\begin{instruction}{setbit}
  \info{Instruction}{Gsystem}{Gsetbit}{Models
    \code{SYSTEM.BITS}}{Gsetbit.Mod}{setbit}{Gsystem}

  \nresults

  \begin{operands}
  \item \operand{Usable}{The address of the memory containing the bit to set.}
  \item \operand{Usable}{The constant value of the bit to set.
        \inv{$\textrm{bit} \in {0..8}$}}
  \end{operands}
\end{instruction}



%%%%%%%%%%%%%%%%%%%%%%%%%%%%%%%%%%%%%%%%%%%%%%%%%%%%%%%%%%%%%%%%%%%%%%%%%%%%%%
\section{Trap}\label{class:trap}

The \emph{trap} class of instructions model various types of a
\emph{premature program halt}.  This instructions are inserted by the
compiler to implicitly check the validity of the program being
translated; these checks ensure that an erroneous program does not
continue to execute.

Trap instructions will be marked \emph{always live} and therefore
cannot be removed by any \ac{cia}.

Traps are implemented as simple \code{Instruction} records in the
\ac{ir}.

\begin{table}[h!]
  \begin{tabularx}{\linewidth}{|l|X|}
    \hline Instruction & Description \\
    \hline \gsainst{index} & Index out-of-bounds. \\
    \hline \gsainst{range} & Calculation result out-of-range. \\
    \hline \gsainst{eguard} & Explicit type guard failure. \\
    \hline \gsainst{iguard} & Implicit type guard failure.\\
    \hline \gsainst{return} & Function procedure ends without \code{RETURN} \\
    \hline \gsainst{merge-case} & No match for \code{CASE} selections. \\
    \hline \gsainst{assert} & \code{ASSERT} failure. \\
    \hline \gsainst{halt} & \code{HALT} \\
    \hline \gsainst{with} & No match for \code{WITH} selections. \\
    \hline
  \end{tabularx}
\caption{Trap Instructions}\label{tab:instruction-trap}
\end{table}

%-----------------------------------------------------------------------------
\begin{instruction}{assert}
  \info{Instruction}{Ttrap}{Tassert}{This instruction models the
    result of an assertion failure.}{Tassert.Mod}{trap-assert}{Ttrap}

  \nresults

  \noperands
\end{instruction}

%-----------------------------------------------------------------------------
\begin{instruction}{case}
  \info{Instruction}{Ttrap}{Tcase}{Models a \emph{no match in
      \code{CASE} statement} halt.}{Tcase.Mod}{trap-case}{Ttrap}

  \nresults

  \noperands
\end{instruction}

%-----------------------------------------------------------------------------
\begin{instruction}{eguard}
  \info{Instruction}{Ttrap}{Teguard}{Models an \emph{explicit type
      guard failure} halt.}{Teguard.Mod}{trap-eguard}{Ttrap}

  \nresults

  \noperands
\end{instruction}

%-----------------------------------------------------------------------------
\begin{instruction}{halt}
  \info{Instruction}{Ttrap}{Thalt}{Models
    \code{HALT}.}{Thalt.Mod}{trap-halt}{Ttrap}

  \nresults

  \begin{operands}
  \item \operand{ConstInt}{The value used as the \emph{halt value}.}
  \end{operands}
\end{instruction}

%-----------------------------------------------------------------------------
\begin{instruction}{iguard}
  \info{Instruction}{Ttrap}{Tiguard}{Models an \emph{implicit type
      guard failure} halt.}{Tiguard.Mod}{trap-iguard}{Ttrap}

  \nresults

  \noperands
\end{instruction}

%-----------------------------------------------------------------------------
\begin{instruction}{index}
  \info{Instruction}{Ttrap}{Tindex}{Models an \emph{out-of-bounds}
    index value halt.}{Tindex.Mod}{trap-index}{Ttrap}

  \nresults

  \noperands
\end{instruction}

%-----------------------------------------------------------------------------
\begin{instruction}{range}
  \info{Instruction}{Ttrap}{Trange}{Models \emph{calculation overflow}
    and \emph{calculation underflow} halt.}{Trange.Mod}{trap-range}{Ttrap}

  \nresults

  \noperands
\end{instruction}

%-----------------------------------------------------------------------------
\begin{instruction}{return}
  \info{Instruction}{Ttrap}{Treturn}{Models a \emph{function procedure
      without \code{RETURN}} halt.}{Treturn.Mod}{trap-return}{Ttrap}

  \nresults

  \noperands
\end{instruction}

%-----------------------------------------------------------------------------
\begin{instruction}{with}
  \info{Instruction}{Ttrap}{Twith}{Models a \emph{no match in
      \code{WITH} statement} halt.}{Twith.Mod}{trap-with}{Ttrap}

  \nresults

  \noperands
\end{instruction}


%%%%%%%%%%%%%%%%%%%%%%%%%%%%%%%%%%%%%%%%%%%%%%%%%%%%%%%%%%%%%%%%%%%%%%%%%%%%%%
\section{Type Descriptor}\label{class:td}
\begin{table}[h!]
  \begin{tabularx}{\linewidth}{|l|X|}
    \hline Instruction & Description \\
    \hline \gsainst{initrec} & Initializes a record type descriptor.\\
    \hline \gsainst{initarr} & Initializes a static array type descriptor. \\
    \hline \gsainst{initdarr} & Initialize a dynamic array type
    descriptor. \\
    \hline
  \end{tabularx}
\caption{Type Descriptor Instructions}\label{tab:instruction-td}
\end{table}

%-----------------------------------------------------------------------------
\begin{instruction}{initarr}
  \info{Instruction}{Gtypedesc}{Ginitarr}{Models the initialization of
    fields in a statically-sized array
    \ac{td}.}{Ginitarr.Mod}{initarr}{Gtypedesc}

  \begin{notes}
    If no initialization is required a runtime, then this instruction
    is a \emph{nop}, and it is assumed that the type descriptor is
    initialized with generated data and relocations provided by the
    linker or loader.
  \end{notes}

  \begin{results}
  \item An initialized copy fo the \ac{td}.
  \end{results}

  \begin{operands}
  \item \operand{Symbol}{The type symbol which is to have its type
      descriptor initialized.}
  \end{operands}

  \begin{seealso}
    \refgsainst{initialize}{Mmemory}, \refgsainst{initrec}{Gtypedesc},
    \refgsainst{initdarr}{Gtypedesc}
  \end{seealso}
\end{instruction}

%-----------------------------------------------------------------------------
\begin{instruction}{initdarr}
  \info{Instruction}{Gtypedesc}{Ginitdarr}{Models the initialization
    of fields in a statically-sized array
    \ac{td}.}{Ginitdarr.Mod}{initdarr}{Gtypedesc}

  \begin{notes}
    If no initialization is required a runtime, then this instruction
    is a \emph{nop}, and it is assumed that the type descriptor is
    initialized with generated data and relocations provided by the
    linker or loader.
  \end{notes}

  \begin{results}
  \item An initialized copy fo the \ac{td}.
  \end{results}

  \begin{operands}
  \item \operand{Symbol}{The type symbol which is to have its type
      descriptor initialized.}
  \end{operands}
  \begin{seealso}
    \refgsainst{initialize}{Mmemory}, \refgsainst{initrec}{Gtypedesc},
    \refgsainst{initarr}{Gtypedesc}
  \end{seealso}
\end{instruction}

%-----------------------------------------------------------------------------
\begin{instruction}{initrec}
  \info{Instruction}{Gtypedesc}{Ginitrec}{Models the initialization of
    fields in a record \ac{td} which are not set by the compiler or
    linker.}{Ginitrec.Mod}{initrec}{Gtypedesc}

  \begin{notes}
    If no initialization is required a runtime, then this instruction
    is a \emph{nop}, and it is assumed that the type descriptor is
    initialized with generated data and relocations provided by the
    linker or loader.
  \end{notes}

  \begin{results}
  \item An initialized copy of the \ac{td}.
  \end{results}

  \begin{operands}
  \item \operand{Symbol}{The type symbol which is to have its type
      descriptor initialized.}
  \end{operands}

  \begin{seealso}
    \refgsainst{initialize}{Mmemory}, \refgsainst{initarr}{Gtypedesc},
    \refgsainst{initdarr}{Gtypedesc}
  \end{seealso}
\end{instruction}
