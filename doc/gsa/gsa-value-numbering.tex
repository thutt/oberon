% Copyright (c) 2001-2023 Logic Magicians Software
\newcommand{\vnopclass}[1] {\emph{opcode}[#1]\xspace}
\chapter{Value Numbering}
\begin{quote}
  \ac{vn}, also known as \ac{cse}, is an algorithm which assigns each
  instruction in a region a \emph{number} based on the \emph{value} it
  computes.

  By design, expressions which compute the same \emph{value} receive
  the same \emph{number}.

  Following the calculation of \emph{value numbers}, the algorithm
  replaces all uses of the same \emph{value number} by the result of
  the first instruction which computes that value; all other
  instructions which compute the same value become \emph{dead} and
  will be removed by the \ac{dce} algorithm.
\end{quote}

This compiler uses an \emph{optimistic} \ac{vn} algorithm which is
designed to find two instructions to be congruent until it can be
proven that they are not congruent.

\begin{algorithm}[h!]
  \caption{Value Numbering (core)}
  \begin{algorithmic}[1]
    \REQUIRE{greg $\neq$ nil}

    \STATE{worklist $=$ nil}
    \FORALL{i: i is an opcode}
      \STATE create parition \vnopclass{i}
      \STATE add \vnopclass{i} to worklist
    \ENDFOR

    \FORALL{region: region $\in$ greg}
      \FORALL{inst: inst $\in$ region}
      \STATE add inst to partition \vnopclass{inst}
      \ENDFOR
    \ENDFOR

    \FORALL{i: i is an opcode}
      \STATE split \vnopclass{i} by \emph{unique result}
      \STATE split \vnopclass{i} by \emph{loops}
      \STATE split \vnopclass{i} by \emph{const}
      \STATE split \vnopclass{i} by \emph{addressable}
    \ENDFOR

    \FORALL{p: p $\in$ worklist}
    \STATE split p by \emph{results} and \emph{operands}
    \ENDFOR

    \STATE coalesce \ac{vn}
  \end{algorithmic}
\end{algorithm}

In basic terms, the algorithm functions by iteratively splitting the
instructions in a region into partitions, based on the opcode and the
operands, until all instructions in a given partition have the same
opcode and all the operands are equivalent.

Partitioning of a region occurs based on the results and operands of
each instruction.  When complete, each partition will only contain
instructions with the same \emph{opcode}, \emph{results} and congruent
\emph{operands} in the same position.

\section{Special Splitting Criteria}

The algorithm must take a few special features into consideration, or
the value numbering which is produced will be incorrect.

First, instructions, such as \gsainst{newrecord}, can be marked as
\emph{always returning a unique result}.  Instructions marked in this
manner must be split into their own partitions so that the scenario
shown in figure \ref{fig:unique-results} will not have all uses of
these instructions collapsed into the first.

\begin{figure}[h!]
\begin{verbatim}
   NEW(p);
   p.x := 0;
   NEW(p);
\end{verbatim}
  \caption{Partitioning \emph{unique results}}\label{fig:unique-results}
\end{figure}

Further, since this compiler only has one \gsainst{merge-loop} type,
all looping constructs must be explicitly partitioned into their own
partitions to prevent all loops from being considered congruent.

\section{Partitioning by Operands}
It can be observed that the operands of any instruction are going to
fall into three (3) categories which make it easy to split partitions
based on operands.

\todo{Diagrams or better descriptions would be nice here.}
\begin{enumerate}
\item The result of an instruction

  An instruction which is not used as an operand of another
  instruction is \emph{dead}, and not interesting for \ac{vn}.

  An instruction which is used will appear as the operand of at least
  one other instruction.  The algorithm is designed to traverse the
  \emph{uses} of an instruction and split each use based on the
  operand position.  In other words, all \emph{uses} as ${operand}_0$
  will be split, all \emph{uses} as ${operand}_1$ will be split, and
  so forth.

\item An addressable item which is not a constant

  The compiler keeps a list of all \emph{addressable} items used in
  each global region, which are in turned linked through the
  \emph{def-uses} chain.  This chain is traversed and the partitions
  are split in the same way as for \emph{instruction results}.

\item A constant value

  The compiler keeps a list of all \emph{constant} items used in each
  global region, which are in turned linked through the \emph{uses}
  chain.  This chain is traversed and the partitions are split in the
  same way as for \emph{instruction results}.

\end{enumerate}

\section{Operand Replacement}

\begin{algorithm}[h!]
  \caption{Value Numbering (coalesce)}
  \begin{algorithmic}[1]
    \REQUIRE greg $\neq$ nil
    \FORALL{region: region $\in$ greg}
      \FORALL{inst: inst $\in$ region}
        \IF{$\neg$ (\emph{partition-replacement} of \emph{partition}(inst))}
          \STATE \emph{partition-replacement} $=$ inst
        \ELSE
          \FORALL{result: result $\in$ \emph{results-of}(inst)}
             \FORALL{use-of result}
               \STATE replace result by \emph{partition-replacement}
             \ENDFOR
          \ENDFOR
        \ENDIF
      \ENDFOR
    \ENDFOR
  \end{algorithmic}
\end{algorithm}

Since the core of the algorithm matches operands in a particular
position, it is assumed that operands have been be normalized to
ensure that all opportunities for finding congruence are available.
For example, the core algorithm will not find the following
instructions to be congruent:

\begin{figure}
\begin{verbatim}
  i := j + 5;
  k := 5 + j;
\end{verbatim}
\caption{Congruence will not be found}
\end{figure}

However it will find congruence if the operands are normalized, such
as the following:

\begin{figure}
\begin{verbatim}
  i := j + 5;
  k := j + 5;
\end{verbatim}
\caption{Congruence will be found}
\end{figure}

This \ac{vn} system does not normalize the operands of the
instructions.

An instruction which produces a \emph{unique result}, such as
\code{NEW}, can never be considered congruent with any other
instruction.  Consequently, a pass over all the partitions is required
to ensure that \emph{unique result} instructions are placed into their
own partition.

Similarly, \gsainst{merge-loop} can never be congruent with any other
instruction, so they are placed into unique paritions.
